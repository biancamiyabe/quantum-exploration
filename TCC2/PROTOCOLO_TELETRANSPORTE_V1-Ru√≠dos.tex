%%%
\documentclass[12pt,oneside,brazil,hidelinks,a4paper]{abntex2}

\usepackage{amsmath}
\usepackage{amssymb}
\usepackage{mathtools}
\usepackage{indentfirst}
\usepackage{fontawesome5}

\setcounter{MaxMatrixCols}{20}


%\usepackage[math-style=TeX]{unicode-math}
%\setmathfont{Asana-Math.otf}[Scale=MatchLowercase]
\usepackage[euler-hat-accent,small]{eulervm}

\usepackage{polyglossia}
\setmainlanguage{portuges}
\setotherlanguage{english}
\usepackage{csquotes}
\usepackage{graphicx}  % Figuras em vários formatos (png, pdf)
\usepackage{color}     % Cores
\usepackage{pgfgantt}  % Cronogramas
\usepackage{multicol}  % Documentos com várias colunas
\usepackage{booktabs}  % Tabelas profissionais
\usepackage[final]{microtype}
\usepackage{siunitx}   % Pacote para unidades físicas (recomendo muito!)
\sisetup{output-decimal-marker={,}}
\sisetup{mode=text}
\usepackage{makecell}
\usepackage{listings}
\usepackage{xcolor}

    \usepackage[breakable]{tcolorbox}
    \usepackage{parskip} % Stop auto-indenting (to mimic markdown behaviour)
    

    % Basic figure setup, for now with no caption control since it's done
    % automatically by Pandoc (which extracts ![](path) syntax from Markdown).
    \usepackage{graphicx}
    % Maintain compatibility with old templates. Remove in nbconvert 6.0
    \let\Oldincludegraphics\includegraphics
    % Ensure that by default, figures have no caption (until we provide a
    % proper Figure object with a Caption API and a way to capture that
    % in the conversion process - todo).
    \usepackage{caption}
    \DeclareCaptionFormat{nocaption}{}
    \captionsetup{format=nocaption,aboveskip=0pt,belowskip=0pt}

    \usepackage{float}
    \floatplacement{figure}{H} % forces figures to be placed at the correct location
    \usepackage{xcolor} % Allow colors to be defined
    \usepackage{enumerate} % Needed for markdown enumerations to work
    \usepackage{geometry} % Used to adjust the document margins
    \usepackage{amsmath} % Equations
    \usepackage{amssymb} % Equations
    \usepackage{textcomp} % defines textquotesingle
    % Hack from http://tex.stackexchange.com/a/47451/13684:
    \AtBeginDocument{%
        \def\PYZsq{\textquotesingle}% Upright quotes in Pygmentized code
    }
    \usepackage{upquote} % Upright quotes for verbatim code
    \usepackage{eurosym} % defines \euro

    \usepackage{iftex}
    \ifPDFTeX
        \usepackage[T1]{fontenc}
        \IfFileExists{alphabeta.sty}{
              \usepackage{alphabeta}
          }{
              \usepackage[mathletters]{ucs}
              \usepackage[utf8x]{inputenc}
          }
    \else
        \usepackage{fontspec}
        \usepackage{unicode-math}
    \fi

    \usepackage{fancyvrb} % verbatim replacement that allows latex
    \usepackage{grffile} % extends the file name processing of package graphics
                         % to support a larger range
    \makeatletter % fix for old versions of grffile with XeLaTeX
    \@ifpackagelater{grffile}{2019/11/01}
    {
      % Do nothing on new versions
    }
    {
      \def\Gread@@xetex#1{%
        \IfFileExists{"\Gin@base".bb}%
        {\Gread@eps{\Gin@base.bb}}%
        {\Gread@@xetex@aux#1}%
      }
    }
    \makeatother
    \usepackage[Export]{adjustbox} % Used to constrain images to a maximum size
    \adjustboxset{max size={0.9\linewidth}{0.9\paperheight}}

    % The hyperref package gives us a pdf with properly built
    % internal navigation ('pdf bookmarks' for the table of contents,
    % internal cross-reference links, web links for URLs, etc.)
    \usepackage{hyperref}
    % The default LaTeX title has an obnoxious amount of whitespace. By default,
    % titling removes some of it. It also provides customization options.
    \usepackage{titling}
    \usepackage{longtable} % longtable support required by pandoc >1.10
    \usepackage{booktabs}  % table support for pandoc > 1.12.2
    \usepackage{array}     % table support for pandoc >= 2.11.3
    \usepackage{calc}      % table minipage width calculation for pandoc >= 2.11.1

  
                                % normalem makes italics be italics, not underlines
    \usepackage{mathrsfs}
    

    
    % Colors for the hyperref package
    \definecolor{urlcolor}{rgb}{0,.145,.698}
    \definecolor{linkcolor}{rgb}{.71,0.21,0.01}
    \definecolor{citecolor}{rgb}{.12,.54,.11}

    % ANSI colors
    \definecolor{ansi-black}{HTML}{3E424D}
    \definecolor{ansi-black-intense}{HTML}{282C36}
    \definecolor{ansi-red}{HTML}{E75C58}
    \definecolor{ansi-red-intense}{HTML}{B22B31}
    \definecolor{ansi-green}{HTML}{00A250}
    \definecolor{ansi-green-intense}{HTML}{007427}
    \definecolor{ansi-yellow}{HTML}{DDB62B}
    \definecolor{ansi-yellow-intense}{HTML}{B27D12}
    \definecolor{ansi-blue}{HTML}{208FFB}
    \definecolor{ansi-blue-intense}{HTML}{0065CA}
    \definecolor{ansi-magenta}{HTML}{D160C4}
    \definecolor{ansi-magenta-intense}{HTML}{A03196}
    \definecolor{ansi-cyan}{HTML}{60C6C8}
    \definecolor{ansi-cyan-intense}{HTML}{258F8F}
    \definecolor{ansi-white}{HTML}{C5C1B4}
    \definecolor{ansi-white-intense}{HTML}{A1A6B2}
    \definecolor{ansi-default-inverse-fg}{HTML}{FFFFFF}
    \definecolor{ansi-default-inverse-bg}{HTML}{000000}

    % common color for the border for error outputs.
    \definecolor{outerrorbackground}{HTML}{FFDFDF}

    % commands and environments needed by pandoc snippets
    % extracted from the output of `pandoc -s`
    \providecommand{\tightlist}{%
      \setlength{\itemsep}{0pt}\setlength{\parskip}{0pt}}
    \DefineVerbatimEnvironment{Highlighting}{Verbatim}{commandchars=\\\{\}}
    % Add ',fontsize=\small' for more characters per line
    \newenvironment{Shaded}{}{}
    \newcommand{\KeywordTok}[1]{\textcolor[rgb]{0.00,0.44,0.13}{\textbf{{#1}}}}
    \newcommand{\DataTypeTok}[1]{\textcolor[rgb]{0.56,0.13,0.00}{{#1}}}
    \newcommand{\DecValTok}[1]{\textcolor[rgb]{0.25,0.63,0.44}{{#1}}}
    \newcommand{\BaseNTok}[1]{\textcolor[rgb]{0.25,0.63,0.44}{{#1}}}
    \newcommand{\FloatTok}[1]{\textcolor[rgb]{0.25,0.63,0.44}{{#1}}}
    \newcommand{\CharTok}[1]{\textcolor[rgb]{0.25,0.44,0.63}{{#1}}}
    \newcommand{\StringTok}[1]{\textcolor[rgb]{0.25,0.44,0.63}{{#1}}}
    \newcommand{\CommentTok}[1]{\textcolor[rgb]{0.38,0.63,0.69}{\textit{{#1}}}}
    \newcommand{\OtherTok}[1]{\textcolor[rgb]{0.00,0.44,0.13}{{#1}}}
    \newcommand{\AlertTok}[1]{\textcolor[rgb]{1.00,0.00,0.00}{\textbf{{#1}}}}
    \newcommand{\FunctionTok}[1]{\textcolor[rgb]{0.02,0.16,0.49}{{#1}}}
    \newcommand{\RegionMarkerTok}[1]{{#1}}
    \newcommand{\ErrorTok}[1]{\textcolor[rgb]{1.00,0.00,0.00}{\textbf{{#1}}}}
    \newcommand{\NormalTok}[1]{{#1}}

    % Additional commands for more recent versions of Pandoc
    \newcommand{\ConstantTok}[1]{\textcolor[rgb]{0.53,0.00,0.00}{{#1}}}
    \newcommand{\SpecialCharTok}[1]{\textcolor[rgb]{0.25,0.44,0.63}{{#1}}}
    \newcommand{\VerbatimStringTok}[1]{\textcolor[rgb]{0.25,0.44,0.63}{{#1}}}
    \newcommand{\SpecialStringTok}[1]{\textcolor[rgb]{0.73,0.40,0.53}{{#1}}}
    \newcommand{\ImportTok}[1]{{#1}}
    \newcommand{\DocumentationTok}[1]{\textcolor[rgb]{0.73,0.13,0.13}{\textit{{#1}}}}
    \newcommand{\AnnotationTok}[1]{\textcolor[rgb]{0.38,0.63,0.69}{\textbf{\textit{{#1}}}}}
    \newcommand{\CommentVarTok}[1]{\textcolor[rgb]{0.38,0.63,0.69}{\textbf{\textit{{#1}}}}}
    \newcommand{\VariableTok}[1]{\textcolor[rgb]{0.10,0.09,0.49}{{#1}}}
    \newcommand{\ControlFlowTok}[1]{\textcolor[rgb]{0.00,0.44,0.13}{\textbf{{#1}}}}
    \newcommand{\OperatorTok}[1]{\textcolor[rgb]{0.40,0.40,0.40}{{#1}}}
    \newcommand{\BuiltInTok}[1]{{#1}}
    \newcommand{\ExtensionTok}[1]{{#1}}
    \newcommand{\PreprocessorTok}[1]{\textcolor[rgb]{0.74,0.48,0.00}{{#1}}}
    \newcommand{\AttributeTok}[1]{\textcolor[rgb]{0.49,0.56,0.16}{{#1}}}
    \newcommand{\InformationTok}[1]{\textcolor[rgb]{0.38,0.63,0.69}{\textbf{\textit{{#1}}}}}
    \newcommand{\WarningTok}[1]{\textcolor[rgb]{0.38,0.63,0.69}{\textbf{\textit{{#1}}}}}


    % Define a nice break command that doesn't care if a line doesn't already
    % exist.
    \def\br{\hspace*{\fill} \\* }
    % Math Jax compatibility definitions
    \def\gt{>}
    \def\lt{<}
    \let\Oldtex\TeX
    \let\Oldlatex\LaTeX
    \renewcommand{\TeX}{\textrm{\Oldtex}}
    \renewcommand{\LaTeX}{\textrm{\Oldlatex}}
    % Document parameters
    % Document title
    \title{PROTOCOLO\_TELETRANSPORTE\_V1-Ruídos}
    
    
    
    
    
% Pygments definitions
\makeatletter
\def\PY@reset{\let\PY@it=\relax \let\PY@bf=\relax%
    \let\PY@ul=\relax \let\PY@tc=\relax%
    \let\PY@bc=\relax \let\PY@ff=\relax}
\def\PY@tok#1{\csname PY@tok@#1\endcsname}
\def\PY@toks#1+{\ifx\relax#1\empty\else%
    \PY@tok{#1}\expandafter\PY@toks\fi}
\def\PY@do#1{\PY@bc{\PY@tc{\PY@ul{%
    \PY@it{\PY@bf{\PY@ff{#1}}}}}}}
\def\PY#1#2{\PY@reset\PY@toks#1+\relax+\PY@do{#2}}

\@namedef{PY@tok@w}{\def\PY@tc##1{\textcolor[rgb]{0.73,0.73,0.73}{##1}}}
\@namedef{PY@tok@c}{\let\PY@it=\textit\def\PY@tc##1{\textcolor[rgb]{0.24,0.48,0.48}{##1}}}
\@namedef{PY@tok@cp}{\def\PY@tc##1{\textcolor[rgb]{0.61,0.40,0.00}{##1}}}
\@namedef{PY@tok@k}{\let\PY@bf=\textbf\def\PY@tc##1{\textcolor[rgb]{0.00,0.50,0.00}{##1}}}
\@namedef{PY@tok@kp}{\def\PY@tc##1{\textcolor[rgb]{0.00,0.50,0.00}{##1}}}
\@namedef{PY@tok@kt}{\def\PY@tc##1{\textcolor[rgb]{0.69,0.00,0.25}{##1}}}
\@namedef{PY@tok@o}{\def\PY@tc##1{\textcolor[rgb]{0.40,0.40,0.40}{##1}}}
\@namedef{PY@tok@ow}{\let\PY@bf=\textbf\def\PY@tc##1{\textcolor[rgb]{0.67,0.13,1.00}{##1}}}
\@namedef{PY@tok@nb}{\def\PY@tc##1{\textcolor[rgb]{0.00,0.50,0.00}{##1}}}
\@namedef{PY@tok@nf}{\def\PY@tc##1{\textcolor[rgb]{0.00,0.00,1.00}{##1}}}
\@namedef{PY@tok@nc}{\let\PY@bf=\textbf\def\PY@tc##1{\textcolor[rgb]{0.00,0.00,1.00}{##1}}}
\@namedef{PY@tok@nn}{\let\PY@bf=\textbf\def\PY@tc##1{\textcolor[rgb]{0.00,0.00,1.00}{##1}}}
\@namedef{PY@tok@ne}{\let\PY@bf=\textbf\def\PY@tc##1{\textcolor[rgb]{0.80,0.25,0.22}{##1}}}
\@namedef{PY@tok@nv}{\def\PY@tc##1{\textcolor[rgb]{0.10,0.09,0.49}{##1}}}
\@namedef{PY@tok@no}{\def\PY@tc##1{\textcolor[rgb]{0.53,0.00,0.00}{##1}}}
\@namedef{PY@tok@nl}{\def\PY@tc##1{\textcolor[rgb]{0.46,0.46,0.00}{##1}}}
\@namedef{PY@tok@ni}{\let\PY@bf=\textbf\def\PY@tc##1{\textcolor[rgb]{0.44,0.44,0.44}{##1}}}
\@namedef{PY@tok@na}{\def\PY@tc##1{\textcolor[rgb]{0.41,0.47,0.13}{##1}}}
\@namedef{PY@tok@nt}{\let\PY@bf=\textbf\def\PY@tc##1{\textcolor[rgb]{0.00,0.50,0.00}{##1}}}
\@namedef{PY@tok@nd}{\def\PY@tc##1{\textcolor[rgb]{0.67,0.13,1.00}{##1}}}
\@namedef{PY@tok@s}{\def\PY@tc##1{\textcolor[rgb]{0.73,0.13,0.13}{##1}}}
\@namedef{PY@tok@sd}{\let\PY@it=\textit\def\PY@tc##1{\textcolor[rgb]{0.73,0.13,0.13}{##1}}}
\@namedef{PY@tok@si}{\let\PY@bf=\textbf\def\PY@tc##1{\textcolor[rgb]{0.64,0.35,0.47}{##1}}}
\@namedef{PY@tok@se}{\let\PY@bf=\textbf\def\PY@tc##1{\textcolor[rgb]{0.67,0.36,0.12}{##1}}}
\@namedef{PY@tok@sr}{\def\PY@tc##1{\textcolor[rgb]{0.64,0.35,0.47}{##1}}}
\@namedef{PY@tok@ss}{\def\PY@tc##1{\textcolor[rgb]{0.10,0.09,0.49}{##1}}}
\@namedef{PY@tok@sx}{\def\PY@tc##1{\textcolor[rgb]{0.00,0.50,0.00}{##1}}}
\@namedef{PY@tok@m}{\def\PY@tc##1{\textcolor[rgb]{0.40,0.40,0.40}{##1}}}
\@namedef{PY@tok@gh}{\let\PY@bf=\textbf\def\PY@tc##1{\textcolor[rgb]{0.00,0.00,0.50}{##1}}}
\@namedef{PY@tok@gu}{\let\PY@bf=\textbf\def\PY@tc##1{\textcolor[rgb]{0.50,0.00,0.50}{##1}}}
\@namedef{PY@tok@gd}{\def\PY@tc##1{\textcolor[rgb]{0.63,0.00,0.00}{##1}}}
\@namedef{PY@tok@gi}{\def\PY@tc##1{\textcolor[rgb]{0.00,0.52,0.00}{##1}}}
\@namedef{PY@tok@gr}{\def\PY@tc##1{\textcolor[rgb]{0.89,0.00,0.00}{##1}}}
\@namedef{PY@tok@ge}{\let\PY@it=\textit}
\@namedef{PY@tok@gs}{\let\PY@bf=\textbf}
\@namedef{PY@tok@gp}{\let\PY@bf=\textbf\def\PY@tc##1{\textcolor[rgb]{0.00,0.00,0.50}{##1}}}
\@namedef{PY@tok@go}{\def\PY@tc##1{\textcolor[rgb]{0.44,0.44,0.44}{##1}}}
\@namedef{PY@tok@gt}{\def\PY@tc##1{\textcolor[rgb]{0.00,0.27,0.87}{##1}}}
\@namedef{PY@tok@err}{\def\PY@bc##1{{\setlength{\fboxsep}{\string -\fboxrule}\fcolorbox[rgb]{1.00,0.00,0.00}{1,1,1}{\strut ##1}}}}
\@namedef{PY@tok@kc}{\let\PY@bf=\textbf\def\PY@tc##1{\textcolor[rgb]{0.00,0.50,0.00}{##1}}}
\@namedef{PY@tok@kd}{\let\PY@bf=\textbf\def\PY@tc##1{\textcolor[rgb]{0.00,0.50,0.00}{##1}}}
\@namedef{PY@tok@kn}{\let\PY@bf=\textbf\def\PY@tc##1{\textcolor[rgb]{0.00,0.50,0.00}{##1}}}
\@namedef{PY@tok@kr}{\let\PY@bf=\textbf\def\PY@tc##1{\textcolor[rgb]{0.00,0.50,0.00}{##1}}}
\@namedef{PY@tok@bp}{\def\PY@tc##1{\textcolor[rgb]{0.00,0.50,0.00}{##1}}}
\@namedef{PY@tok@fm}{\def\PY@tc##1{\textcolor[rgb]{0.00,0.00,1.00}{##1}}}
\@namedef{PY@tok@vc}{\def\PY@tc##1{\textcolor[rgb]{0.10,0.09,0.49}{##1}}}
\@namedef{PY@tok@vg}{\def\PY@tc##1{\textcolor[rgb]{0.10,0.09,0.49}{##1}}}
\@namedef{PY@tok@vi}{\def\PY@tc##1{\textcolor[rgb]{0.10,0.09,0.49}{##1}}}
\@namedef{PY@tok@vm}{\def\PY@tc##1{\textcolor[rgb]{0.10,0.09,0.49}{##1}}}
\@namedef{PY@tok@sa}{\def\PY@tc##1{\textcolor[rgb]{0.73,0.13,0.13}{##1}}}
\@namedef{PY@tok@sb}{\def\PY@tc##1{\textcolor[rgb]{0.73,0.13,0.13}{##1}}}
\@namedef{PY@tok@sc}{\def\PY@tc##1{\textcolor[rgb]{0.73,0.13,0.13}{##1}}}
\@namedef{PY@tok@dl}{\def\PY@tc##1{\textcolor[rgb]{0.73,0.13,0.13}{##1}}}
\@namedef{PY@tok@s2}{\def\PY@tc##1{\textcolor[rgb]{0.73,0.13,0.13}{##1}}}
\@namedef{PY@tok@sh}{\def\PY@tc##1{\textcolor[rgb]{0.73,0.13,0.13}{##1}}}
\@namedef{PY@tok@s1}{\def\PY@tc##1{\textcolor[rgb]{0.73,0.13,0.13}{##1}}}
\@namedef{PY@tok@mb}{\def\PY@tc##1{\textcolor[rgb]{0.40,0.40,0.40}{##1}}}
\@namedef{PY@tok@mf}{\def\PY@tc##1{\textcolor[rgb]{0.40,0.40,0.40}{##1}}}
\@namedef{PY@tok@mh}{\def\PY@tc##1{\textcolor[rgb]{0.40,0.40,0.40}{##1}}}
\@namedef{PY@tok@mi}{\def\PY@tc##1{\textcolor[rgb]{0.40,0.40,0.40}{##1}}}
\@namedef{PY@tok@il}{\def\PY@tc##1{\textcolor[rgb]{0.40,0.40,0.40}{##1}}}
\@namedef{PY@tok@mo}{\def\PY@tc##1{\textcolor[rgb]{0.40,0.40,0.40}{##1}}}
\@namedef{PY@tok@ch}{\let\PY@it=\textit\def\PY@tc##1{\textcolor[rgb]{0.24,0.48,0.48}{##1}}}
\@namedef{PY@tok@cm}{\let\PY@it=\textit\def\PY@tc##1{\textcolor[rgb]{0.24,0.48,0.48}{##1}}}
\@namedef{PY@tok@cpf}{\let\PY@it=\textit\def\PY@tc##1{\textcolor[rgb]{0.24,0.48,0.48}{##1}}}
\@namedef{PY@tok@c1}{\let\PY@it=\textit\def\PY@tc##1{\textcolor[rgb]{0.24,0.48,0.48}{##1}}}
\@namedef{PY@tok@cs}{\let\PY@it=\textit\def\PY@tc##1{\textcolor[rgb]{0.24,0.48,0.48}{##1}}}

\def\PYZbs{\char`\\}
\def\PYZus{\char`\_}
\def\PYZob{\char`\{}
\def\PYZcb{\char`\}}
\def\PYZca{\char`\^}
\def\PYZam{\char`\&}
\def\PYZlt{\char`\<}
\def\PYZgt{\char`\>}
\def\PYZsh{\char`\#}
\def\PYZpc{\char`\%}
\def\PYZdl{\char`\$}
\def\PYZhy{\char`\-}
\def\PYZsq{\char`\'}
\def\PYZdq{\char`\"}
\def\PYZti{\char`\~}
% for compatibility with earlier versions
\def\PYZat{@}
\def\PYZlb{[}
\def\PYZrb{]}
\makeatother


    % For linebreaks inside Verbatim environment from package fancyvrb.
    \makeatletter
        \newbox\Wrappedcontinuationbox
        \newbox\Wrappedvisiblespacebox
        \newcommand*\Wrappedvisiblespace {\textcolor{red}{\textvisiblespace}}
        \newcommand*\Wrappedcontinuationsymbol {\textcolor{red}{\llap{\tiny$\m@th\hookrightarrow$}}}
        \newcommand*\Wrappedcontinuationindent {3ex }
        \newcommand*\Wrappedafterbreak {\kern\Wrappedcontinuationindent\copy\Wrappedcontinuationbox}
        % Take advantage of the already applied Pygments mark-up to insert
        % potential linebreaks for TeX processing.
        %        {, <, #, %, $, ' and ": go to next line.
        %        _, }, ^, &, >, - and ~: stay at end of broken line.
        % Use of \textquotesingle for straight quote.
        \newcommand*\Wrappedbreaksatspecials {%
            \def\PYGZus{\discretionary{\char`\_}{\Wrappedafterbreak}{\char`\_}}%
            \def\PYGZob{\discretionary{}{\Wrappedafterbreak\char`\{}{\char`\{}}%
            \def\PYGZcb{\discretionary{\char`\}}{\Wrappedafterbreak}{\char`\}}}%
            \def\PYGZca{\discretionary{\char`\^}{\Wrappedafterbreak}{\char`\^}}%
            \def\PYGZam{\discretionary{\char`\&}{\Wrappedafterbreak}{\char`\&}}%
            \def\PYGZlt{\discretionary{}{\Wrappedafterbreak\char`\<}{\char`\<}}%
            \def\PYGZgt{\discretionary{\char`\>}{\Wrappedafterbreak}{\char`\>}}%
            \def\PYGZsh{\discretionary{}{\Wrappedafterbreak\char`\#}{\char`\#}}%
            \def\PYGZpc{\discretionary{}{\Wrappedafterbreak\char`\%}{\char`\%}}%
            \def\PYGZdl{\discretionary{}{\Wrappedafterbreak\char`\$}{\char`\$}}%
            \def\PYGZhy{\discretionary{\char`\-}{\Wrappedafterbreak}{\char`\-}}%
            \def\PYGZsq{\discretionary{}{\Wrappedafterbreak\textquotesingle}{\textquotesingle}}%
            \def\PYGZdq{\discretionary{}{\Wrappedafterbreak\char`\"}{\char`\"}}%
            \def\PYGZti{\discretionary{\char`\~}{\Wrappedafterbreak}{\char`\~}}%
        }
        % Some characters . , ; ? ! / are not pygmentized.
        % This macro makes them "active" and they will insert potential linebreaks
        \newcommand*\Wrappedbreaksatpunct {%
            \lccode`\~`\.\lowercase{\def~}{\discretionary{\hbox{\char`\.}}{\Wrappedafterbreak}{\hbox{\char`\.}}}%
            \lccode`\~`\,\lowercase{\def~}{\discretionary{\hbox{\char`\,}}{\Wrappedafterbreak}{\hbox{\char`\,}}}%
            \lccode`\~`\;\lowercase{\def~}{\discretionary{\hbox{\char`\;}}{\Wrappedafterbreak}{\hbox{\char`\;}}}%
            \lccode`\~`\:\lowercase{\def~}{\discretionary{\hbox{\char`\:}}{\Wrappedafterbreak}{\hbox{\char`\:}}}%
            \lccode`\~`\?\lowercase{\def~}{\discretionary{\hbox{\char`\?}}{\Wrappedafterbreak}{\hbox{\char`\?}}}%
            \lccode`\~`\!\lowercase{\def~}{\discretionary{\hbox{\char`\!}}{\Wrappedafterbreak}{\hbox{\char`\!}}}%
            \lccode`\~`\/\lowercase{\def~}{\discretionary{\hbox{\char`\/}}{\Wrappedafterbreak}{\hbox{\char`\/}}}%
            \catcode`\.\active
            \catcode`\,\active
            \catcode`\;\active
            \catcode`\:\active
            \catcode`\?\active
            \catcode`\!\active
            \catcode`\/\active
            \lccode`\~`\~
        }
    \makeatother

    \let\OriginalVerbatim=\Verbatim
    \makeatletter
    \renewcommand{\Verbatim}[1][1]{%
        %\parskip\z@skip
        \sbox\Wrappedcontinuationbox {\Wrappedcontinuationsymbol}%
        \sbox\Wrappedvisiblespacebox {\FV@SetupFont\Wrappedvisiblespace}%
        \def\FancyVerbFormatLine ##1{\hsize\linewidth
            \vtop{\raggedright\hyphenpenalty\z@\exhyphenpenalty\z@
                \doublehyphendemerits\z@\finalhyphendemerits\z@
                \strut ##1\strut}%
        }%
        % If the linebreak is at a space, the latter will be displayed as visible
        % space at end of first line, and a continuation symbol starts next line.
        % Stretch/shrink are however usually zero for typewriter font.
        \def\FV@Space {%
            \nobreak\hskip\z@ plus\fontdimen3\font minus\fontdimen4\font
            \discretionary{\copy\Wrappedvisiblespacebox}{\Wrappedafterbreak}
            {\kern\fontdimen2\font}%
        }%

        % Allow breaks at special characters using \PYG... macros.
        \Wrappedbreaksatspecials
        % Breaks at punctuation characters . , ; ? ! and / need catcode=\active
        \OriginalVerbatim[#1,codes*=\Wrappedbreaksatpunct]%
    }
    \makeatother

    % Exact colors from NB
    \definecolor{incolor}{HTML}{303F9F}
    \definecolor{outcolor}{HTML}{D84315}
    \definecolor{cellborder}{HTML}{CFCFCF}
    \definecolor{cellbackground}{HTML}{F7F7F7}

    % prompt
    \makeatletter
    \newcommand{\boxspacing}{\kern\kvtcb@left@rule\kern\kvtcb@boxsep}
    \makeatother
    \newcommand{\prompt}[4]{
        {\ttfamily\llap{{\color{#2}[#3]:\hspace{3pt}#4}}\vspace{-\baselineskip}}
    }
    

    
    % Prevent overflowing lines due to hard-to-break entities
    \sloppy
    % Setup hyperref package
    \hypersetup{
      breaklinks=true,  % so long urls are correctly broken across lines
      colorlinks=true,
      urlcolor=urlcolor,
      linkcolor=linkcolor,
      citecolor=citecolor,
      }
    % Slightly bigger margins than the latex defaults
    
    \geometry{verbose,tmargin=1in,bmargin=1in,lmargin=1in,rmargin=1in}
    
    

\begin{document}
\textual
\setcounter{page}{50}

\begin{appendices}    

Para a implementação do protocolo, iniciaremos carregando as seguintes
bibliotecas e funções. Vale ressaltar que as bibliotecas nos fornecem os
recursos necessários para o desenvolvimento das operações. As funções
\textbf{TensorProduct} e \textbf{sqrt} foram importadas para tornar as
operações mais objetivas.

    \begin{tcolorbox}[breakable, size=fbox, boxrule=1pt, pad at break*=1mm,colback=cellbackground, colframe=cellborder]
\prompt{In}{incolor}{1}{\boxspacing}
\begin{Verbatim}[commandchars=\\\{\}]
\PY{k+kn}{import} \PY{n+nn}{numpy} \PY{k}{as} \PY{n+nn}{np}
\PY{k+kn}{import} \PY{n+nn}{math} \PY{k}{as} \PY{n+nn}{mt}
\PY{k+kn}{import} \PY{n+nn}{sympy} \PY{k}{as} \PY{n+nn}{sp}
\PY{k+kn}{from} \PY{n+nn}{sympy}\PY{n+nn}{.}\PY{n+nn}{physics}\PY{n+nn}{.}\PY{n+nn}{quantum} \PY{k+kn}{import} \PY{n}{TensorProduct} \PY{k}{as} \PY{n}{TP}
\PY{k+kn}{from} \PY{n+nn}{math} \PY{k+kn}{import} \PY{n}{sqrt}
\PY{k+kn}{import} \PY{n+nn}{sys}
\PY{k+kn}{import} \PY{n+nn}{random}
\end{Verbatim}
\end{tcolorbox}

    \hypertarget{qubits-iniciais}{%
\subsubsection*{Qubits iniciais}\label{qubits-iniciais}}

Para iniciar o protocolo, iremos definir os possíveis estados de entrada
dos qubits que serão operacionalizados. As bases da Computação Quântica
definem as bases \(|0>\) e \(|1>\) e sua representação matricial é
descrita por 
$
\begin{bmatrix}
1 \\
0
\end{bmatrix}
 e 
\begin{bmatrix}
0 \\
1
\end{bmatrix}
$
 respectivamente.

\hypertarget{operadores-quuxe2nticos}{%
\subsubsection*{Operadores quânticos}\label{operadores-quuxe2nticos}}

Devem ser definidos também, todas os operadores/portas lógicas que
atuarão durante o protocolo. Sendo eles os operadores
\textbf{\emph{CNOT}}, \textbf{\emph{Hadamard}}, \textbf{\emph{Pauli-X}},
\textbf{\emph{Pauli-Z}} e \textbf{\emph{Identidade}}. Suas
representações matriciais são descritas por:

$ CNOT =
\begin{bmatrix}
1 & 0 & 0 & 0 \\
0 & 1 & 0 & 0 \\
0 & 0 & 0 & 1 \\
0 & 0 & 1 & 0
\end{bmatrix}
 ,  H = \frac{1}{\sqrt2}
\begin{bmatrix}
1 & 1 \\
1 & -1
\end{bmatrix}
 ,  X =
\begin{bmatrix}
0 & 1 \\
1 & 0
\end{bmatrix}
,  Z =
\begin{bmatrix}
1 & 0 \\
0 & -1
\end{bmatrix}
 e  I =
\begin{bmatrix}
1 & 0 \\
0 & 1
\end{bmatrix}
$.

    \begin{tcolorbox}[breakable, size=fbox, boxrule=1pt, pad at break*=1mm,colback=cellbackground, colframe=cellborder]
\prompt{In}{incolor}{2}{\boxspacing}
\begin{Verbatim}[commandchars=\\\{\}]
\PY{c+c1}{\PYZsh{}Variáveis que descrevem os qubits}
\PY{n}{qbit0}\PY{o}{=} \PY{n}{np}\PY{o}{.}\PY{n}{matrix}\PY{p}{(}\PY{p}{[}\PY{l+m+mi}{1}\PY{p}{,}\PY{l+m+mi}{0}\PY{p}{]}\PY{p}{)}\PY{o}{.}\PY{n}{transpose}\PY{p}{(}\PY{p}{)}
\PY{n}{qbit1}\PY{o}{=} \PY{n}{np}\PY{o}{.}\PY{n}{matrix}\PY{p}{(}\PY{p}{[}\PY{l+m+mi}{0}\PY{p}{,}\PY{l+m+mi}{1}\PY{p}{]}\PY{p}{)}\PY{o}{.}\PY{n}{transpose}\PY{p}{(}\PY{p}{)}

\PY{c+c1}{\PYZsh{}Variáveis que descrevem os operadores quânticos}
\PY{c+c1}{\PYZsh{}CNOT}
\PY{n}{CNOT} \PY{o}{=} \PY{n}{np}\PY{o}{.}\PY{n}{matrix}\PY{p}{(}\PY{p}{[}\PY{p}{[}\PY{l+m+mi}{1}\PY{p}{,}\PY{l+m+mi}{0}\PY{p}{,}\PY{l+m+mi}{0}\PY{p}{,}\PY{l+m+mi}{0}\PY{p}{]}\PY{p}{,}\PY{p}{[}\PY{l+m+mi}{0}\PY{p}{,}\PY{l+m+mi}{1}\PY{p}{,}\PY{l+m+mi}{0}\PY{p}{,}\PY{l+m+mi}{0}\PY{p}{]}\PY{p}{,}\PY{p}{[}\PY{l+m+mi}{0}\PY{p}{,}\PY{l+m+mi}{0}\PY{p}{,}\PY{l+m+mi}{0}\PY{p}{,}\PY{l+m+mi}{1}\PY{p}{]}\PY{p}{,}\PY{p}{[}\PY{l+m+mi}{0}\PY{p}{,}\PY{l+m+mi}{0}\PY{p}{,}\PY{l+m+mi}{1}\PY{p}{,}\PY{l+m+mi}{0}\PY{p}{]}\PY{p}{]}\PY{p}{)}

\PY{c+c1}{\PYZsh{}Hadamard}
\PY{n}{H} \PY{o}{=} \PY{l+m+mi}{1}\PY{o}{/}\PY{n}{sqrt}\PY{p}{(}\PY{l+m+mi}{2}\PY{p}{)}\PY{o}{*}\PY{p}{(}\PY{n}{np}\PY{o}{.}\PY{n}{matrix}\PY{p}{(}\PY{p}{[}\PY{p}{[}\PY{l+m+mi}{1}\PY{p}{,}\PY{l+m+mi}{1}\PY{p}{]}\PY{p}{,} \PY{p}{[}\PY{l+m+mi}{1}\PY{p}{,}\PY{o}{\PYZhy{}}\PY{l+m+mi}{1}\PY{p}{]}\PY{p}{]}\PY{p}{)}\PY{p}{)}

\PY{c+c1}{\PYZsh{} Pauli\PYZhy{}X}
\PY{n}{X} \PY{o}{=} \PY{n}{np}\PY{o}{.}\PY{n}{matrix}\PY{p}{(}\PY{p}{[}\PY{p}{[}\PY{l+m+mi}{0}\PY{p}{,} \PY{l+m+mi}{1}\PY{p}{]}\PY{p}{,} \PY{p}{[}\PY{l+m+mi}{1}\PY{p}{,} \PY{l+m+mi}{0}\PY{p}{]}\PY{p}{]}\PY{p}{)}

\PY{c+c1}{\PYZsh{} Pauli\PYZhy{}Z}
\PY{n}{Z} \PY{o}{=} \PY{n}{np}\PY{o}{.}\PY{n}{matrix}\PY{p}{(}\PY{p}{[}\PY{p}{[}\PY{l+m+mi}{1}\PY{p}{,} \PY{l+m+mi}{0}\PY{p}{]}\PY{p}{,} \PY{p}{[}\PY{l+m+mi}{0}\PY{p}{,} \PY{o}{\PYZhy{}}\PY{l+m+mi}{1}\PY{p}{]}\PY{p}{]}\PY{p}{)}

\PY{c+c1}{\PYZsh{}Identidade}
\PY{n}{I} \PY{o}{=} \PY{n}{np}\PY{o}{.}\PY{n}{matrix} \PY{p}{(}\PY{p}{[}\PY{p}{[}\PY{l+m+mi}{1}\PY{p}{,}\PY{l+m+mi}{0}\PY{p}{]}\PY{p}{,} \PY{p}{[}\PY{l+m+mi}{0}\PY{p}{,}\PY{l+m+mi}{1}\PY{p}{]}\PY{p}{]}\PY{p}{)}
\end{Verbatim}
\end{tcolorbox}

    \hypertarget{emaranhamento}{%
\subsection*{Emaranhamento}\label{emaranhamento}}

O protocolo é iniciado com a realização do emaranhamento de modo que
aplicaremos a porta \textbf{\emph{H}} no estado \(|0>\) e em seguida,
definiremos o qubit alvo e o qubit controle para a aplicação da porta
\textbf{\emph{CNOT}}. O qubit alvo e controle são definidos realizando o
produto tensorial do primeiro pelo segundo de modo que:
$ AC = \frac{1}{\sqrt2} \begin{bmatrix}
1 \\
1
\end{bmatrix} \otimes \begin{bmatrix}
1 \\
0
\end{bmatrix}
$
Vale ressaltar que nesse caso estamos emaranhando dois qubits no
estado \(|0>\) criando uma das Bases de Bell que, nesse caso é descrita
por: \[\Phi_+ = \frac{1}{\sqrt2} \begin{bmatrix}
1 \\
0 \\
0 \\
1
\end{bmatrix}
\]

    \begin{tcolorbox}[breakable, size=fbox, boxrule=1pt, pad at break*=1mm,colback=cellbackground, colframe=cellborder]
\prompt{In}{incolor}{3}{\boxspacing}
\begin{Verbatim}[commandchars=\\\{\}]
\PY{c+c1}{\PYZsh{}Aplicação da porta Hadamard no qubit\PYZhy{}0}
\PY{n}{Hqbit0}\PY{o}{=} \PY{n}{H}\PY{o}{*}\PY{n}{qbit0}

\PY{c+c1}{\PYZsh{}Definição Alvo e Controle}
\PY{n}{AC}\PY{o}{=} \PY{n}{TP}\PY{p}{(}\PY{n}{Hqbit0}\PY{p}{,}\PY{n}{qbit0}\PY{p}{)}

\PY{c+c1}{\PYZsh{}Aplicação da porta CNOT}
\PY{n}{Bell00} \PY{o}{=} \PY{n}{CNOT} \PY{o}{*} \PY{n}{AC}
  
\end{Verbatim}
\end{tcolorbox}

\hypertarget{protocolo}{%
\subsection*{Início do protocolo}\label{emaranhamento}}
    Iremos agora definir o estado do qubit que será teletransportado. Para
isso é solicitado ao usuário que insira as probabilidades de \(\alpha\)
e \(\beta\) desejados. Lembrando que estas devem estar normalizadas e
sua soma deve ser igual a 1, caso contrário o programa não pode
continuar operando. Como o estado do qubit generalizado é definido por:
\[ |\psi> = \alpha \begin{bmatrix}
1 \\
0
\end{bmatrix} + \beta \begin{bmatrix}
0 \\
1
\end{bmatrix}
\] Atribuímos os valores inseridos de \(\alpha\) e \(\beta\) ao estado
inicial.

    \begin{tcolorbox}[breakable, size=fbox, boxrule=1pt, pad at break*=1mm,colback=cellbackground, colframe=cellborder]
\prompt{In}{incolor}{4}{\boxspacing}
\begin{Verbatim}[commandchars=\\\{\}]
\PY{c+c1}{\PYZsh{}Determinação do qubit para o protocolo}
\PY{n}{estado\PYZus{}alfa} \PY{o}{=} \PY{n+nb}{float}\PY{p}{(}\PY{n+nb}{input}\PY{p}{(}\PY{l+s+s1}{\PYZsq{}}\PY{l+s+s1}{Qual o estado alfa do Qubit que deseja enviar?}\PY{l+s+s1}{\PYZsq{}}\PY{p}{)}\PY{p}{)}
\PY{n}{estado\PYZus{}beta} \PY{o}{=} \PY{n+nb}{float}\PY{p}{(}\PY{n+nb}{input}\PY{p}{(}\PY{l+s+s1}{\PYZsq{}}\PY{l+s+s1}{Qual o estado beta do Qubit que deseja enviar?}\PY{l+s+s1}{\PYZsq{}}\PY{p}{)}\PY{p}{)}


\PY{c+c1}{\PYZsh{}Verifica se a soma é igual a 1 (condição de normalização)}
\PY{k}{if} \PY{o+ow}{not} \PY{n}{mt}\PY{o}{.}\PY{n}{isclose}\PY{p}{(}\PY{n}{estado\PYZus{}alfa}  \PY{o}{+} \PY{n}{estado\PYZus{}beta}\PY{p}{,} \PY{l+m+mf}{1.0}\PY{p}{)}\PY{p}{:}
    \PY{n+nb}{print}\PY{p}{(}\PY{l+s+s2}{\PYZdq{}}\PY{l+s+s2}{Erro: A soma das entradas não é igual a 1.}\PY{l+s+s2}{\PYZdq{}}\PY{p}{)}
    \PY{n}{sys}\PY{o}{.}\PY{n}{exit}\PY{p}{(}\PY{p}{)}
    
\PY{c+c1}{\PYZsh{}Determina a mensagem a ser enviada}
\PY{n}{alfa\PYZus{}inicial} \PY{o}{=} \PY{p}{(}\PY{n}{estado\PYZus{}alfa} \PY{o}{*} \PY{n}{qbit0}\PY{p}{)}
\PY{n}{beta\PYZus{}inicial} \PY{o}{=} \PY{p}{(}\PY{n}{estado\PYZus{}beta} \PY{o}{*} \PY{n}{qbit1}\PY{p}{)}
\PY{n}{estado\PYZus{}inicial} \PY{o}{=} \PY{n}{alfa\PYZus{}inicial} \PY{o}{+} \PY{n}{beta\PYZus{}inicial}

\PY{n+nb}{print}\PY{p}{(}\PY{n}{estado\PYZus{}inicial}\PY{p}{)}
\end{Verbatim}
\end{tcolorbox}

 

    A próxima etapa do protocolo consiste na aplicação da porta
\textbf{\emph{CNOT}} nos três qubits que compõem o estado geral do
sistema. Os três qubits são o estado inicial a ser teletransportado
atuando como controle e o par emaranhado atuando como alvo. A atuação
ocorre apenas no qubit presente no local \textbf{A}, porém afeta
probabilisticamente o qubit no local \textbf{B}. Para definir o estado
geral são somados os produtos tensoriais de todos os estados dos qubits.
Em seguida, a porta \textbf{\emph{CNOT}} é dimensionada para atuar sob
três qubits com a realização do produto tensorial entre
\textbf{\emph{CNOT}} e \textbf{\emph{I}} e por último multiplicamos a
porta \textbf{\emph{CNOT}} pela soma dos estados dos qubits. Esse estado
é chamado de \(|\psi_1>\) e pode ser definido por:
\[ |\psi_1> = \frac{1}{\sqrt2} \left\{ \alpha \begin{bmatrix}
1 \\
0
\end{bmatrix} \otimes  \begin{bmatrix}
1 \\
0 \\
0 \\
1
\end{bmatrix}  + \beta \begin{bmatrix}
0 \\
1
\end{bmatrix} \otimes  \begin{bmatrix}
0 \\
1 \\
1 \\
0
\end{bmatrix}  \right\}
\] Em seguida, a porta \textbf{\emph{H}} é aplicada no qubit a ser
teletransportado. Sua aplicação foi realizada em cada um dos estados do
qubit de maneira isolada para melhor observação da modificação dos
estados deste. Após a aplicação da porta \textbf{\emph{H}} o estado
\(|\psi_2>\) é definido por:
\[|\psi_2> = \frac{1}{2} \left\{ \alpha \begin{bmatrix}
1 \\
1
\end{bmatrix} \begin{bmatrix}
1 \\
0 \\
0 \\
1
\end{bmatrix} + \beta \begin{bmatrix}
1 \\
-1
\end{bmatrix} \begin{bmatrix}
0 \\
1 \\
1 \\
0
\end{bmatrix} \right\}
\]

    \begin{tcolorbox}[breakable, size=fbox, boxrule=1pt, pad at break*=1mm,colback=cellbackground, colframe=cellborder]
\prompt{In}{incolor}{5}{\boxspacing}
\begin{Verbatim}[commandchars=\\\{\}]
\PY{c+c1}{\PYZsh{}Estado geral (produto tensorial entre os estados de todos os qubits do sistema) }
\PY{n}{psi\PYZus{}0} \PY{o}{=} \PY{n}{TP}\PY{p}{(}\PY{n}{qbit0}\PY{p}{,} \PY{n}{qbit0}\PY{p}{,} \PY{n}{qbit0}\PY{p}{)} \PY{o}{+} \PY{n}{TP}\PY{p}{(}\PY{n}{qbit0}\PY{p}{,} \PY{n}{qbit1}\PY{p}{,} \PY{n}{qbit1}\PY{p}{)} \PY{o}{+} \PY{n}{TP}\PY{p}{(}\PY{n}{qbit1}\PY{p}{,}\PY{n}{qbit0}\PY{p}{,} \PY{n}{qbit0}\PY{p}{)} \PY{o}{+} \PY{n}{TP}\PY{p}{(}\PY{n}{qbit1}\PY{p}{,}\PY{n}{qbit1}\PY{p}{,}\PY{n}{qbit1}\PY{p}{)}

\PY{c+c1}{\PYZsh{}Dimensionamento da porta CNOT para 3 qubits}
\PY{n}{CNOT\PYZus{}I} \PY{o}{=} \PY{n}{TP}\PY{p}{(}\PY{n}{CNOT}\PY{p}{,}\PY{n}{I}\PY{p}{)}

\PY{c+c1}{\PYZsh{}Aplicação da porta CNOT}
\PY{n}{psi\PYZus{}1} \PY{o}{=} \PY{n}{CNOT\PYZus{}I} \PY{o}{*} \PY{n}{psi\PYZus{}0}

\PY{c+c1}{\PYZsh{}Aplicação da porta Hadamard}
\PY{n}{H\PYZus{}estado\PYZus{}alfa} \PY{o}{=} \PY{n}{H} \PY{o}{*} \PY{n}{alfa\PYZus{}inicial}
\PY{n}{H\PYZus{}estado\PYZus{}beta} \PY{o}{=} \PY{n}{H} \PY{o}{*} \PY{n}{beta\PYZus{}inicial}
\end{Verbatim}
\end{tcolorbox}

    Para realizar a Medição e a Reconstrução do estado inicial, iremos
separar os estados associados às probabilidades \(\alpha\) e \(\beta\)
de modo que: \[ \alpha \begin{bmatrix}
1 \\
1
\end{bmatrix} = \alpha \begin{bmatrix}
1 \\
0
\end{bmatrix} + \alpha \begin{bmatrix}
0 \\
1
\end{bmatrix} \quad e  \quad     \beta \begin{bmatrix}
1 \\
-1
\end{bmatrix} = \beta \begin{bmatrix}
1 \\
0 
\end{bmatrix} - \beta \begin{bmatrix}
0 \\
1
\end{bmatrix}
\]

    \begin{tcolorbox}[breakable, size=fbox, boxrule=1pt, pad at break*=1mm,colback=cellbackground, colframe=cellborder]
\prompt{In}{incolor}{6}{\boxspacing}
\begin{Verbatim}[commandchars=\\\{\}]
\PY{c+c1}{\PYZsh{}Separação dos estados Alfa e Beta}

\PY{n}{estadoa0} \PY{o}{=} \PY{n+nb}{float}\PY{p}{(}\PY{n}{H\PYZus{}estado\PYZus{}alfa}\PY{p}{[}\PY{l+m+mi}{0}\PY{p}{]}\PY{p}{[}\PY{l+m+mi}{0}\PY{p}{]}\PY{p}{)}
\PY{n}{estadoa1} \PY{o}{=} \PY{n+nb}{float}\PY{p}{(}\PY{n}{H\PYZus{}estado\PYZus{}alfa}\PY{p}{[}\PY{l+m+mi}{1}\PY{p}{]}\PY{p}{[}\PY{l+m+mi}{0}\PY{p}{]}\PY{p}{)}
\PY{n}{estadob0} \PY{o}{=} \PY{n+nb}{float}\PY{p}{(}\PY{n}{H\PYZus{}estado\PYZus{}beta}\PY{p}{[}\PY{l+m+mi}{0}\PY{p}{]}\PY{p}{[}\PY{l+m+mi}{0}\PY{p}{]}\PY{p}{)}
\PY{n}{estadob1} \PY{o}{=} \PY{n+nb}{float}\PY{p}{(}\PY{n}{H\PYZus{}estado\PYZus{}beta}\PY{p}{[}\PY{l+m+mi}{1}\PY{p}{]}\PY{p}{[}\PY{l+m+mi}{0}\PY{p}{]}\PY{p}{)}

\PY{c+c1}{\PYZsh{}Determinação dos estados do qbit enviado}
\PY{n}{a0} \PY{o}{=} \PY{n}{np}\PY{o}{.}\PY{n}{matrix}\PY{p}{(}\PY{p}{[}\PY{n}{estadoa0}\PY{p}{,}\PY{l+m+mi}{0}\PY{p}{]}\PY{p}{)}\PY{o}{.}\PY{n}{transpose}\PY{p}{(}\PY{p}{)}
\PY{n}{a1} \PY{o}{=} \PY{n}{np}\PY{o}{.}\PY{n}{matrix}\PY{p}{(}\PY{p}{[}\PY{l+m+mi}{0}\PY{p}{,}\PY{n}{estadoa1}\PY{p}{]}\PY{p}{)}\PY{o}{.}\PY{n}{transpose}\PY{p}{(}\PY{p}{)}
\PY{n}{b0} \PY{o}{=} \PY{n}{np}\PY{o}{.}\PY{n}{matrix}\PY{p}{(}\PY{p}{[}\PY{n}{estadob0}\PY{p}{,}\PY{l+m+mi}{0}\PY{p}{]}\PY{p}{)}\PY{o}{.}\PY{n}{transpose}\PY{p}{(}\PY{p}{)}
\PY{n}{b1} \PY{o}{=} \PY{n}{np}\PY{o}{.}\PY{n}{matrix}\PY{p}{(}\PY{p}{[}\PY{l+m+mi}{0}\PY{p}{,}\PY{n}{estadob1}\PY{p}{]}\PY{p}{)}\PY{o}{.}\PY{n}{transpose}\PY{p}{(}\PY{p}{)}
\end{Verbatim}
\end{tcolorbox}

    Para realizar a medição, foram estabelecidos os possíveis estados
presentes em \textbf{A}, e um deles foi escolhido de maneira aleatória,
lembrando que a probabilidade de colapso em cada um dos estados é de
\(\frac{1}{4}\). Os estados que não foram escolhidos foram deletados,
simulando o colapso do sistema.

    \begin{tcolorbox}[breakable, size=fbox, boxrule=1pt, pad at break*=1mm,colback=cellbackground, colframe=cellborder]
\prompt{In}{incolor}{7}{\boxspacing}
\begin{Verbatim}[commandchars=\\\{\}]
\PY{c+c1}{\PYZsh{}Medição}

\PY{n}{estado\PYZus{}00} \PY{o}{=} \PY{n}{TP}\PY{p}{(}\PY{n}{qbit0}\PY{p}{,} \PY{n}{qbit0}\PY{p}{)}
\PY{n}{estado\PYZus{}11} \PY{o}{=} \PY{n}{TP}\PY{p}{(}\PY{n}{qbit1}\PY{p}{,} \PY{n}{qbit1}\PY{p}{)}
\PY{n}{estado\PYZus{}10} \PY{o}{=} \PY{n}{TP}\PY{p}{(}\PY{n}{qbit1}\PY{p}{,} \PY{n}{qbit0}\PY{p}{)}
\PY{n}{estado\PYZus{}01} \PY{o}{=} \PY{n}{TP}\PY{p}{(}\PY{n}{qbit0}\PY{p}{,} \PY{n}{qbit1}\PY{p}{)}

\PY{n}{group\PYZus{}estados} \PY{o}{=} \PY{p}{[}\PY{n}{estado\PYZus{}00}\PY{p}{,}\PY{n}{estado\PYZus{}11}\PY{p}{,}\PY{n}{estado\PYZus{}10}\PY{p}{,}\PY{n}{estado\PYZus{}01}\PY{p}{]}
\PY{n}{chosen\PYZus{}estados} \PY{o}{=} \PY{n}{random}\PY{o}{.}\PY{n}{choice}\PY{p}{(}\PY{n}{group\PYZus{}estados}\PY{p}{)}
\PY{n}{Medição} \PY{o}{=} \PY{n}{chosen\PYZus{}estados}

\PY{k}{del} \PY{n}{estado\PYZus{}00}\PY{p}{,} \PY{n}{estado\PYZus{}11}\PY{p}{,} \PY{n}{estado\PYZus{}10}\PY{p}{,} \PY{n}{estado\PYZus{}01}

\PY{n+nb}{print}\PY{p}{(}\PY{n}{Medição}\PY{p}{)}
\end{Verbatim}
\end{tcolorbox}

 
    As etapas a seguir acontecem no local \textbf{B} com o qubit do par
emaranhado. Segundo o estado colapsado enviado por \textbf{A}, o qubit
em \textbf{B} terá um estado correspondente de acordo com a relação:

\begin{longtable}[]{@{}cc@{}}
\toprule
\begin{minipage}[b]{0.37\columnwidth}\centering
Estado Medido em \textbf{A}\strut
\end{minipage} & \begin{minipage}[b]{0.57\columnwidth}\centering
Estado do qubit emaranhado em \textbf{B}\strut
\end{minipage}\tabularnewline
\midrule
\endhead
\begin{minipage}[t]{0.37\columnwidth}\centering
\(|00>\)\strut
\end{minipage} & \begin{minipage}[t]{0.57\columnwidth}\centering
\(\alpha \begin{bmatrix} 1 \\ 0 \end{bmatrix} + \beta \begin{bmatrix} 0 \\ 1 \end{bmatrix}\)\strut
\end{minipage}\tabularnewline
\begin{minipage}[t]{0.37\columnwidth}\centering
\(|11>\)\strut
\end{minipage} & \begin{minipage}[t]{0.57\columnwidth}\centering
\(\alpha \begin{bmatrix} 0 \\ 1 \end{bmatrix} - \beta \begin{bmatrix} 1 \\ 0 \end{bmatrix}\)\strut
\end{minipage}\tabularnewline
\begin{minipage}[t]{0.37\columnwidth}\centering
\(|01>\)\strut
\end{minipage} & \begin{minipage}[t]{0.57\columnwidth}\centering
\(\alpha \begin{bmatrix} 0 \\ 1 \end{bmatrix} + \beta \begin{bmatrix} 1 \\ 0 \end{bmatrix}\)\strut
\end{minipage}\tabularnewline
\begin{minipage}[t]{0.37\columnwidth}\centering
\(|10>\)\strut
\end{minipage} & \begin{minipage}[t]{0.57\columnwidth}\centering
\(\alpha \begin{bmatrix} 1 \\ 0 \end{bmatrix} - \beta \begin{bmatrix} 0 \\ 1 \end{bmatrix}\)\strut
\end{minipage}\tabularnewline
\bottomrule
\end{longtable}

    \begin{tcolorbox}[breakable, size=fbox, boxrule=1pt, pad at break*=1mm,colback=cellbackground, colframe=cellborder]
\prompt{In}{incolor}{8}{\boxspacing}
\begin{Verbatim}[commandchars=\\\{\}]
\PY{c+c1}{\PYZsh{}Define o estado em B em função da medição realizada em A}

\PY{k}{if}  \PY{n}{np}\PY{o}{.}\PY{n}{all}\PY{p}{(}\PY{n}{Medição} \PY{o}{==} \PY{p}{(}\PY{n}{TP}\PY{p}{(}\PY{n}{qbit0}\PY{p}{,}\PY{n}{qbit0}\PY{p}{)}\PY{p}{)}\PY{p}{)}\PY{p}{:}
    \PY{n}{estado\PYZus{}tp} \PY{o}{=} \PY{p}{(}\PY{n}{a0}\PY{o}{\PYZhy{}}\PY{n}{b1}\PY{p}{)}
\PY{k}{elif}  \PY{n}{np}\PY{o}{.}\PY{n}{all}\PY{p}{(}\PY{n}{Medição} \PY{o}{==} \PY{p}{(}\PY{n}{TP}\PY{p}{(}\PY{n}{qbit1}\PY{p}{,}\PY{n}{qbit1}\PY{p}{)}\PY{p}{)}\PY{p}{)}\PY{p}{:}
    \PY{n}{estado\PYZus{}tp} \PY{o}{=} \PY{p}{(}\PY{n}{a1}\PY{o}{\PYZhy{}}\PY{n}{b0}\PY{p}{)}
\PY{k}{elif}  \PY{n}{np}\PY{o}{.}\PY{n}{all}\PY{p}{(}\PY{n}{Medição} \PY{o}{==} \PY{p}{(}\PY{n}{TP}\PY{p}{(}\PY{n}{qbit0}\PY{p}{,}\PY{n}{qbit1}\PY{p}{)}\PY{p}{)}\PY{p}{)}\PY{p}{:}
    \PY{n}{estado\PYZus{}tp} \PY{o}{=} \PY{p}{(}\PY{n}{a1}\PY{o}{+}\PY{n}{b0}\PY{p}{)}
\PY{k}{elif} \PY{n}{np}\PY{o}{.}\PY{n}{all}\PY{p}{(}\PY{n}{Medição} \PY{o}{==} \PY{p}{(}\PY{n}{TP}\PY{p}{(}\PY{n}{qbit1}\PY{p}{,}\PY{n}{qbit0}\PY{p}{)}\PY{p}{)}\PY{p}{)}\PY{p}{:}
    \PY{n}{estado\PYZus{}tp} \PY{o}{=} \PY{p}{(}\PY{n}{a0}\PY{o}{+}\PY{n}{b1}\PY{p}{)}
\end{Verbatim}
\end{tcolorbox}

    Agora, de acordo com o estado associado em B, uma sequência de operações deverão ser realizadas conforme a relação: 

\begin{table}[ht!]
  \centering
  \begin{tabular}{cc}
    \toprule
    {Estado do qubit emaranhado em \textbf{B}}  & {Operação realizada}\\
	\midrule
	\(\alpha \begin{bmatrix} 1 \\ 0 \end{bmatrix} + \beta \begin{bmatrix} 0 \\ 1 \end{bmatrix}\) & Nenhuma operação \\
	\(\alpha \begin{bmatrix} 0 \\ 1 \end{bmatrix} - \beta \begin{bmatrix} 1 \\ 0 \end{bmatrix}\) & X e Z \\
	\(\alpha \begin{bmatrix} 0 \\ 1 \end{bmatrix} + \beta \begin{bmatrix} 1 \\ 0 \end{bmatrix}\) & X \\
	\(\alpha \begin{bmatrix} 1 \\ 0 \end{bmatrix} - \beta \begin{bmatrix} 0 \\ 1 \end{bmatrix}\) & Z
  \end{tabular}
\end{table}

    \begin{tcolorbox}[breakable, size=fbox, boxrule=1pt, pad at break*=1mm,colback=cellbackground, colframe=cellborder]
\prompt{In}{incolor}{12}{\boxspacing}
\begin{Verbatim}[commandchars=\\\{\}]
\PY{k}{if} \PY{n}{np}\PY{o}{.}\PY{n}{all}\PY{p}{(}\PY{n}{estado\PYZus{}tp} \PY{o}{==} \PY{p}{(}\PY{n}{a0}\PY{o}{\PYZhy{}}\PY{n}{b1}\PY{p}{)}\PY{p}{)}\PY{p}{:}
    \PY{n}{estado\PYZus{}final} \PY{o}{=} \PY{n}{estado\PYZus{}tp}
\PY{k}{elif} \PY{n}{np}\PY{o}{.}\PY{n}{all}\PY{p}{(}\PY{n}{estado\PYZus{}tp} \PY{o}{==} \PY{p}{(}\PY{n}{a1}\PY{o}{\PYZhy{}}\PY{n}{b0}\PY{p}{)}\PY{p}{)}\PY{p}{:}
    \PY{n}{estado\PYZus{}final} \PY{o}{=} \PY{n}{Z} \PY{o}{*} \PY{n}{X} \PY{o}{*} \PY{n}{estado\PYZus{}tp} 
\PY{k}{elif} \PY{n}{np}\PY{o}{.}\PY{n}{all}\PY{p}{(}\PY{n}{estado\PYZus{}tp} \PY{o}{==} \PY{p}{(}\PY{n}{a1}\PY{o}{+}\PY{n}{b0}\PY{p}{)}\PY{p}{)}\PY{p}{:}
    \PY{n}{estado\PYZus{}final} \PY{o}{=} \PY{n}{X} \PY{o}{*} \PY{n}{estado\PYZus{}tp} 
\PY{k}{elif} \PY{n}{np}\PY{o}{.}\PY{n}{all}\PY{p}{(}\PY{n}{estado\PYZus{}tp} \PY{o}{==} \PY{p}{(}\PY{n}{a0}\PY{o}{+}\PY{n}{b1}\PY{p}{)}\PY{p}{)}\PY{p}{:}
    \PY{n}{estado\PYZus{}final} \PY{o}{=} \PY{n}{Z} \PY{o}{*} \PY{n}{estado\PYZus{}tp} 
    
\PY{n}{estado\PYZus{}teletransportado} \PY{o}{=} \PY{n}{estado\PYZus{}final}\PY{o}{*}\PY{n}{sqrt}\PY{p}{(}\PY{l+m+mi}{2}\PY{p}{)}    

\PY{n}{alfa} \PY{o}{=} \PY{n+nb}{float}\PY{p}{(}\PY{n}{estado\PYZus{}teletransportado}\PY{p}{[}\PY{l+m+mi}{0}\PY{p}{]}\PY{p}{[}\PY{l+m+mi}{0}\PY{p}{]}\PY{p}{)}
\PY{n}{beta} \PY{o}{=} \PY{n+nb}{float}\PY{p}{(}\PY{n}{estado\PYZus{}teletransportado}\PY{p}{[}\PY{l+m+mi}{1}\PY{p}{]}\PY{p}{[}\PY{l+m+mi}{0}\PY{p}{]}\PY{p}{)}

\PY{n}{alfa\PYZus{}final} \PY{o}{=} \PY{n}{np}\PY{o}{.}\PY{n}{matrix}\PY{p}{(}\PY{p}{[}\PY{n}{alfa}\PY{p}{,}\PY{l+m+mi}{0}\PY{p}{]}\PY{p}{)}\PY{o}{.}\PY{n}{transpose}\PY{p}{(}\PY{p}{)}
\PY{n}{beta\PYZus{}final} \PY{o}{=} \PY{n}{np}\PY{o}{.}\PY{n}{matrix}\PY{p}{(}\PY{p}{[}\PY{l+m+mi}{0}\PY{p}{,}\PY{n}{beta}\PY{p}{]}\PY{p}{)}\PY{o}{.}\PY{n}{transpose}\PY{p}{(}\PY{p}{)}
\end{Verbatim}
\end{tcolorbox}

    Se durante o processo, algum ruído ocorresse, o estado do qubit
teletransportado seria alterado. A simulação de uma interação com os
ruídos \textbf{\emph{bitflip}} e \textbf{\emph{phaseflip}} foi
implementada de modo que o usuário pode selecionar se o teletransporte
teve ou não ruído e em caso afirmativo, qual ruído ocorreu. O ruído do
tipo \textbf{\emph{bitflip}} inverte o estado do qubit, ou seja, se este
era \(|0>\) passa a ser \(|1>\) e vice-versa. O ruído do tipo
\textbf{\emph{phaseflip}} inverte a fase do qubit.

    \begin{tcolorbox}[breakable, size=fbox, boxrule=1pt, pad at break*=1mm,colback=cellbackground, colframe=cellborder]
\prompt{In}{incolor}{ }{\boxspacing}
\begin{Verbatim}[commandchars=\\\{\}]
\PY{c+c1}{\PYZsh{}Aplicação de ruído}
\PY{c+c1}{\PYZsh{}Determinando os ruídos}
\PY{n}{bitflip} \PY{o}{=} \PY{n}{np}\PY{o}{.}\PY{n}{matrix}\PY{p}{(}\PY{p}{[}\PY{p}{[}\PY{l+m+mi}{0}\PY{p}{,} \PY{l+m+mi}{1}\PY{p}{]}\PY{p}{,} \PY{p}{[}\PY{l+m+mi}{1}\PY{p}{,} \PY{l+m+mi}{0}\PY{p}{]}\PY{p}{]}\PY{p}{)}
\PY{n}{phaseflip} \PY{o}{=} \PY{n}{np}\PY{o}{.}\PY{n}{matrix}\PY{p}{(}\PY{p}{[}\PY{p}{[}\PY{l+m+mi}{1}\PY{p}{,} \PY{l+m+mi}{0}\PY{p}{]}\PY{p}{,} \PY{p}{[}\PY{l+m+mi}{0}\PY{p}{,} \PY{o}{\PYZhy{}}\PY{l+m+mi}{1}\PY{p}{]}\PY{p}{]}\PY{p}{)}
\PY{n}{noisebit} \PY{o}{=} \PY{l+m+mi}{0}

\PY{c+c1}{\PYZsh{} Escolha do ruído aplicado}
\PY{k}{while} \PY{k+kc}{True}\PY{p}{:}
    \PY{n}{noise} \PY{o}{=} \PY{n+nb}{input}\PY{p}{(}\PY{l+s+s1}{\PYZsq{}}\PY{l+s+s1}{Escolha o ruído:[0]\PYZhy{} sem ruído, [1]\PYZhy{} bitflip, [2]\PYZhy{} phaseflip: }\PY{l+s+s1}{\PYZsq{}}\PY{p}{)}
    \PY{k}{if} \PY{n}{noise} \PY{o+ow}{in} \PY{p}{[}\PY{l+s+s1}{\PYZsq{}}\PY{l+s+s1}{0}\PY{l+s+s1}{\PYZsq{}}\PY{p}{,} \PY{l+s+s1}{\PYZsq{}}\PY{l+s+s1}{1}\PY{l+s+s1}{\PYZsq{}}\PY{p}{,} \PY{l+s+s1}{\PYZsq{}}\PY{l+s+s1}{2}\PY{l+s+s1}{\PYZsq{}}\PY{p}{]}\PY{p}{:}
        \PY{n}{noise} \PY{o}{=} \PY{n+nb}{int}\PY{p}{(}\PY{n}{noise}\PY{p}{)}
        \PY{k}{break}
    \PY{k}{else}\PY{p}{:}
        \PY{n+nb}{print}\PY{p}{(}\PY{l+s+s2}{\PYZdq{}}\PY{l+s+s2}{Opção inválida. Por favor, digite 0, 1, 2 ou 3.}\PY{l+s+s2}{\PYZdq{}}\PY{p}{)}

\PY{k}{if} \PY{n}{noise} \PY{o}{==} \PY{l+m+mi}{1}\PY{p}{:}
    \PY{n}{noisebit} \PY{o}{=} \PY{n}{bitflip}
\PY{k}{elif} \PY{n}{noise} \PY{o}{==} \PY{l+m+mi}{2}\PY{p}{:}
    \PY{n}{noisebit} \PY{o}{=} \PY{n}{phaseflip}
\PY{k}{elif} \PY{n}{noise} \PY{o}{==} \PY{l+m+mi}{0}\PY{p}{:}
    \PY{n}{noisebit} \PY{o}{=} \PY{l+m+mi}{1}

\PY{n}{aplicação\PYZus{}noise\PYZus{}a} \PY{o}{=} \PY{n}{noisebit} \PY{o}{*} \PY{n}{alfa\PYZus{}final}
\PY{n}{aplicação\PYZus{}noise\PYZus{}b} \PY{o}{=} \PY{n}{noisebit} \PY{o}{*} \PY{n}{beta\PYZus{}final}

\PY{n}{alfa\PYZus{}noise} \PY{o}{=} \PY{n}{aplicação\PYZus{}noise\PYZus{}a}
\PY{n}{beta\PYZus{}noise} \PY{o}{=} \PY{n}{aplicação\PYZus{}noise\PYZus{}b}

\PY{n}{alfa\PYZus{}final} \PY{o}{=} \PY{n}{alfa\PYZus{}noise}
\PY{n}{beta\PYZus{}final} \PY{o}{=} \PY{n}{beta\PYZus{}noise}
\end{Verbatim}
\end{tcolorbox}

    Por fim, para verificar se o teletransporte funcionou e ainda, se houve
ou não ruído e qual a natureza deste, a sequência de testes abaixo foi
implementada. Neles, comparamos os estados associados as probabilidades
\(\alpha\) e \(\beta\) antes(inicial) e depois(final) do teletransporte.

    \begin{tcolorbox}[breakable, size=fbox, boxrule=1pt, pad at break*=1mm,colback=cellbackground, colframe=cellborder]
\prompt{In}{incolor}{11}{\boxspacing}
\begin{Verbatim}[commandchars=\\\{\}]
\PY{k}{if} \PY{n}{np}\PY{o}{.}\PY{n}{all}\PY{p}{(}\PY{p}{(}\PY{n}{alfa\PYZus{}final} \PY{o}{==} \PY{n}{alfa\PYZus{}inicial}\PY{p}{)} \PY{o}{\PYZam{}} \PY{p}{(}\PY{n}{beta\PYZus{}final} \PY{o}{==} \PY{n}{beta\PYZus{}inicial}\PY{p}{)}\PY{p}{)}\PY{p}{:}
    \PY{n+nb}{print}\PY{p}{(}\PY{l+s+s1}{\PYZsq{}}\PY{l+s+s1}{Teletransporte concluído com sucesso e sem presença de ruídos}\PY{l+s+s1}{\PYZsq{}}\PY{p}{,} \PY{n}{alfa\PYZus{}final}\PY{o}{+}\PY{n}{beta\PYZus{}final}\PY{p}{)}
\PY{k}{else}\PY{p}{:}
    \PY{k}{if} \PY{n}{np}\PY{o}{.}\PY{n}{all}\PY{p}{(}\PY{p}{(}\PY{n}{alfa\PYZus{}inicial}\PY{p}{[}\PY{l+m+mi}{0}\PY{p}{]}\PY{p}{[}\PY{l+m+mi}{0}\PY{p}{]} \PY{o}{==} \PY{n}{alfa\PYZus{}final}\PY{p}{[}\PY{l+m+mi}{1}\PY{p}{]}\PY{p}{[}\PY{l+m+mi}{0}\PY{p}{]}\PY{p}{)} \PY{o}{\PYZam{}} \PY{p}{(}\PY{n}{alfa\PYZus{}inicial}\PY{p}{[}\PY{l+m+mi}{1}\PY{p}{]}\PY{p}{[}\PY{l+m+mi}{0}\PY{p}{]} \PY{o}{==} \PY{n}{alfa\PYZus{}final}\PY{p}{[}\PY{l+m+mi}{0}\PY{p}{]}\PY{p}{[}\PY{l+m+mi}{0}\PY{p}{]}\PY{p}{)} \PY{o}{\PYZam{}} \PY{p}{(}\PY{n}{beta\PYZus{}inicial}\PY{p}{[}\PY{l+m+mi}{1}\PY{p}{]}\PY{p}{[}\PY{l+m+mi}{0}\PY{p}{]} \PY{o}{==} \PY{n}{beta\PYZus{}final}\PY{p}{[}\PY{l+m+mi}{0}\PY{p}{]}\PY{p}{[}\PY{l+m+mi}{0}\PY{p}{]}\PY{p}{)} \PY{o}{\PYZam{}} \PY{p}{(}\PY{n}{beta\PYZus{}inicial}\PY{p}{[}\PY{l+m+mi}{0}\PY{p}{]}\PY{p}{[}\PY{l+m+mi}{0}\PY{p}{]} \PY{o}{==} \PY{n}{beta\PYZus{}final}\PY{p}{[}\PY{l+m+mi}{1}\PY{p}{]}\PY{p}{[}\PY{l+m+mi}{0}\PY{p}{]}\PY{p}{)}\PY{p}{)}\PY{p}{:}
        \PY{n+nb}{print}\PY{p}{(}\PY{l+s+s2}{\PYZdq{}}\PY{l+s+s2}{Mensagem teletransportada com ruído do tipo bitflip}\PY{l+s+s2}{\PYZdq{}}\PY{p}{,} \PY{n}{alfa\PYZus{}final}\PY{o}{+}\PY{n}{beta\PYZus{}final}\PY{p}{)}
    \PY{k}{elif} \PY{n}{np}\PY{o}{.}\PY{n}{all}\PY{p}{(}\PY{p}{(}\PY{n}{alfa\PYZus{}inicial}\PY{p}{[}\PY{l+m+mi}{0}\PY{p}{]}\PY{p}{[}\PY{l+m+mi}{0}\PY{p}{]} \PY{o}{==} \PY{n}{alfa\PYZus{}final}\PY{p}{[}\PY{l+m+mi}{0}\PY{p}{]}\PY{p}{[}\PY{l+m+mi}{0}\PY{p}{]}\PY{p}{)} \PY{o}{\PYZam{}} \PY{p}{(}\PY{n}{alfa\PYZus{}inicial}\PY{p}{[}\PY{l+m+mi}{1}\PY{p}{]}\PY{p}{[}\PY{l+m+mi}{0}\PY{p}{]} \PY{o}{==} \PY{o}{\PYZhy{}}\PY{n}{alfa\PYZus{}final}\PY{p}{[}\PY{l+m+mi}{1}\PY{p}{]}\PY{p}{[}\PY{l+m+mi}{0}\PY{p}{]}\PY{p}{)} \PY{o}{\PYZam{}} \PY{p}{(}\PY{n}{beta\PYZus{}inicial}\PY{p}{[}\PY{l+m+mi}{1}\PY{p}{]}\PY{p}{[}\PY{l+m+mi}{0}\PY{p}{]} \PY{o}{==} \PY{o}{\PYZhy{}}\PY{n}{beta\PYZus{}final}\PY{p}{[}\PY{l+m+mi}{1}\PY{p}{]}\PY{p}{[}\PY{l+m+mi}{0}\PY{p}{]}\PY{p}{)} \PY{o}{\PYZam{}}\PY{p}{(}\PY{n}{beta\PYZus{}inicial}\PY{p}{[}\PY{l+m+mi}{0}\PY{p}{]}\PY{p}{[}\PY{l+m+mi}{0}\PY{p}{]} \PY{o}{==} \PY{n}{beta\PYZus{}final}\PY{p}{[}\PY{l+m+mi}{0}\PY{p}{]}\PY{p}{[}\PY{l+m+mi}{0}\PY{p}{]}\PY{p}{)}\PY{p}{)}\PY{p}{:}
        \PY{n+nb}{print}\PY{p}{(}\PY{l+s+s2}{\PYZdq{}}\PY{l+s+s2}{Mensagem teletransportada com ruído do tipo phaseflip}\PY{l+s+s2}{\PYZdq{}}\PY{p}{,} \PY{n}{alfa\PYZus{}final}\PY{o}{+}\PY{n}{beta\PYZus{}final}\PY{p}{)}
    \PY{k}{else}\PY{p}{:}
        \PY{n+nb}{print}\PY{p}{(}\PY{l+s+s2}{\PYZdq{}}\PY{l+s+s2}{Erro de teletransporte}\PY{l+s+s2}{\PYZdq{}}\PY{p}{)}
\end{Verbatim}
\end{tcolorbox}


    Vale ressaltar que o protocolo acima é uma simplificação de uma situação
real e tem como finalidade compreender os procedimentos associados a
ele.

   


    % Add a bibliography block to the postdoc
    
    
\end{appendices}   
\end{document}
