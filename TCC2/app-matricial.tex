%%%  __  __       _        _      _       _
%%% |  \/  | __ _| |_ _ __(_) ___(_) __ _| |
%%% | |\/| |/ _` | __| '__| |/ __| |/ _` | |
%%% | |  | | (_| | |_| |  | | (__| | (_| | |
%%% |_|  |_|\__,_|\__|_|  |_|\___|_|\__,_|_|
%%%
%%% TCC de Bianca Miyabe Santos Freitas
%%% Licenciatura em Física - UFSCar, Sorocaba
%%%

\chapter{Representação Matricial dos protocolos de Emaranhamento e Teletransporte}\label{app:matricial}

As representações matriciais dos protocolos de Emaranhamento e Teletransporte nos permitem observar com maior clareza a natureza binária dos qubits numa representação de como seria seu comportamento em uma situação real. Para todos os efeitos não é considerada a origem do qubit mas sim sua natureza, ou seja, uma entidade quântica. A notação de \textit{bracket} permite que as operações sejam realizadas considerando os estados quânticos possíveis armazenados dentro do qubit como versores num espaço complexo.

Para iniciar o protocolo, consideraremos as seguintes representações binárias dos estados quânticos
\begin{equation} \label{eq:01}
\ket{0} = \begin{pmatrix}
1 \\
0
\end{pmatrix} \quad \text{e} \quad
\ket{1} = \begin{pmatrix}
0 \\
1
\end{pmatrix}.
\end{equation}

Consideraremos também que estados quânticos que dependem de mais de um qubit (emaranhados ou não), são representados pelo produto tensorial dos seus estados internos, como segue o exemplo:
\begin{equation}\label{eq:00}
\ket{00} = \begin{pmatrix}
1 \\
0
\end{pmatrix} \otimes \begin{pmatrix}
1 \\
0
\end{pmatrix} = \begin{pmatrix}
1 \\
0 \\
0 \\
0
\end{pmatrix}
\end{equation}

A representação gráfica do circuito quântico que realiza o protocolo de emaranhamento possibilita o entendimento dos procedimentos contidos no mesmo, conforme a região destacada da Figura~\ref{fig:protocoloteletransporte}

Para que o emaranhamento seja possível, ambos os qubits representados por $\ket{q_x}$ e $\ket{q_y}$ devem existir no mesmo espaço de Hilbert e portanto a operação de produto tensorial entre eles deve ser possível de modo que, considerando o estado quântico $\ket{0}$:
\begin{equation}
\ket{q_x} \otimes \ket{q_y} = \ket{0} \otimes \ket{0} = \ket{00} = \begin{pmatrix}
1 \\
0
\end{pmatrix} \otimes \begin{pmatrix}
1 \\
0
\end{pmatrix} = \begin{pmatrix}
1 \\
0 \\
0 \\
0
\end{pmatrix}
\end{equation}

Seguindo a Figura~\ref{fig:protocoloteletransporte}, aplicamos a porta Hadamard, \(\HAD\), no estado $\ket{q_x}$:

\begin{equation}\label{eq:Hqbit0}
\HAD \ket{q_x}= \frac{1}{\sqrt{2}} \begin{pmatrix*}[r]
1 & 1 \\
1 & -1
\end{pmatrix*} \begin{pmatrix}
1  \\
0 
\end{pmatrix} = \frac{1}{\sqrt{2}} \begin{pmatrix*}[r]
1 \\
1
\end{pmatrix*}
\end{equation}


Em seguida, aplicamos a porta \(\CNOT\) tendo como controle o qubit $\ket{q_x}$, após a aplicação de \(\HAD\) e o alvo o qubit $\ket{q_y}$:
\begin{equation}\label{eq:matrizbell00}
	\begin{split}
 		\CNOT \bigl( \ket{q_x}\ket{q_y} \bigr) =& 
		\begin{pmatrix}
		1 & 0 & 0 & 0 \\
		0 & 1 & 0 & 0 \\
		0 & 0 & 0 & 1 \\
		0 & 0 & 1 & 0
		\end{pmatrix} 
		\frac{1}{\sqrt{2}} \begin{pmatrix*}[r]
		1 \\
		1
		\end{pmatrix*} \otimes  \begin{pmatrix*}[r]
		1 \\
		0
		\end{pmatrix*} = \\
		&= \begin{pmatrix}
		1 & 0 & 0 & 0 \\
		0 & 1 & 0 & 0 \\
		0 & 0 & 0 & 1 \\
		0 & 0 & 1 & 0
		\end{pmatrix} \frac{1}{\sqrt{2}}\begin{pmatrix}
		1 \\
		0 \\
		1 \\
		0
		\end{pmatrix} = \frac{1}{\sqrt{2}} \begin{pmatrix}
		1 \\
		0 \\
		0 \\
		1
		\end{pmatrix}
	\end{split}	
\end{equation}

A equação também é conhecida como uma das Bases de Bell. A comparação da Base de Bell no estado $\ket{00}$ em notação de Dirac e em notação matricial é feita considerando as Equações~\eqref{eq:00} e~\eqref{eq:01} de modo que, seja \(\ket{\beta_{00}}\) a Base de Bell em notação de Dirac dada por \(\frac{1}{\sqrt{2}} \bigl(\ket{00} + \ket{11}\bigr)\), reescrevendo teremos:
\begin{equation}\label{eq:comparacaobeta}
\ket{\beta_{00}} = \frac{1}{\sqrt{2}} \left[ \begin{pmatrix}
1 \\
0
\end{pmatrix} \otimes \begin{pmatrix}
1 \\
0
\end{pmatrix} + \begin{pmatrix}
0 \\
1
\end{pmatrix} \otimes \begin{pmatrix}
0 \\
1
\end{pmatrix}\right] = \frac{1}{\sqrt{2}} \left[ \begin{pmatrix}
1 \\
0 \\
0 \\
0 
\end{pmatrix} + \begin{pmatrix}
0 \\
0 \\
0 \\
1 
\end{pmatrix}\right] = \frac{1}{\sqrt{2}} \begin{pmatrix}
1 \\
0 \\
0 \\
1 
\end{pmatrix}
\end{equation}

O resultado obtido em~\eqref{eq:comparacaobeta} é exatamente o mesmo obtido na~\eqref{eq:bell00}.
A próxima etapa do protocolo de teletransporte, consiste na aplicação da porta lógica quântica \(\CNOT\). Portanto, seja o estado $\ket{\psi_0}$ descrito por:
\begin{equation}\label{eq:psi0}
\ket{\psi_0}=\ket{\psi}\ket{\beta_{00}}=\left[\alpha \begin{pmatrix}
1 \\
0 
\end{pmatrix} + \beta \begin{pmatrix}
0 \\
1
\end{pmatrix}\right] \left[\frac{1}{\sqrt{2}} \begin{pmatrix}
1 \\
0 \\
0 \\
1
\end{pmatrix}\right]
\end{equation}

Conforme Figura~\ref{fig:protocoloteletransporte}, a porta \(\CNOT\) possui como controle o qubit descrito pelo estado $\ket{\psi}$ e como alvo o par emaranhado presente no local A $\ket{\beta_{00}}$. Sua atuação apenas ocorrerá quando o qubit de controle estiver no estado $\ket{1}$. Desse modo, reescrevendo~\eqref{eq:psi0}

\begin{equation}\label{eq:psi0cnot}
\ket{\psi_0} = \frac{1}{\sqrt{2}}\left\{\left[\alpha \begin{pmatrix}
1 \\
0 
\end{pmatrix}  \begin{pmatrix}
1 \\
0 \\
0 \\
1
\end{pmatrix}\right] + \left[\beta \begin{pmatrix}
0 \\
1
\end{pmatrix}  \begin{pmatrix}
1 \\
0 \\
0 \\
1
\end{pmatrix}\right] \right\}
\end{equation}

evidenciando os produtos tensoriais de~\eqref{eq:psi0cnot}
\begin{equation}\label{eq:prodtenspsi0}
	\begin{split}
\ket{\psi_0} &= \frac{1}{\sqrt{2}}\left\{\alpha \begin{pmatrix}
1 \\
0 
\end{pmatrix} \left[ \begin{pmatrix}
1 \\
0 
\end{pmatrix} \otimes \begin{pmatrix}
1 \\
0
\end{pmatrix} + \begin{pmatrix}
0 \\
1
\end{pmatrix} \otimes \begin{pmatrix}
0 \\
1
\end{pmatrix}\right]\right\} \\
&+ \frac{1}{\sqrt{2}}\left\{\beta \begin{pmatrix}
0 \\
1
\end{pmatrix} \left[ \begin{pmatrix}
1 \\
0 
\end{pmatrix} \otimes \begin{pmatrix}
1 \\
0
\end{pmatrix} + \begin{pmatrix}
0 \\
1
\end{pmatrix} \otimes \begin{pmatrix}
0 \\
1
\end{pmatrix}\right]\right\}
	\end{split}
\end{equation}

Reagrupando temos portanto, quatro possíveis estados formados por 3-qubits sendo eles:

\begin{equation}\label{eq:3qubit}
	\begin{split}
\ket{000} &= \left[ \begin{pmatrix}
1 \\
0
\end{pmatrix} \otimes \begin{pmatrix}
1 \\
0
\end{pmatrix} \otimes \begin{pmatrix}
1 \\
0
\end{pmatrix}\right] = \begin{pmatrix}
1 & 0 & 0 & 0 & 0 & 0 & 0 & 0
\end{pmatrix}^T \\
\ket{011} &= \left[ \begin{pmatrix}
1 \\
0
\end{pmatrix} \otimes \begin{pmatrix}
0 \\
1
\end{pmatrix} \otimes \begin{pmatrix}
0 \\
1
\end{pmatrix}\right] = \begin{pmatrix}
0 & 0 & 0 & 1 & 0 & 0 & 0 & 0
\end{pmatrix}^T \\
\ket{100} &= \left[ \begin{pmatrix}
0 \\
1
\end{pmatrix} \otimes \begin{pmatrix}
1 \\
0
\end{pmatrix} \otimes \begin{pmatrix}
1 \\
0
\end{pmatrix}\right] = \begin{pmatrix}
0 & 0 & 0 & 0 & 1 & 0 & 0 & 0
\end{pmatrix}^T \\
\ket{111} &=\left[ \begin{pmatrix}
0 \\
1
\end{pmatrix} \otimes \begin{pmatrix}
0 \\
1
\end{pmatrix} \otimes \begin{pmatrix}
0 \\
1
\end{pmatrix}\right] = \begin{pmatrix}
0 & 0 & 0 & 0 & 0 & 0 & 0 & 1
\end{pmatrix}^T
	\end{split}
\end{equation}

Somando todos os estados possíveis temos:
\begin{equation}\label{eq:3estados}
\ket{000}+\ket{001}+\ket{100}+\ket{111}= \begin{pmatrix}
1 & 0 & 0 & 1 & 1 & 0 & 0 & 1
\end{pmatrix}^T
\end{equation}

Como a porta \(\CNOT\) está dimensionada para dois estados e, nesse caso, possuímos três, iremos aplicar a matriz identidade para dimensioná-la. Desse modo:

\begin{equation}\label{eq:cnotI}
\CNOT \otimes I = \begin{pmatrix}
		1 & 0 & 0 & 0 \\
		0 & 1 & 0 & 0 \\
		0 & 0 & 0 & 1 \\
		0 & 0 & 1 & 0
		\end{pmatrix} \otimes \begin{pmatrix}
		1 & 0 \\
		0 & 1
		\end{pmatrix} = \begin{pmatrix}
		1 & 0 & 0 & 0 & 0 & 0 & 0 & 0 \\
		0 & 1 & 0 & 0 & 0 & 0 & 0 & 0 \\
		0 & 0 & 1 & 0 & 0 & 0 & 0 & 0 \\
		0 & 0 & 0 & 1 & 0 & 0 & 0 & 0 \\
		0 & 0 & 0 & 0 & 0 & 0 & 1 & 0 \\
		0 & 0 & 0 & 0 & 0 & 0 & 0 & 1 \\
		0 & 0 & 0 & 0 & 1 & 0 & 0 & 0 \\
		0 & 0 & 0 & 0 & 0 & 1 & 0 & 0 		
		\end{pmatrix}
\end{equation}

Aplicando~\eqref{eq:cnotI} em~\eqref{eq:3estados}, teremos o estado:

\begin{equation}\label{eq:cnot3estados}
	\begin{split}
		\CNOT (\ket{000}\ket{001}\ket{100}\ket{111}) &= \begin{pmatrix}
		1 & 0 & 0 & 0 & 0 & 0 & 0 & 0 \\
		0 & 1 & 0 & 0 & 0 & 0 & 0 & 0 \\
		0 & 0 & 1 & 0 & 0 & 0 & 0 & 0 \\
		0 & 0 & 0 & 1 & 0 & 0 & 0 & 0 \\
		0 & 0 & 0 & 0 & 0 & 0 & 1 & 0 \\
		0 & 0 & 0 & 0 & 0 & 0 & 0 & 1 \\
		0 & 0 & 0 & 0 & 1 & 0 & 0 & 0 \\
		0 & 0 & 0 & 0 & 0 & 1 & 0 & 0 		
		\end{pmatrix} \begin{pmatrix}
		1 \\
		0 \\
		0 \\
		1 \\
		1 \\
		0 \\
		0 \\
		1
		\end{pmatrix} = \\
		&=\begin{pmatrix}
		1 & 0 &	0 & 1 &	0 &	1 &	1 &	0
		\end{pmatrix}^T
	\end{split}
\end{equation}

Usando a lógica de construção de~\eqref{eq:3estados}, podemos separar as tríades obtidas em~\eqref{eq:cnot3estados}, de modo que:

\begin{equation}
	\begin{split}
	\begin{pmatrix}
	1 \\
	0 \\
	0 \\
	1 \\
	0 \\
	1 \\
	1 \\
	0
	\end{pmatrix} = \begin{pmatrix}
	1 \\
	0 \\
	0 \\
	0 \\
	0 \\
	0 \\
	0 \\
	0
	\end{pmatrix} + \begin{pmatrix}
	0 \\
	0 \\
	0 \\
	1 \\
	0 \\
	0 \\
	0 \\
	0
	\end{pmatrix} + \begin{pmatrix}
	0 \\
	0 \\
	0 \\
	0 \\
	0 \\
	0 \\
	1 \\
	0
	\end{pmatrix} + \begin{pmatrix}
	0 \\
	0 \\
	0 \\
	0 \\
	0 \\
	1 \\
	0 \\
	0
	\end{pmatrix} = \ket{000}+\ket{011}+\ket{110}+\ket{101}
	\end{split}
\end{equation}

Portanto, reorganizando e agrupando os estados definidos em~\eqref{eq:cnot3estados}, o estado $\ket{\psi_1}$ pode ser descrito por:
 
\begin{equation}\label{eq:psi1}
\ket{\psi_1} = \frac{1}{\sqrt{2}} \left\{  \alpha \begin{pmatrix}
1 \\
0
\end{pmatrix} \otimes \left[ \begin{pmatrix}
1 \\
0 \\
0 \\
0
\end{pmatrix} + \begin{pmatrix}
0 \\
0 \\
0 \\
1
\end{pmatrix}\right] + \beta \begin{pmatrix}
0 \\
1
\end{pmatrix} \otimes \left[ \begin{pmatrix}
0 \\
0 \\
1 \\
0
\end{pmatrix} + \begin{pmatrix}
0 \\
1 \\
0 \\
0
\end{pmatrix}\right] \right\}
\end{equation}

A próxima etapa, consiste na aplicação da porta \(\HAD\) no qubit $\ket{\psi}$, de modo que:

\begin{equation}\label{eq:halfabeta}
  	\begin{split}
		\HAD \ket{\psi} &= \frac{1}{\sqrt{2}} \begin{pmatrix*}[r]
		1 & 1 \\
		1 & -1
		\end{pmatrix*} \frac{1}{\sqrt{2}}\left(\alpha \begin{pmatrix}
		1 \\
		0 
		\end{pmatrix} + \beta \begin{pmatrix}
		0 \\
		1
		\end{pmatrix} \right) = \frac{1}{2} \left[\alpha \begin{pmatrix}
		1 \\
		1 
		\end{pmatrix} + \beta \begin{pmatrix*}[r]
		1 \\
		-1
		\end{pmatrix*} \right]
  	\end{split}
\end{equation}

Os resultados de~\eqref{eq:halfabeta} podem ser reescritos como:
\begin{equation} \label{eq:alfabetasoma}
  \begin{split}
\frac{1}{2} \alpha \begin{pmatrix}
1 \\
1 
\end{pmatrix} &= \frac{1}{2} \alpha \begin{pmatrix}
1 \\
0 
\end{pmatrix} + \begin{pmatrix}
0 \\
1 
\end{pmatrix} \\
\frac{1}{2} \beta \begin{pmatrix*}[r]
1 \\
-1
\end{pmatrix*} &= \frac{1}{2} \beta \begin{pmatrix}
1 \\
0
\end{pmatrix} - \begin{pmatrix}
0 \\
1
\end{pmatrix} 
  \end{split}
\end{equation}

Portanto, reagrupando~\eqref{eq:psi1} e~\eqref{eq:halfabeta} teremos o estado $\ket{\psi_2}$ que é descrito por:

\begin{align}\label{eq:psi2matrix}
\ket{\psi_2} = \frac{1}{2} \left\{\left[ \alpha \begin{pmatrix}
1 \\
1 
\end{pmatrix} \begin{pmatrix}
1 \\
0 \\
0 \\
0
\end{pmatrix} +\begin{pmatrix}
0 \\
0 \\
0 \\
1
\end{pmatrix} \right] + \left[\beta \begin{pmatrix*}[r]
1 \\
-1
\end{pmatrix*} \begin{pmatrix}
0 \\
1 \\
0 \\
0
\end{pmatrix} + \begin{pmatrix}
0 \\
0 \\
1 \\
0
\end{pmatrix}\right]\right\}
\end{align}



No resultado~\eqref{eq:psi2matrix}, podemos evidenciar as seguintes relações:
\begin{equation}\label{eq:prodtens1}
	\begin{split}
&\begin{pmatrix}
1 \\
0 \\
0 \\
0
\end{pmatrix} = \begin{pmatrix}
1 \\
0
\end{pmatrix} \otimes \begin{pmatrix}
1 \\
0
\end{pmatrix},
\begin{pmatrix}
0 \\
0 \\
0 \\
1
\end{pmatrix} = \begin{pmatrix}
0 \\
1
\end{pmatrix} \otimes \begin{pmatrix}
0 \\
1
\end{pmatrix}, \\
&\begin{pmatrix}
0 \\
1 \\
0 \\
0
\end{pmatrix} = \begin{pmatrix}
1 \\
0
\end{pmatrix} \otimes \begin{pmatrix}
0 \\
1
\end{pmatrix},
\begin{pmatrix}
0 \\
0 \\
1 \\
0
\end{pmatrix} = \begin{pmatrix}
0 \\
1
\end{pmatrix} \otimes \begin{pmatrix}
1 \\
0
\end{pmatrix} 
	\end{split}
\end{equation}

Substituindo as relações de~\eqref{eq:prodtens1} e~\eqref{eq:alfabetasoma} em~\eqref{eq:psi2matrix}
\begin{equation}\label{eq:help}
  \begin{split}
\ket{\psi_2} &= \frac{1}{2}\left\{\left[\alpha\begin{pmatrix}
1 \\
0
\end{pmatrix}+\alpha\begin{pmatrix}
0 \\
1
\end{pmatrix}\right]\left[\left(\begin{pmatrix}
1 \\
0
\end{pmatrix} \otimes \begin{pmatrix}
1 \\
0
\end{pmatrix}\right) + \left(\begin{pmatrix}
0 \\
1
\end{pmatrix} \otimes \begin{pmatrix}
0 \\
1
\end{pmatrix}\right)\right]\right\} \nonumber \\
&= \frac{1}{2}\left\{\left[\beta\begin{pmatrix}
1 \\
0
\end{pmatrix}-\beta\begin{pmatrix}
0 \\
1
\end{pmatrix}\right]\left[\left(\begin{pmatrix}
1 \\
0
\end{pmatrix} \otimes \begin{pmatrix}
0 \\
1
\end{pmatrix}\right) + \left(\begin{pmatrix}
0 \\
1
\end{pmatrix} \otimes \begin{pmatrix}
1 \\
0
\end{pmatrix}\right)\right]\right\}
  \end{split}
\end{equation}
Podemos reorganizar~\eqref{eq:help} evidenciando os termos correspondentes às possíveis medidas realizadas no local A e no resultado correspondente do par emaranhado no local B:

\begin{equation}\label{eq:final}
  \begin{split}
\ket{\psi_2} &=\frac{1}{2} \left\{ \left[\begin{pmatrix}
1 \\
0
\end{pmatrix} \otimes \begin{pmatrix}
1 \\
0
\end{pmatrix}\right] \left[\alpha \begin{pmatrix}
1 \\
0
\end{pmatrix} + \beta \begin{pmatrix}
0 \\
1
\end{pmatrix}\right]\right\} \nonumber \\
&+\frac{1}{2} \left\{ \left[\begin{pmatrix}
0 \\
1
\end{pmatrix} \otimes \begin{pmatrix}
0 \\
1
\end{pmatrix}\right] \left[\alpha \begin{pmatrix}
0 \\
1
\end{pmatrix} - \beta \begin{pmatrix}
1 \\
0
\end{pmatrix}\right]\right\} \nonumber \\
&+\frac{1}{2} \left\{\left[ \begin{pmatrix}
1 \\
0
\end{pmatrix} \otimes \begin{pmatrix}
0 \\
1
\end{pmatrix}\right] \left[\alpha \begin{pmatrix}
0 \\
1
\end{pmatrix} + \beta \begin{pmatrix}
1 \\
0
\end{pmatrix}\right]\right\} \nonumber \\
&+\frac{1}{2} \left\{ \left[ \begin{pmatrix}
0 \\
1
\end{pmatrix} \otimes \begin{pmatrix}
1 \\
0
\end{pmatrix}\right] \left[\alpha \begin{pmatrix}
1 \\
0
\end{pmatrix} - \beta \begin{pmatrix}
0 \\
1
\end{pmatrix}\right] \right\}
  \end{split}
\end{equation}

O resultado acima é exatamente mesmo obtido na Tabela~\ref{tab:medidas}.

\clearpage

\includepdf[pages=-]{PROTOCOLO_TELETRANSPORTE_V1-Ruídos.pdf}

