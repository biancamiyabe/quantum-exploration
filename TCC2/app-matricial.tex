%%%  __  __       _        _      _       _
%%% |  \/  | __ _| |_ _ __(_) ___(_) __ _| |
%%% | |\/| |/ _` | __| '__| |/ __| |/ _` | |
%%% | |  | | (_| | |_| |  | | (__| | (_| | |
%%% |_|  |_|\__,_|\__|_|  |_|\___|_|\__,_|_|
%%%
%%% TCC de Bianca Miyabe Santos Freitas
%%% Licenciatura em Física - UFSCar, Sorocaba
%%%

\chapter{Representação Matricial dos protocolos de Emaranhamento e Teletransporte}

As representações matriciais dos protocolos de Emaranhamento e Teletransporte nos permitem observar com maior clareza a natureza binária dos qubits numa representação de como seria seu comportamento em uma situação real. Para todos os efeitos não é considerada a origem do qubit mas sim sua natureza, ou seja, uma entidade quântica. A notação de \textit{bracket} permite que as operações sejam realizadas considerando os estados quânticos possíveis armazenados dentro do qubit como versores num espaço complexo.

Para iniciar o protocolo, consideraremos as seguintes representações binárias dos estados quânticos
\begin{equation} \label{eq:01}
\ket{0} = \begin{pmatrix}
1 \\
0
\end{pmatrix} \quad \text{e} \quad
\ket{1} = \begin{pmatrix}
0 \\
1
\end{pmatrix}.
\end{equation}

Consideraremos também que estados quânticos que dependem de mais de um qubit (emaranhados ou não), são representados pelo produto tensorial dos seus estados internos, como segue o exemplo:
\begin{equation}\label{eq:00}
\ket{00} = \begin{pmatrix}
1 \\
0
\end{pmatrix} \otimes \begin{pmatrix}
1 \\
0
\end{pmatrix} = \begin{pmatrix}
1 \\
0 \\
0 \\
0
\end{pmatrix}
\end{equation}

A representação gráfica do circuito quântico que realiza o protocolo de emaranhamento possibilita o entendimento dos procedimentos contidos no mesmo, conforme a região destacada da Figura~\ref{fig:protocoloteletransporte}

Para que o emaranhamento seja possível, ambos os qubits representados por $\ket{q_x}$ e $\ket{q_y}$ devem existir no mesmo espaço de Hilbert e portanto a operação de produto tensorial entre eles deve ser possível de modo que, considerando o estado quântico $\ket{0}$:
\begin{equation}
\ket{q_x} \otimes \ket{q_y} = \ket{0} \otimes \ket{0} = \ket{00} = \begin{pmatrix}
1 \\
0
\end{pmatrix} \otimes \begin{pmatrix}
1 \\
0
\end{pmatrix} = \begin{pmatrix}
1 \\
0 \\
0 \\
0
\end{pmatrix}
\end{equation}

Seguindo a Figura~\ref{fig:protocoloteletransporte}, aplicamos a porta Hadamard, \(\HAD\), multiplicando-a pela matriz identidade \(I\), obtendo:
\begin{equation}
\HAD \otimes I = \frac{1}{\sqrt{2}} \begin{pmatrix*}[r]
1 & 1 \\
1 & -1
\end{pmatrix*} \otimes \begin{pmatrix}
1 & 0 \\
0 & 1
\end{pmatrix} = \frac{1}{\sqrt{2}} \begin{pmatrix*}[r]
1 & 0 & 1 & 0 \\
0 & 1 & 0 & 1 \\
1 & 0 & -1 & 0 \\
0 & 1 & 0 & -1
\end{pmatrix*}
\end{equation}
e, aplicando o resultado acima em~\eqref{eq:00}
\begin{equation}
\HAD \ket{00} = \frac{1}{\sqrt{2}} \begin{pmatrix*}[r]
1 & 0 & 1 & 0 \\
0 & 1 & 0 & 1 \\
1 & 0 & -1 & 0 \\
0 & 1 & 0 & -1
\end{pmatrix*} \begin{pmatrix}
1 \\
0 \\
0 \\
0
\end{pmatrix} = \frac{1}{\sqrt{2}} \begin{pmatrix}
1 \\
0 \\
1 \\
0
\end{pmatrix}.
\end{equation}

Em seguida, aplicamos a porta \(\CNOT\):
\begin{equation}
  \CNOT \bigl( \HAD \ket{00} \bigr) =
\begin{pmatrix}
1 & 0 & 0 & 0 \\
0 & 1 & 0 & 0 \\
0 & 0 & 0 & 1 \\
0 & 1 & 1 & 0
\end{pmatrix}
\frac{1}{\sqrt{2}}\begin{pmatrix}
1 \\
0 \\
1 \\
0
\end{pmatrix} = \frac{1}{\sqrt{2}} \begin{pmatrix}
1 \\
0 \\
0 \\
1
\end{pmatrix}
\end{equation}

A equação também é conhecida como uma das Bases de Bell. A comparação da Base de Bell no estado $\ket{00}$ em notação de Dirac e em notação matricial é feita considerando as Equações~\eqref{eq:00} e~\eqref{eq:01} de modo que, seja \(\ket{\beta_{00}}\) a Base de Bell em notação de Dirac dada por \(\frac{1}{\sqrt{2}} \bigl(\ket{00} + \ket{11}\bigr)\), reescrevendo teremos:
\begin{equation}\label{eq:comparacaobeta}
\ket{\beta_{00}} = \frac{1}{\sqrt{2}} \left[ \begin{pmatrix}
1 \\
0
\end{pmatrix} \otimes \begin{pmatrix}
1 \\
0
\end{pmatrix} + \begin{pmatrix}
0 \\
1
\end{pmatrix} \otimes \begin{pmatrix}
0 \\
1
\end{pmatrix}\right] = \frac{1}{\sqrt{2}} \left[ \begin{pmatrix}
1 \\
0 \\
0 \\
0 
\end{pmatrix} + \begin{pmatrix}
0 \\
0 \\
0 \\
1 
\end{pmatrix}\right] = \frac{1}{\sqrt{2}} \begin{pmatrix}
1 \\
0 \\
0 \\
1 
\end{pmatrix}
\end{equation}

O resultado obtido em~\eqref{eq:comparacaobeta} é exatamente o mesmo obtido na~\eqref{eq:bell00}.
A próxima etapa do protocolo de teletransporte, consiste na aplicação da porta lógica quântica \(\CNOT\). Portanto, seja o estado $\ket{\psi_0}$ descrito por:
\begin{equation}\label{eq:psi0}
\ket{\psi_0}=\ket{\psi}\ket{\beta_{00}}=\left[\alpha \begin{pmatrix}
1 \\
0 
\end{pmatrix} + \beta \begin{pmatrix}
0 \\
1
\end{pmatrix}\right] \left[\frac{1}{\sqrt{2}} \begin{pmatrix}
1 \\
0 \\
0 \\
1
\end{pmatrix}\right]
\end{equation}

Conforme Figura~\ref{fig:protocoloteletransporte}, a porta \(\CNOT\) possui como controle o qubit descrito pelo estado $\ket{\psi}$ e como alvo o par emaranhado presente no Local A $\ket{\beta_{00}}$. Sua atuação apenas ocorrerá quando o qubit de controle estiver no estado $\ket{1}$. Desse modo, reescrevendo~\eqref{eq:psi0}

\begin{equation}\label{eq:psi0cnot}
\ket{\psi_0} = \frac{1}{\sqrt{2}}\left\{\left[\alpha \begin{pmatrix}
1 \\
0 
\end{pmatrix}  \begin{pmatrix}
1 \\
0 \\
0 \\
1
\end{pmatrix}\right] + \left[\beta \begin{pmatrix}
0 \\
1
\end{pmatrix}  \begin{pmatrix}
1 \\
0 \\
0 \\
1
\end{pmatrix}\right] \right\}
\end{equation}

evidenciando os produtos tensoriais de~\eqref{eq:psi0cnot}
\begin{equation}
	\begin{split}
\ket{\psi_0} &= \frac{1}{\sqrt{2}}\left\{\alpha \begin{pmatrix}
1 \\
0 
\end{pmatrix} \left[ \begin{pmatrix}
1 \\
0 
\end{pmatrix} \otimes \begin{pmatrix}
1 \\
0
\end{pmatrix} + \begin{pmatrix}
0 \\
1
\end{pmatrix} \otimes \begin{pmatrix}
0 \\
1
\end{pmatrix}\right]\right\} \\
&+ \frac{1}{\sqrt{2}}\left\{\beta \begin{pmatrix}
0 \\
1
\end{pmatrix} \left[ \begin{pmatrix}
1 \\
0 
\end{pmatrix} \otimes \begin{pmatrix}
1 \\
0
\end{pmatrix} + \begin{pmatrix}
0 \\
1
\end{pmatrix} \otimes \begin{pmatrix}
0 \\
1
\end{pmatrix}\right]\right\}
	\end{split}
\end{equation}

Reagrupando temos portanto, quatro possíveis estados formados por 3-qubits sendo eles:

\begin{equation}\label{eq:3qubit}
	\begin{split}
\ket{000} &= \left[ \begin{pmatrix}
1 \\
0
\end{pmatrix} \otimes \begin{pmatrix}
1 \\
0
\end{pmatrix} \otimes \begin{pmatrix}
1 \\
0
\end{pmatrix}\right] \\
\ket{011} &= \left[ \begin{pmatrix}
1 \\
0
\end{pmatrix} \otimes \begin{pmatrix}
0 \\
1
\end{pmatrix} \otimes \begin{pmatrix}
0 \\
1
\end{pmatrix}\right] \\
\ket{100} &= \left[ \begin{pmatrix}
0 \\
1
\end{pmatrix} \otimes \begin{pmatrix}
1 \\
0
\end{pmatrix} \otimes \begin{pmatrix}
1 \\
0
\end{pmatrix}\right] \\
\ket{111} &=\left[ \begin{pmatrix}
0 \\
1
\end{pmatrix} \otimes \begin{pmatrix}
0 \\
1
\end{pmatrix} \otimes \begin{pmatrix}
0 \\
1
\end{pmatrix}\right]
	\end{split}
\end{equation}

Em~\eqref{eq:3qubit}, o primeiro da tríade representa o qubit de controle e os dois demais, os qubits alvos. Seguindo as definições de ~\ref{definition:cnot} teremos a inversão do segundo qubit de $\ket{100} \rightarrow \ket{110}$ e de $\ket{111} \rightarrow \ket{101}$. Portanto, teremos o estado $\ket{\psi_1}$ descrito por:

\begin{equation}
	\begin{split}
\ket{\psi_1} &= \frac{1}{\sqrt{2}}\left\{\left[\alpha \begin{pmatrix}
1 \\
0 
\end{pmatrix}  \left(\begin{pmatrix}
1 \\
0 
\end{pmatrix} \otimes \begin{pmatrix}
1 \\
0
\end{pmatrix} + \begin{pmatrix}
0 \\
1
\end{pmatrix} \otimes \begin{pmatrix}
0 \\
1
\end{pmatrix}\right)\right] + \left[\beta \begin{pmatrix}
0 \\
1
\end{pmatrix}  \left(\begin{pmatrix}
0 \\
1 
\end{pmatrix} \otimes \begin{pmatrix}
1 \\
0
\end{pmatrix} + \begin{pmatrix}
1 \\
0
\end{pmatrix} \otimes \begin{pmatrix}
0 \\
1
\end{pmatrix}\right)\right] \right\} \\
&=\frac{1}{\sqrt{2}}\left\{\left[\alpha \begin{pmatrix}
1 \\
0 
\end{pmatrix}  \begin{pmatrix}
1 \\
0 \\
0 \\
1
\end{pmatrix}\right] + \left[\beta \begin{pmatrix} 
0 \\
1
\end{pmatrix} \begin{pmatrix}
0 \\
1 \\
1 \\
0
\end{pmatrix}\right]\right\}
  \end{split}
\end{equation}


Está portanto definido o estado $\ket{\psi_1}$. A próxima etapa, consiste na aplicação da porta \(\HAD\) no qubit $\ket{\psi_{1}}$, de modo que:
\begin{equation}\label{eq:halfabeta}
  \begin{split}
\HAD \ket{\psi_1} &= \frac{1}{\sqrt{2}} \begin{pmatrix*}[r]
1 & 1 \\
1 & -1
\end{pmatrix*} \left(\alpha \begin{pmatrix}
1 \\
0 
\end{pmatrix} + \beta \begin{pmatrix}
0 \\
1
\end{pmatrix} \right) \\
&\HAD \alpha \begin{pmatrix}
1 \\
0 
\end{pmatrix} = \frac{1}{2} \alpha \begin{pmatrix}
1 \\
1 
\end{pmatrix} \\
&\HAD \beta \begin{pmatrix}
0 \\
1
\end{pmatrix} =  \frac{1}{2} \beta \begin{pmatrix*}[r]
1 \\
-1
\end{pmatrix*}
  \end{split}
\end{equation}

Os resultados de~\eqref{eq:halfabeta} podem ser reescritos como:
\begin{equation} \label{eq:alfabetasoma}
  \begin{split}
\frac{1}{2} \alpha \begin{pmatrix}
1 \\
1 
\end{pmatrix} &= \frac{1}{2} \alpha \begin{pmatrix}
1 \\
0 
\end{pmatrix} + \begin{pmatrix}
0 \\
1 
\end{pmatrix} \\
\frac{1}{2} \beta \begin{pmatrix*}[r]
1 \\
-1
\end{pmatrix*} &= \frac{1}{2} \beta \begin{pmatrix}
1 \\
0
\end{pmatrix} - \begin{pmatrix}
0 \\
1
\end{pmatrix} 
  \end{split}
\end{equation}

Portanto, o estado $\ket{\psi_2}$ é descrito por:
\begin{align}\label{eq:psi2matrix}
\ket{\psi_2} = \frac{1}{2} \left\{\left[ \alpha \begin{pmatrix}
1 \\
1 
\end{pmatrix} \begin{pmatrix}
1 \\
0 \\
0 \\
1
\end{pmatrix}\right] + \left[\beta \begin{pmatrix*}[r]
1 \\
-1
\end{pmatrix*} \begin{pmatrix}
0 \\
1 \\
1 \\
0
\end{pmatrix}\right]\right\}
\end{align}

Reescrevendo~\eqref{eq:psi2matrix}
\begin{equation}\label{eq:psi2aberto}
\ket{\psi_2} = \frac{1}{2} \left\{\left[ \alpha \begin{pmatrix}
1 \\
1 
\end{pmatrix} \begin{pmatrix}
1 \\
0 \\
0 \\
0
\end{pmatrix} +\begin{pmatrix}
0 \\
0 \\
0 \\
1
\end{pmatrix} \right] + \left[\beta \begin{pmatrix*}[r]
1 \\
-1
\end{pmatrix*} \begin{pmatrix}
0 \\
1 \\
0 \\
0
\end{pmatrix} + \begin{pmatrix}
0 \\
0 \\
1 \\
0
\end{pmatrix}\right]\right\}
\end{equation}

No resultado~\eqref{eq:psi2aberto}, podemos evidenciar as seguintes relações:
\begin{equation}\label{eq:prodtens1}
	\begin{split}
&\begin{pmatrix}
1 \\
0 \\
0 \\
0
\end{pmatrix} = \begin{pmatrix}
1 \\
0
\end{pmatrix} \otimes \begin{pmatrix}
1 \\
0
\end{pmatrix},
\begin{pmatrix}
0 \\
0 \\
0 \\
1
\end{pmatrix} = \begin{pmatrix}
0 \\
1
\end{pmatrix} \otimes \begin{pmatrix}
0 \\
1
\end{pmatrix}, \\
&\begin{pmatrix}
0 \\
1 \\
0 \\
0
\end{pmatrix} = \begin{pmatrix}
1 \\
0
\end{pmatrix} \otimes \begin{pmatrix}
0 \\
1
\end{pmatrix},
\begin{pmatrix}
0 \\
0 \\
1 \\
0
\end{pmatrix} = \begin{pmatrix}
0 \\
1
\end{pmatrix} \otimes \begin{pmatrix}
1 \\
0
\end{pmatrix} 
	\end{split}
\end{equation}

Substituindo as relações de~\eqref{eq:prodtens1} e~\eqref{eq:alfabetasoma} em~\eqref{eq:psi2aberto}
\begin{equation}\label{eq:help}
  \begin{split}
\ket{\psi_2} &= \frac{1}{2}\left\{\left[\alpha\begin{pmatrix}
1 \\
0
\end{pmatrix}+\alpha\begin{pmatrix}
0 \\
1
\end{pmatrix}\right]\left[\left(\begin{pmatrix}
1 \\
0
\end{pmatrix} \otimes \begin{pmatrix}
1 \\
0
\end{pmatrix}\right) + \left(\begin{pmatrix}
0 \\
1
\end{pmatrix} \otimes \begin{pmatrix}
0 \\
1
\end{pmatrix}\right)\right]\right\} \nonumber \\
&= \frac{1}{2}\left\{\left[\beta\begin{pmatrix}
1 \\
0
\end{pmatrix}-\beta\begin{pmatrix}
0 \\
1
\end{pmatrix}\right]\left[\left(\begin{pmatrix}
1 \\
0
\end{pmatrix} \otimes \begin{pmatrix}
0 \\
1
\end{pmatrix}\right) + \left(\begin{pmatrix}
0 \\
1
\end{pmatrix} \otimes \begin{pmatrix}
1 \\
0
\end{pmatrix}\right)\right]\right\}
  \end{split}
\end{equation}
Podemos reorganizar~\eqref{eq:help} evidenciando os termos correspondentes às possíveis medidas realizadas no Local A e no resultado correspondente do par emaranhado no Local B:
\begin{equation}
  \begin{split}
\ket{\psi_2} &=\frac{1}{2} \left\{ \left[\begin{pmatrix}
1 \\
0
\end{pmatrix} \otimes \begin{pmatrix}
1 \\
0
\end{pmatrix}\right] \left[\alpha \begin{pmatrix}
1 \\
0
\end{pmatrix} + \beta \begin{pmatrix}
0 \\
1
\end{pmatrix}\right]\right\} \nonumber \\
&+\frac{1}{2} \left\{ \left[\begin{pmatrix}
0 \\
1
\end{pmatrix} \otimes \begin{pmatrix}
0 \\
1
\end{pmatrix}\right] \left[\alpha \begin{pmatrix}
0 \\
1
\end{pmatrix} - \beta \begin{pmatrix}
1 \\
0
\end{pmatrix}\right]\right\} \nonumber \\
&+\frac{1}{2} \left\{\left[ \begin{pmatrix}
1 \\
0
\end{pmatrix} \otimes \begin{pmatrix}
0 \\
1
\end{pmatrix}\right] \left[\alpha \begin{pmatrix}
0 \\
1
\end{pmatrix} + \beta \begin{pmatrix}
1 \\
0
\end{pmatrix}\right]\right\} \nonumber \\
&+\frac{1}{2} \left\{ \left[ \begin{pmatrix}
0 \\
1
\end{pmatrix} \otimes \begin{pmatrix}
1 \\
0
\end{pmatrix}\right] \left[\alpha \begin{pmatrix}
1 \\
0
\end{pmatrix} - \beta \begin{pmatrix}
0 \\
1
\end{pmatrix}\right] \right\}
  \end{split}
\end{equation}

O resultado acima é exatamente mesmo obtido na Tabela~\ref{tab:medidas}.

