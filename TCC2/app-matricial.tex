%%%  __  __       _        _      _       _
%%% |  \/  | __ _| |_ _ __(_) ___(_) __ _| |
%%% | |\/| |/ _` | __| '__| |/ __| |/ _` | |
%%% | |  | | (_| | |_| |  | | (__| | (_| | |
%%% |_|  |_|\__,_|\__|_|  |_|\___|_|\__,_|_|
%%%
%%% TCC de Bianca Miyabe Santos Freitas
%%% Licenciatura em Física - UFSCar, Sorocaba
%%%

\chapter{Representação Matricial dos protocolos de Emaranhamento e Teletransporte}

As representações matriciais dos protocolos de Emaranhamento e Teletransporte nos permitem observar com maior clareza a natureza binária dos qubits numa representação de como seria seu comportamento em uma situação real. Para todos os efeitos não é considerada a origem do qubit mas sim sua natureza, ou seja, uma entidade quântica. A notação de \textit{bracket} permite que as operações sejam realizadas considerando os estados quânticos possíveis armazenados dentro do qubit como versores num espaço complexo.

Para iniciar o protocolo, consideraremos as seguintes representações binárias dos estados quânticos
\begin{equation} \label{01}
\ket{0} = \begin{pmatrix}
0 \\
1
\end{pmatrix} \quad \text{e} \quad
\ket{1} = \begin{pmatrix}
1 \\
0
\end{pmatrix}.
\end{equation}

Consideraremos também que estados quânticos que dependem de mais de um qubit (emaranhados ou não), são representados pelo produto tensorial dos seus estados internos, como segue o exemplo:
\begin{equation}\label{00}
\ket{00} = \begin{pmatrix}
0 \\
1
\end{pmatrix} \otimes \begin{pmatrix}
0 \\
1
\end{pmatrix} = \begin{pmatrix}
1 \\
0 \\
0 \\
0
\end{pmatrix}
\end{equation}

A representação gráfica do circuito quântico que realiza o protocolo de emaranhamento possibilita o entendimento dos procedimentos contidos no mesmo, conforme a Figura~\ref{circuitoemr}

\begin{figure}[ht!]
  \centering
  \caption{INSERIR FIGURA CIRCUITO EMARANHAMENTO!}\label{circuitoemr}
\end{figure}

Para que o emaranhamento seja possível, ambos os qubits representados por $\ket{q_x}$ e $\ket{q_y}$ devem existir no mesmo espaço de Hilbert e portanto a operação de produto tensorial entre eles deve ser possível de modo que, considerando o estado quântico $\ket{0}$:
\begin{equation}
\ket{q_x} \otimes \ket{q_y} = \ket{0} \otimes \ket{0} = \ket{00} = \begin{pmatrix}
0 \\
1
\end{pmatrix} \otimes \begin{pmatrix}
0 \\
1
\end{pmatrix} = \begin{pmatrix}
1 \\
0 \\
0 \\
0
\end{pmatrix}
\end{equation}

Seguindo a figura, aplicamos a porta Hadamard, \(\HAD\), multiplicando-a pela matriz identidade \(I\), obtendo
\begin{equation}
\HAD \otimes I = \frac{1}{\sqrt{2}} \begin{pmatrix*}[r]
1 & 1 \\
1 & -1
\end{pmatrix*} \otimes \begin{pmatrix}
1 & 0 \\
0 & 1
\end{pmatrix} = \frac{1}{\sqrt{2}} \begin{pmatrix*}[r]
1 & 0 & 1 & 0 \\
0 & 1 & 0 & 1 \\
1 & 0 & -1 & 0 \\
0 & 1 & 0 & -1
\end{pmatrix*}
\end{equation}
e, aplicando o resultado acima em~\eqref{00}
\begin{equation}
\HAD \ket{00} = \frac{1}{\sqrt{2}} \begin{pmatrix*}[r]
1 & 0 & 1 & 0 \\
0 & 1 & 0 & 1 \\
1 & 0 & -1 & 0 \\
0 & 1 & 0 & -1
\end{pmatrix*} \begin{pmatrix}
1 \\
0 \\
0 \\
0
\end{pmatrix} = \frac{1}{\sqrt{2}} \begin{pmatrix}
1 \\
0 \\
1 \\
0
\end{pmatrix}.
\end{equation}

Em seguida, aplicamos a porta \(\CNOT\)
\begin{equation}
  \CNOT \bigl( \HAD \ket{00} \bigr) =
\begin{pmatrix}
1 & 0 & 0 & 0 \\
0 & 1 & 0 & 0 \\
0 & 0 & 0 & 1 \\
0 & 1 & 1 & 0
\end{pmatrix}
\frac{1}{\sqrt{2}}\begin{pmatrix}
1 \\
0 \\
1 \\
0
\end{pmatrix} = \frac{1}{\sqrt{2}} \begin{pmatrix}
1 \\
0 \\
0 \\
1
\end{pmatrix}
\end{equation}

A equação também é conhecida por ser uma das Bases de Bell. A comparação da Base de Bell no estado $\ket{00}$ em notação de Dirac e em notação matricial é dada por:

Seja \(\ket{\beta_{00}}\) a Base de Bell em notação de Dirac dada por \(\frac{1}{\sqrt{2}} \bigl(\ket{00} + \ket{11}\bigr)\), reescrevendo teremos:

% FALTA TERMINAAAAAAAR!
%
