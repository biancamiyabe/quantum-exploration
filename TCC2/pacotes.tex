%%%
%%%  ____                 _
%%% |  _ \ __ _  ___ ___ | |_ ___  ___
%%% | |_) / _` |/ __/ _ \| __/ _ \/ __|
%%% |  __/ (_| | (_| (_) | ||  __/\__ \
%%% |_|   \__,_|\___\___/ \__\___||___/
%%%
%%% TCC de Bianca Miyabe Santos Freitas
%%% Licenciatura em Física - UFSCar, Sorocaba
%%%
%%% TCC: Documento modelo em LaTeX, padrão ABNT usando ABNTeX2
%%%

%%% IDIOMAS: PORTUGUÊS E INGLÊS
%%% ----------------------------------------------------------------------------
%%% Certifique-se de ter os pacotes corretos (GNU/Linux)!
%%%
%%% O TeXLive (recomendo no caso do Windows) em geral já tem esses pacotes instalados
%%%
\usepackage{indentfirst}
\usepackage{polyglossia}
\setmainlanguage{portuges}
\setotherlanguage{english}
\usepackage{csquotes}

%%% Pacotes padrão da American Mathematical Society
\usepackage{amsmath}
\usepackage{amssymb}
\usepackage{mathtools}
\usepackage{fontawesome5}

\setcounter{MaxMatrixCols}{20}

%%% FONTES
%%% ----------------------------------------------------------------------------
%%% Essas fontes são *opcionais*. Se você comentar as linhas com \setmainfont,
%%% \setsansfont e \setmonofont, o LaTeX usará as fontes padrão.
%%%
%%% São usadas as fontes:
%%% - Crimsom Pro: texto principal
%%% - Noto Sans: sem serifa, usada nas seções, subseções, etc
%%% - DejaVu Sans Mono: monoespaçada para listagems, URLs, etc
%%%
\usepackage{fontspec}
% CrimsonText é similar à Minion Pro, que é comercial
\setmainfont{CrimsonPro}[
  Path           = ./fonts/,
  Extension      = .ttf,
  UprightFont    = *-Regular,
  BoldFont       = *-Bold,
  ItalicFont     = *-Italic,
  BoldItalicFont = *-BoldItalic,
  SmallCapsFont  = AlegreyaSC-Regular
]
% NotoSans é similar à Myriad Pro, que é comercial
\setsansfont{NotoSans}[
  Path           = ./fonts/,
  Extension      = .ttf,
  UprightFont    = *-Regular,
  BoldFont       = *-Bold,
  ItalicFont     = *-Italic,
  BoldItalicFont = *-BoldItalic
]
\setmonofont[Scale=0.75]{DejaVuSansMono}[
  Path           = ./fonts/,
  Extension      = .ttf,
  UprightFont    = *,
  BoldFont       = *-Bold,
  ItalicFont     = *-Oblique,
  BoldItalicFont = *-BoldOblique
]

\usepackage[euler-hat-accent,small]{eulervm}

%%%
%%% Outros pacotes úteis
%%%
\usepackage{graphicx}         % Figuras em vários formatos (png, pdf)
\usepackage{color}            % Cores
\usepackage{multicol}         % Documentos com várias colunas
\usepackage{booktabs}         % Tabelas profissionais
\usepackage[final,nopatch=toc]{microtype} % Pequenos ajustes tipográficos
\usepackage{siunitx}          % Pacote para unidades físicas (recomendo muito!)
\sisetup{output-decimal-marker={,}}
\sisetup{mode=text}
\sisetup{detect-all=true}
\usepackage{placeins}
\usepackage[xindy,abbreviations]{glossaries-extra}
\makeglossaries
\usepackage{blochsphere}
\usepackage{braket}
\usepackage{makecell}
\usepackage{enumitem}
\usepackage{multirow}
\usepackage{xcolor}
\usepackage{tikz}
\usetikzlibrary{angles,matrix,arrows.meta,calc,positioning,intersections,shadows,plotmarks,3d,quotes,quantikz}

%%%
%%% CORES MANEIRÍSSIMAS
%%%
\definecolor{green}{RGB}{174,226,57}    % colourlovers.com/palette/46688/fresh_cut_day (atomic bikini)
\definecolor{yellow}{RGB}{237,229,116}  % colourlovers.com/palette/937624/Dance_To_Forget (Give Your Heart)
\definecolor{orange}{RGB}{255,164,70}   % colourlovers.com/palette/1107950/Indecent_Proposal (exotic orange)
\definecolor{onlyorange}{RGB}{191,77,40}% colourlovers.com/palette/953498/Headache (Only Orange)
\definecolor{cyan}{RGB}{108,243,213}    % colourlovers.com/palette/940927/Acused (wrong cyan)
\definecolor{red}{RGB}{199,8,8}         % colourlovers.com/palette/79468/LipstickOnHisCollar (Flagged Down)
\definecolor{melon}{RGB}{209,49,92}     % colourlovers.com/palette/2350697/This_is_for_YOU! (melon)
\definecolor{blue}{RGB}{62,122,162}     % colourlovers.com/palette/794774/be_here_for_me (kreuger)
\definecolor{berry}{RGB}{95,13,59}      % colourlovers.com/palette/117122/BurberryTenderTouch (BurberryTender)
\definecolor{violet}{RGB}{87,30,240}    % From the original template (Violet Thanos)
\definecolor{hymnroyale}{RGB}{42,4,72}  % colourlovers.com/palette/81885/Hymn_For_My_Soul (Hymn Royale)
\definecolor{think}{HTML}{607848}       % colourlovers.com/palette/38562/Hands_On
\definecolor{gray}{HTML}{444444}
\definecolor{intelligentsia}{RGB}{7,69,111} % colourlovers.com/palette/4792400/Intelligentsia (Intelligentsia circle)

\usepackage{pdfpages}
\usepackage{tcolorbox}
\tcbuselibrary{many}
\tcbuselibrary{theorems}
\tcbuselibrary{minted}
\usepackage[backgroundcolor=orange!45]{todonotes}
% \usepackage{minted}
\usemintedstyle{colorful}
\newcommand{\py}[1]{\mintinline[breaklines]{python3}{#1}}
\renewcommand{\listingscaption}{Listagem}
\newcommand{\norma}{0,5+0,4i}
\newcommand{\normb}{0,3+0,6i}
%%%
%%% TIKZ
%%%
\tikzset{%
  eixos/.style={draw=black!80,text=black!80,arrows=-{Latex[width=4pt,length=6pt]}},
  eixos sem flecha/.style={draw=black!80,text=black!80},
  eixos fantasma/.style={draw=black!20,text=black!60,arrows=-{Latex[width=4pt,length=6pt]}},
  vetor/.style={draw=blue!80,text=blue!80,cap=round,arrows=-{Triangle[width=5pt,length=7pt]},very thick},
  linhaforte/.style={draw=#1,ultra thick,cap=round},
  linhamedia/.style={draw=#1,thick,cap=round},
  ponto/.style={fill=#1!40,draw=#1,semithick,inner sep=2pt,circle},
  pontinho/.style={fill=#1,draw=#1,semithick,inner sep=1pt,circle},
  projecao/.style={draw=#1,densely dotted,thick},
  projecao 2/.style={draw=#1,densely dash dot,thin},
  etiqueta/.style n args={3}{text=#1,draw=#2,fill=#3,solid,font=\scriptsize,inner sep=2pt,minimum height=13pt,drop shadow={opacity=0.8,shadow xshift=.3ex,shadow yshift=-.3ex}},
  face/.style={draw=black,fill=white,thick,cap=round},
  blocoq/.style={very thick,draw=black!70,fill=black!5,inner sep=8pt,align=center,drop shadow={opacity=0.8,shadow xshift=.3ex,shadow yshift=-.3ex}},
  blocor/.style={very thick,draw=red!20,fill=red!5,inner sep=8pt,align=center,rounded corners,drop shadow={opacity=0.8,shadow xshift=.3ex,shadow yshift=-.3ex}},
  blococ/.style={very thick,draw=green!20,ellipse,fill=green!5,inner sep=8pt,align=center,rounded corners,drop shadow={opacity=0.8,shadow xshift=.3ex,shadow yshift=-.3ex}},
  conecta/.style={draw=black!50,text=black!80,cap=round,arrows=-{Triangle[width=5pt,length=7pt]},thick},
  labelst/.style={label distance=-6pt,font=\scriptsize\scshape,align=center,black!80,inner sep=2pt}
}

%%%
%%% Color boxes
%%%
\newtcolorbox{destaque}{
  breakable,
  notitle,
  boxrule=0pt,
  colback=yellow,
  colframe=yellow
}

\newtcblisting{pycode}{
  listing engine=minted,
  minted language=python3,
  minted options={fontsize=\normalsize,breaklines},
  colback=white,
  %colback=blue!3!white,
  enhanced,
  breakable,
  listing only,
  colframe=black!10,
  arc=0mm,
  top=8pt,
  bottom=8pt,
  left=10pt,
  right=10pt,
  boxrule=1pt,
}

\newtcblisting{pycodewhite}{
  listing engine=minted,
  minted language=python3,
  minted options={fontsize=\normalsize,breaklines},
  colback=white,
  enhanced,
  breakable,
  listing only,
  colframe=black!30,
  arc=0mm,
  top=8pt,
  bottom=8pt,
  left=10pt,
  right=10pt,
  boxrule=0pt,
}

\newtcbtheorem[number within=chapter]{post}{Postulado}{
  enhanced,
  %skin=bicolor,
  arc=0mm,
  colback=black!3,
  colframe=black!50,
  colbacktitle=white,
  colbacklower=white,
  coltitle=black,
  fonttitle=\small,
  toptitle=3pt,
  bottomtitle=3pt,
  top=5pt,
  bottom=5pt,
  left=10pt,
  right=10pt,
  boxrule=0pt,
  titlerule=0pt}{post}

\newtcbtheorem[number within=chapter]{definition}{Definição}{
  enhanced,
  %skin=bicolor,
  arc=0mm,
  colback=black!3,
  colframe=black!50,
  colbacktitle=white,
  colbacklower=white,
  coltitle=black,
  fonttitle=\small,
  toptitle=3pt,
  bottomtitle=3pt,
  top=5pt,
  bottom=5pt,
  left=10pt,
  right=10pt,
  boxrule=0pt,
  titlerule=0pt}{definition}

\newtcbtheorem[number within=chapter]{theo}{Teorema}{
  enhanced,
  %skin=bicolor,
  arc=0mm,
  colback=black!3,
  colframe=black!50,
  colbacktitle=white,
  colbacklower=white,
  coltitle=black,
  fonttitle=\small,
  toptitle=3pt,
  bottomtitle=3pt,
  top=5pt,
  bottom=5pt,
  left=10pt,
  right=10pt,
  boxrule=0pt,
  titlerule=0pt}{theo}

%%% Função seno em PT-BR
\DeclareMathOperator{\sen}{sen}
\DeclareMathOperator{\CNOT}{\textsc{cnot}}
\DeclareMathOperator{\HAD}{\textsc{h}}
\DeclareMathOperator{\XXX}{\textsc{x}}
\DeclareMathOperator{\ZZZ}{\textsc{z}}
\DeclareMathOperator{\III}{\textsc{i}}

%%% end of pacotes.tex
