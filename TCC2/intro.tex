%%%  ___       _
%%% |_ _|_ __ | |_ _ __ ___
%%%  | || '_ \| __| '__/ _ \
%%%  | || | | | |_| | | (_) |
%%% |___|_| |_|\__|_|  \___/
%%%
%%% TCC de Bianca Miyabe Santos Freitas
%%% Licenciatura em Física - UFSCar, Sorocaba
%%%
\chapter{Introdução}

Nas últimas décadas a humanidade passou por um intenso e revolucionário processo de inovações e renovações tecnológicas envolvendo o dispositivo que conhecemos por computador. Basta recordar que o tamanho de um smartphone moderno é muito menor do que a primeira unidade de computador eletrônico criado, o ENIAC, que ocupava um espaço de \SI{180}{\square\meter} \cite{eniac}.

Nesse processo evolutivo do computador podemos destacar que a miniaturização dos processadores resultou no aumento da sua capacidade de processamento de informação e estes foram essenciais para a popularização dos dispositivos e ainda, para o aumento da sua velocidade operacional. Diante dessa constante mudança, em 1965 foi estabelecido por Gordon E. Moore (1929) um limite de processamento devido ao número de transistores\footnote{O transistor é um componente eletrônico desenvolvido por John Bardeen, William Shockley e Walter Brattain em meados de 1947. O dispositivo passou por diversos aperfeiçoamentos desde então e sua principal função consiste em amplificar ou interromper sinais elétricos. Nos computadores, são os responsáveis por indicar a presença ou ausência do sinal elétrico, sendo possível a interpretação da informação nos dispositivos de processamento \cite{transistor}.} necessários comprimidos em um pequeno espaço versus sua dissipação de calor, o que corrompe a informação. Esse limite recebeu o nome de ``Lei de Moore''. Nela, \textcite{moore} estimou que o número de transistores de um computador dobraria a cada dois anos sem que seu valor fosse alterado. Esse limite foi brevemente superado por novas tecnologias de materiais\footnote{A empresa IBM, produziu em 2014 um nanochip de silício de \SI{7}{\nano\meter} e em 2015 anunciou a produção de chips de processamento com nanotubos de carbono de tamanho \SI{1.8}{\nano\meter} \cite{chipibm}.}, deixando evidente, entretanto, a necessidade de expandir a capacidade de processamento dos sistemas atuais, visto que a tendência de crescimento na quantidade de informação processada é cada vez maior.

Diante do limite físico para o tamanho dos processadores e do crescimento do volume de informação a ser processado, uma possível solução foi proposta pelo físico Richard Feynman em 1981. Feynman, na tentativa de compreender a simulação de sistemas físicos para seus estudos, propõe que se sistemas físicos são regidos pela física quântica, sua simulação deve ser feita por um dispositivo que corresponda a mesma natureza \cite{caldeira}.

Nesse período temos portanto a junção de três importantes áreas de estudo: a computação, a informação e a física quântica. Esta união visava superar os limites da computação até então, em relação à velocidade de processamento e volume de armazenamento de informação, dando início aos estudos da chamada \textit{Computação Quântica}. A computação quântica é um campo emergente cujo objetivo é desenvolver computação com base nos princípios da Mecânica Quântica (MQ) que, conforme veremos na Seção~\ref{sec:Mecanicaquantica}, é a teoria da Física que descreve o comportamento dos átomos, íons e partículas subatômicas. As partículas quânticas podem existir em múltiplos estados ao mesmo tempo e essa característica única permite que os computadores quânticos manipulem simultaneamente muitos estados de dados, o que não é possível com computadores convencionais, permitindo aos computadores quânticos processar muito mais informação, de forma mais rápida e eficiente do que os computadores convencionais, que utilizam a arquitetura de dados clássica, ou seja, informação clássica \cite{CompInfoQuantica}.

Em um computador clássico, por exemplo, armazenamos informações através das unidades binárias chamadas \textit{bits}\footnote{Nome proposto, segundo o artigo original de Shannon por J.W. Turkey \cite{MTC}.}. Dessa forma, os bits são a menor unidade de armazenamento de informação em um computador de arquitetura clássica, podendo representar o estado 1 ou o estado 0 \cite{MTC}.

A combinação desses bits faz com que uma mensagem possa ser armazenada, processada ou transmitida em um computador clássico. Nesse sentido, quão maior, ou ainda, quão mais complexa for a mensagem a se operar, mais bits serão necessários e consequentemente mais recursos físicos para a representação destes.

A descrição da arquitetura de um computador quântico esbarra no mesmo princípio daquela de um computador clássico, ou seja, em sua unidade fundamental de armazenamento de informação. De maneira análoga ao computador clássico, que utiliza como unidade de informação o bit, o computador quântico utilizará o \textit{qubit} (ou q-bit, ou ainda, quantum bit).

Um qubit, ou bit quântico, pode ser produzido de maneiras distintas\footnote{Qubits podem ser fisicamente criados utilizando, por exemplo, spins de átomos presos em uma armadilha. Essa armadilha pode ser do tipo óptica ou até mesmo magnética. É possível também polarizar fótons para sua obtenção. A determinação do método é definida principalmente pelo mecanismo que melhor conseguir isolar o qubit, já que este é facilmente influenciado pelo ambiente externo \cite{materialdidaticomecquantica}.}, porém nosso foco de estudo está nas suas propriedades. Um qubit é uma unidade com propriedades quânticas que atua sob o regime de superposição de estados. Isso significa que ele consegue armazenar simultaneamente mais de um estado de informação, diferente do bit clássico que armazena apenas um dos estados por vez. Decorre desta propriedade a maior capacidade de operar a informação em comparação aos mecanismos clássicos segundo apresentado na Tabela~\ref{tabelabit}.

\begin{table}[ht]
  \centering
  \caption{Comparação entre a quantidade de bits clássicos e quânticos necessários para se operar uma informação.}\label{tabelabit}
  \begin{tabular}{ccc}
    \toprule
    \thead{Quantidade \\ de bytes \\ (informação)} & \thead{Quantidade \\ de bits clássicos} & \thead{Quantidade \\ de qubits} \\
    \midrule
    1         & 8            & 3  \\
    \num{e6}  & \num{8.3e6}  & 23 \\
    \num{e12} & \num{8.8e12} & 43 \\
    \bottomrule
  \end{tabular}
  \fonte{Elaborada pelo autor.}
\end{table}

De modo a generalizar a comparação entre bits classicos e quânticos, podemos estabelecer a relação:
\begin{equation} \label{bitvsqubit}
n\, \text{qubits} = 2^{n}\,\text{bits}.
\end{equation}

Portanto, podemos concluir que menos qubits são necessários para operar a informação, em comparação ao bit clássico, o que está diretamente relacionado com a velocidade e com a capacidade de realização deste.

Segundo \textcite{CompInfoQuantica} e \textcite{dwave}, a devida construção de um computador de arquitetura quântica foi precedida pelos eventos descritos a seguir:

\begin{description}
  \item[1985] David Deustch propõe matematicamente o primeiro computador quântico universal;
  \item[1994] Peter Shor cria o primeiro programa essencialmente quântico, ou seja, ele não poderia ser executado em um computador clássico. Este programa, conhecido como Algoritmo de Shor, reduziria o tempo de fatoração de números grandes de possíveis meses para apenas segundos caso fosse utilizado em um computador real de arquitetura quântica;
  \item[1999] O MIT apresenta o primeiro protótipo de um computador quântico real;
  \item[2007] A empresa D-Wave apresenta o primeiro processador essencialmente quântico.
\end{description}

Apesar de na atualidade processadores quânticos existirem e operarem\footnote{Atualmente a IBM possuí um processador que opera com 433 qubits simultâneos, o \textit{Osprey}, anunciado em novembro de 2022 \cite{osprey}.}, ainda estamos distantes da efetiva implementação comercial de um computador quântico. Podemos utilizar de exemplo, o fato de que apesar de possuirmos um análogo para a TMC de Shannon em um computador quântico\footnote{Em 1995, Benjamin Schumacher propõe com êxito um análogo quântico para o TMC \cite{benschu}.}, ainda não temos um análogo quântico para um sistema submetido a ruídos na transmissão\footnote{Contudo, foi desenvolvida a teoria de correção de erros quânticos que permite que computadores quânticos possam operar na presença de ruídos e que a informação quântica seja transmitida de maneira confiável \cite{chuang}}\cite{chuang}.

Portanto, o estudo de simulações de sistemas de informação quânticos se faz necessário para aperfeiçoamento desses mecanismos e ainda para o desenvolvimento da própria Física, visto que o avanço da compreensão da utilização da MQ atrelado ao conceito de informação, possibilita a compreensão da natureza de maneira cada vez mais complexa, sem a necessidade de aproximações e simplificações\footnote{A busca por materiais supercondutores mais eficientes, ou seja, que demandem menor controle de temperatura e pressão pode revolucionar a construção de novas tecnologias de materiais. Simular essas combinações classicamente é quase impossível pois, alem de existir uma infinidade de possibilidades dentro da tabela periódica, temos ainda uma infinidade de condições de cada um dos materiais envolvidos, como temperatura, proporção, pressão etc. Portanto, utilizar uma simulação em um computador de arquitetura quântica, além de mais rápido é também mais próximo da realidade, visto a quantidade de informação possível de se computar com um número pequeno de entidades de memória (qubits)\cite{videoyoutube}}\cite{videoyoutube}.

Os estudos de simulações quânticas é uma necessidade emergente, com a finalidade de desenvolver novas tecnologias que esbarrem na descrição da realidade de maneira mais precisa. A proposição deste trabalho tem como finalidade, introduzir os estudos de aplicações da MQ para a simulação de situações que dependam de sistemas quânticos. Nesse trabalho, propomos a elaboração de uma simulação de um protocolo de teletransporte quântico para verificar suas operações e ainda uma discussão sobre eventuais erros que o protocolo pode apresentar em uma situação real. Nesse sentido, os próximos capítulos irão introduzir conceitos necessários para a compreensão do trabalho desenvolvido.

\clearpage
