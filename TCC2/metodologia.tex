%%%  __  __      _            _
%%% |  \/  | ___| |_ ___   __| | ___
%%% | |\/| |/ _ \ __/ _ \ / _` |/ _ \
%%% | |  | |  __/ || (_) | (_| | (_) |
%%% |_|  |_|\___|\__\___/ \__,_|\___/
%%%
%%% TCC de Bianca Miyabe Santos Freitas
%%% Licenciatura em Física - UFSCar, Sorocaba
%%%

\chapter{Metodologia}
Esse é um código em Python que implementa o protocolo de teleportação quântica. O objetivo é enviar um qubit em um estado desconhecido de um local para outro sem que ele seja fisicamente transferido. O protocolo consiste em três qubits: um qubit que contém a informação a ser transmitida (qubit mensagem), um qubit no estado emaranhado (qubit emaranhado) e um qubit enviado pelo remetente (qubit do remetente) que é usado para transferir a informação do qubit mensagem para o qubit emaranhado.

As bibliotecas importadas no início do código são: numpy, math, sympy, sys e random. A biblioteca numpy é usada para trabalhar com matrizes e operações numéricas. A biblioteca math é usada para funções matemáticas. A biblioteca sympy é usada para simbolizar variáveis e operações simbólicas. A biblioteca sys é usada para acessar algumas variáveis ​​do sistema e a biblioteca random é usada para gerar números aleatórios.

Os dois primeiros qubits, qbit0 e qbit1, são inicializados usando a biblioteca numpy. A matriz H, que representa a porta Hadamard, e a matriz I, que representa a matriz de identidade, são inicializadas usando a biblioteca numpy. A matriz HI é a porta Hadamard tensorizada com a matriz de identidade e é usada para criar os estados de Bell mais tarde no código.

Em seguida, os estados de Bell são criados por meio da aplicação da porta Hadamard no estado tensorizado de dois qubits. Esses estados são criados a partir da operação tensorial entre o qubit emaranhado e o qubit do remetente. Esses estados são salvos nas variáveis Bell00, Bell01, Bell10 e Bell11.

Depois disso, o usuário é solicitado a fornecer o estado alfa e beta do qubit mensagem que ele deseja enviar. O código verifica se a soma de alfa e beta é igual a um (condição de normalização) e, em caso afirmativo, cria a mensagem a ser enviada a partir desses estados.

O estado psi0 é calculado por meio da operação tensorial entre o qubit mensagem e o estado de Bell. Em seguida, a porta CNOT é aplicada ao estado psi0 e o resultado é salvo na variável psi1.

A porta Hadamard é aplicada aos estados alfa e beta do qubit mensagem e o resultado é salvo nas variáveis HMessagealfa e HMessagebeta. Essas variáveis são usadas posteriormente para a reconstrução da mensagem.

O código então gera um dos quatro possíveis estados emaranhados (estados00, estados11, estados01 e estados10) aleatoriamente. O estado resultante é salvo na variável Medição e é usado para reconstruir a mensagem original.

O protocolo de teleportação quântica é concluído pela reconstrução da mensagem original a partir do estado Medição. Isso é feito usando as matrizes de Pauli para recuperar a mensagem original a partir da variável Messagetp e, em seguida, salvar o resultado na variável Messagefinal. Por fim, a mensagem original é impressa na tela.
