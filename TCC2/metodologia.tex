%%%  __  __      _            _
%%% |  \/  | ___| |_ ___   __| | ___
%%% | |\/| |/ _ \ __/ _ \ / _` |/ _ \
%%% | |  | |  __/ || (_) | (_| | (_) |
%%% |_|  |_|\___|\__\___/ \__,_|\___/
%%%
%%% TCC de Bianca Miyabe Santos Freitas
%%% Licenciatura em Física - UFSCar, Sorocaba
%%%

\chapter{Metodologia}
Para iniciar o desenvolvimento de um protocolo de teletransporte quântico, inicialmente realizou-se a dedução do mesmo em forma matricial, conforme apresentado detalhadamente no Apêndice~\ref{app:matricial}

Realizando um exemplo com os valores atribuídos de $\alpha=0,75$ e $\beta=0,25$, teremos que a mensagem a ser enviada é descrita por:
\begin{equation}
\ket{\psi} = 0,75 \begin{pmatrix}
1 \\
0
\end{pmatrix} + 0,25 \begin{pmatrix}
0 \\
1
\end{pmatrix} = \begin{pmatrix}
0,75 \\
0
\end{pmatrix} + \begin{pmatrix}
0 \\
0,25
\end{pmatrix} = \begin{pmatrix}
0,75 \\
0,25
\end{pmatrix}
\end{equation} 

O estado $\ket{\psi_0}$ descrito pela equação~\eqref{eq:psi0} é descrito por:
\begin{equation}
\ket{\psi_0}=\ket{\psi}\ket{\beta_{00}}= \begin{pmatrix}
0,75 \\
0,25  
\end{pmatrix} \left[\frac{1}{\sqrt{2}} \begin{pmatrix}
1 \\
0 \\
0 \\
1
\end{pmatrix}\right]
\end{equation}

Rearranjando os termos segundo~\eqref{eq:psi0cnot}
\begin{equation}
\ket{\psi_0} = \frac{1}{\sqrt{2}}\left\{\left[ \begin{pmatrix}
0,75 \\
0 
\end{pmatrix}  \begin{pmatrix}
1 \\
0 \\
0 \\
1
\end{pmatrix}\right] + \left[ \begin{pmatrix}
0 \\
0,25
\end{pmatrix}  \begin{pmatrix}
1 \\
0 \\
0 \\
1
\end{pmatrix}\right] \right\}
\end{equation}

Explicitando os estados possíveis conforme~\eqref{eq:3qubit}
\begin{equation}
	\begin{split}
\ket{\alpha00} &= \left[ \begin{pmatrix}
0,75 \\
0
\end{pmatrix} \otimes \begin{pmatrix}
1 \\
0
\end{pmatrix} \otimes \begin{pmatrix}
1 \\
0
\end{pmatrix}\right] = \begin{pmatrix}
0,75 & 0 & 0 & 0 & 0 & 0 & 0 & 0
\end{pmatrix}^T \\
\ket{\alpha11} &= \left[ \begin{pmatrix}
0,75 \\
0
\end{pmatrix} \otimes \begin{pmatrix}
0 \\
1
\end{pmatrix} \otimes \begin{pmatrix}
0 \\
1
\end{pmatrix}\right] = \begin{pmatrix}
0 & 0 & 0 & 0,75 & 0 & 0 & 0 & 0
\end{pmatrix}^T \\
\ket{\beta00} &= \left[ \begin{pmatrix}
0 \\
0,25
\end{pmatrix} \otimes \begin{pmatrix}
1 \\
0
\end{pmatrix} \otimes \begin{pmatrix}
1 \\
0
\end{pmatrix}\right] = \begin{pmatrix}
0 & 0 & 0 & 0 & 0,25 & 0 & 0 & 0
\end{pmatrix}^T \\
\ket{\beta11} &=\left[ \begin{pmatrix}
0 \\
0,25
\end{pmatrix} \otimes \begin{pmatrix}
0 \\
1
\end{pmatrix} \otimes \begin{pmatrix}
0 \\
1
\end{pmatrix}\right] = \begin{pmatrix}
0 & 0 & 0 & 0 & 0 & 0 & 0 & 0,25
\end{pmatrix}^T
	\end{split}
\end{equation}
Somando todos os estados possíveis temos:
\begin{equation}
\ket{\alpha00}+\ket{\alpha01}+\ket{\beta00}+\ket{\beta11}= \begin{pmatrix}
0,75 & 0 & 0 & 0,75 & 0,25 & 0 & 0 & 0,25
\end{pmatrix}^T
\end{equation}

Aplicando a porta \(\CNOT\)
\begin{equation}
\begin{split}
		\CNOT (\ket{\alpha00}\ket{\alpha01}\ket{\beta00}\ket{\beta11}) &= \begin{pmatrix}
		1 & 0 & 0 & 0 & 0 & 0 & 0 & 0 \\
		0 & 1 & 0 & 0 & 0 & 0 & 0 & 0 \\
		0 & 0 & 1 & 0 & 0 & 0 & 0 & 0 \\
		0 & 0 & 0 & 1 & 0 & 0 & 0 & 0 \\
		0 & 0 & 0 & 0 & 0 & 0 & 1 & 0 \\
		0 & 0 & 0 & 0 & 0 & 0 & 0 & 1 \\
		0 & 0 & 0 & 0 & 1 & 0 & 0 & 0 \\
		0 & 0 & 0 & 0 & 0 & 1 & 0 & 0 		
		\end{pmatrix} \begin{pmatrix}
		0,75 \\
		0 \\
		0 \\
		0,75 \\
		0,25 \\
		0 \\
		0 \\
		0,25
		\end{pmatrix} = \\
		&=\begin{pmatrix}
		0,75 & 0 &	0 & 0,75 &	0 &	0,25 &	0,25 &	0
		\end{pmatrix}^T
	\end{split}
\end{equation}	
	
Reorganizando o estado $\ket{\psi_1}$ temos, segundo~\eqref{eq:psi1}
\begin{equation}
\ket{\psi_1} = \frac{1}{\sqrt{2}} \left\{ \begin{pmatrix}
0,75 \\
0
\end{pmatrix} \otimes \left[ \begin{pmatrix}
1 \\
0 \\
0 \\
0
\end{pmatrix} + \begin{pmatrix}
0 \\
0 \\
0 \\
1
\end{pmatrix}\right] +  \begin{pmatrix}
0 \\
0,25
\end{pmatrix} \otimes \left[ \begin{pmatrix}
0 \\
0 \\
1 \\
0
\end{pmatrix} + \begin{pmatrix}
0 \\
1 \\
0 \\
0
\end{pmatrix}\right] \right\}
\end{equation}	

Aplicando a porta \(\HAD\)
\begin{equation}
\begin{split}
		\HAD \ket{\psi} &= \frac{1}{\sqrt{2}} \begin{pmatrix*}[r]
		1 & 1 \\
		1 & -1
		\end{pmatrix*} \frac{1}{\sqrt{2}}\left( \begin{pmatrix}
		0,75 \\
		0 
		\end{pmatrix} +  \begin{pmatrix}
		0 \\
		0,25
		\end{pmatrix} \right) = \frac{1}{2} \left[ \begin{pmatrix}
		,75 \\
		0,75 
		\end{pmatrix} + \beta \begin{pmatrix*}[r]
		0,25 \\
		-0,25
		\end{pmatrix*} \right]
  	\end{split}
\end{equation}	

Reorganizando os resultados para definir o estado $\ket{\psi_2}$ temos

\begin{equation}
 \begin{split}
\ket{\psi_2} &=\frac{1}{2} \left\{ \left[\begin{pmatrix}
1 \\
0
\end{pmatrix} \otimes \begin{pmatrix}
1 \\
0
\end{pmatrix}\right] \left[ \begin{pmatrix}
0,75 \\
0
\end{pmatrix} +  \begin{pmatrix}
0 \\
0,25
\end{pmatrix}\right]\right\} \nonumber \\
&+\frac{1}{2} \left\{ \left[\begin{pmatrix}
0 \\
1
\end{pmatrix} \otimes \begin{pmatrix}
0 \\
1
\end{pmatrix}\right] \left[ \begin{pmatrix}
0 \\
0,75
\end{pmatrix} -  \begin{pmatrix}
0,25 \\
0
\end{pmatrix}\right]\right\} \nonumber \\
&+\frac{1}{2} \left\{\left[ \begin{pmatrix}
1 \\
0
\end{pmatrix} \otimes \begin{pmatrix}
0 \\
1
\end{pmatrix}\right] \left[ \begin{pmatrix}
0 \\
0,75
\end{pmatrix} +  \begin{pmatrix}
0,25 \\
0
\end{pmatrix}\right]\right\} \nonumber \\
&+\frac{1}{2} \left\{ \left[ \begin{pmatrix}
0 \\
1
\end{pmatrix} \otimes \begin{pmatrix}
1 \\
0
\end{pmatrix}\right] \left[ \begin{pmatrix}
0,75 \\
0
\end{pmatrix} -  \begin{pmatrix}
0 \\
0,25
\end{pmatrix}\right] \right\}
  \end{split}
\end{equation}	
	
Para reconstruir o estado, aplicando as condições descritas na tabela~\ref{tab:acao-das-portas} teremos:

Se a medida realizada for o estado $\ket{00}$, o estado enviado foi $\ket{\psi} = \begin{pmatrix}
0,75 \\
0
\end{pmatrix} +  \begin{pmatrix}
0 \\
0,25
\end{pmatrix}$	

Se a medida realizada for o estado $\ket{11}$:

\begin{equation}
\begin{pmatrix}
0 \\
0,75
\end{pmatrix} -  \begin{pmatrix}
0,25 \\
0
\end{pmatrix} = \begin{pmatrix}
-0,25 \\
0,75
\end{pmatrix}
\end{equation}

Aplicando \(\XXX\)

\begin{equation}
\begin{pmatrix}
0 & 1 \\
1 & 0
\end{pmatrix} \begin{pmatrix}
-0,25 \\
0,75
\end{pmatrix} = \begin{pmatrix}
0,75 \\
-0,25
\end{pmatrix}
\end{equation}

Aplicando \(\ZZZ\)

\begin{equation}
\begin{pmatrix}
1 & 0 \\
0 & -1
\end{pmatrix}\begin{pmatrix}
0,75 \\
-0,25
\end{pmatrix} = \begin{pmatrix}
0,75 \\
0,25
\end{pmatrix}
\end{equation}`

Portanto o estado recuperado é:

\begin{equation}
\begin{pmatrix}
0,75 \\
0,25
\end{pmatrix} = \begin{pmatrix}
0,75 \\
0
\end{pmatrix} +  \begin{pmatrix}
0 \\
0,25
\end{pmatrix}
\end{equation}

Se a medida realizada for o estado $\ket{10}$:

\begin{equation}
\begin{pmatrix}
0,75 \\
0
\end{pmatrix} -  \begin{pmatrix}
0 \\
0,25
\end{pmatrix} = \begin{pmatrix}
0,75 \\
-0,25
\end{pmatrix}
\end{equation}

Aplicando \(\ZZZ\)

\begin{equation}
\begin{pmatrix}
1 & 0 \\
0 & -1
\end{pmatrix}\begin{pmatrix}
0,75 \\
-0,25
\end{pmatrix} = \begin{pmatrix}
0,75 \\
0,25
\end{pmatrix}
\end{equation}

Por último, se a medida realizada for o estado $\ket{01}$:

\begin{equation}
\begin{pmatrix}
0,25 \\
0
\end{pmatrix} +  \begin{pmatrix}
0 \\
0,75
\end{pmatrix} = \begin{pmatrix}
0,25 \\
0,75
\end{pmatrix}
\end{equation}

Aplicando \(\XXX\)

\begin{equation}
\begin{pmatrix}
0 & 1 \\
1 & 0
\end{pmatrix} \begin{pmatrix}
0,25 \\
0,75
\end{pmatrix} = \begin{pmatrix}
0,75 \\
0,25
\end{pmatrix}
\end{equation}

Todos os estados recuperados são iguais aos estados enviados e portanto a dedução de~\ref{app:matricial} pode ser utilizada como um guia para um programa que automatize tais operações, conforme veremos a seguir.

Para alcançar o objetivo deste trabalho, ou seja, construir uma simulação de um teletransporte quântico, foi necessário inicialmente decidir a linguagem a ser utilizada para o projeto, que nesse caso, foi o Python. Essa escolha foi feita pois, essa linguagem de programação é amplamente utilizada para simular situações físicas e possuí uma série de bibliotecas que permitem desenvolver os mais diversos projetos. As bibliotecas utilizadas no projeto foram a \textit{sympy, numpy, math, sys e random}, e sua utilização será explicitada durante a explicação das etapas do código fonte. Antes de iniciar a construção do código, foi necessário analisar a estrutura de dados que seria utilizada e a escolhida foi a representação matricial. Essa escolha foi fundamentada no fato de que desse modo, as operações ficam mais explicitas, facilitando a compreensão do protocolo.

De posse da estrutura de dados a ser operacionalizada, da linguagem de programação e dos pacotes necessários, foi desenvolvido o código apresentado no Apêndice~\ref{app:protocolo}. Vamos detalhar as operações e correlacioná-las a seguir.

A primeira etapa da elaboração do protocolo consistiu em importar as bibliotecas necessárias e algumas funções dessas bibliotecas como a \textit{TensorProduct} em \textit{sympy.physics.quantum}. Essa função irá facilitar a realização dos produtos tensoriais durante o desenvolvimento do protocolo.

Com as bibliotecas carregadas, a primeira etapa foi a construção das variáveis que descrevem os estados quânticos $\ket{0}$ e $\ket{1}$, elas foram feitas com o auxílio da biblioteca \textit{numpy} e da função\textit{matrix} que cria matrizes à partir de listas. Portando o estado $\ket{0}$ foi definido pela variável \begin{tiny}\textbf{$qbit0= np.matrix([1,0]).transpose()$}\end{tiny} e o estado $\ket{1}$ pela variável \begin{tiny}\textbf{$qbit1= np.matrix([0,1]).transpose()$}\end{tiny}.
Foram estabelecidas também as variáveis que representam os operadores quânticos, para a porta \(\HAD\) criou-se a variável \textit{$H$}, representada por \begin{tiny}\textbf{$H = 1/sqrt(2)*(np.matrix([[1,1], [1,-1]]))$}\end{tiny}, para a porta \(\CNOT\) temos a variável \begin{tiny}\textbf{$CNOT = np.matrix([[1,0,0,0],[0,1,0,0],[0,0,0,1],[0,0,1,0]])$}\end{tiny}, para a porta \(\III\) a variável \begin{tiny}\textbf{$I = np.matrix ([[1,0], [0,1]])$}\end{tiny}, para as matrizes de Pauli \(\XXX\) e \(\ZZZ\) as variáveis \begin{tiny}\textbf{$X = np.matrix([[0, 1], [1, 0]]) \quad \text e \quad Z = np.matrix([[1, 0], [0, -1]])$}\end{tiny} respectivamente.

Conforme a Figura~\ref{fig:protocoloteletransporte}, a primeira operação realizada é o Emaranhamento de dois qubits, que no nosso caso, estão emaranhados pelo estado $\ket{00}$. Iniciamos portanto, com a aplicação da porta \(\HAD\) no qubit $\ket{q_x}$ que está no estado, para isso, criou-se a variável \textit{$Hqbit0$} como \begin{tiny}\textbf{$H*qbit0$}\end{tiny}. Nesse momento, os resultados obtidos são comparáveis com a equação~\eqref{eq:Hqbit0}. 

Em seguida, a porta \(\CNOT\) deve ser aplicada e a definição dos qubits alvo e controle foi implementada na variável \textit{$AC$} e é descrita por \begin{tiny}\textbf{$AC= TP(Hqbit0,qbit0)$}\end{tiny}, ou seja, o produto tensorial de todos os estados possíveis dentro do sistema. A aplicação de \(\CNOT\) resulta na base de Bell no estado $\ket{00}$ e é obtida na variável \begin{tiny}\textbf{$Bell00 = CNOT * AC$}\end{tiny}. Nesse momento, temos o sistema descrito pela base de Bell da mesma forma que demonstra a equação~\eqref{eq:matrizbell00}. 

No protocolo construído, é solicitado ao usuário que descreva as probabilidades associadas às variáveis $\alpha$ e $\beta$. Para tal, utiliza-se a função \textit{$input$} e declarando a variável como decimal (\textit{$float$}), solicita-se ao usuário que entre com os valores desejados para cada variável. Note que, como a superposição de estados tem natureza probabilística, é necessário implementar uma verificação da condição de normalização, para que os valores inseridos pelo usuário contemplem a probabilidade somada de $100\%$ caso contrário, o protocolo é interrompido.

A determinação do qubit que será enviado é construída pela variável \textit{$estado_inicial$} de modo que a probabilidade de $\alpha$ fique associada ao estado $\ket{0}$ e a probabilidade de $\beta$ fique associada ao estado $\ket{1}$, a variável é descrita portanto por \begin{tiny}\textbf{$estado_inicial = (estado\_alfa * qbit0) + (estado\_beta * qbit1)$}\end{tiny}.

A próxima etapa do protocolo consiste na aplicação da porta CNOT nos três qubits que compõem o estado geral do sistema. Os três qubits são o estado inicial a ser teletransportado atuando como controle e o par emaranhado atuando como alvo. A atuação ocorre apenas no qubit presente no local A, porém afeta probabilisticamente o qubit no local B. Para definir o estado geral são somados os produtos tensoriais de todos os estados dos qubits com sua representação no código descrita como \begin{tiny}\textbf{$psi\_0 = TP(qbit0, qbit0, qbit0) + TP(qbit0, qbit1, qbit1) + TP(qbit1,qbit0, qbit0) + TP(qbit1,qbit1,qbit1)$}\end{tiny}. Em seguida, a porta CNOT é dimensionada para atuar sob três qubits com a realização do produto tensorial entre CNOT e I e por último multiplicamos a porta CNOT pela soma dos estados dos qubits. Portanto, o estado $\ket{\psi_1}$ é operacionalizado pela multiplicação da porta \(\CNOT\) dimensionada no estado $\ket{\psi_0}$ ficando evidenciada em \begin{tiny}\textbf{$psi\_1 = CNOT\_I * psi\_0$}\end{tiny}.

Para proceder com a aplicação da porta Hadamard no qubit de Mensagem, os estados associados a $\alpha$ e $\beta$ foram separados e a porta aplicada em cada um deles individualmente pelas multiplicações \begin{tiny}\textbf{$H\_estado\_alfa = H * (estado\_alfa * qbit0)$}\end{tiny} e \begin{tiny}\textbf{$H\_estado\_beta = H * (estado\_beta * qbit1)$}\end{tiny}. Com esses resultados, foram reestruturados os estados $\alpha\ket{0}, \alpha\ket{1}, \beta\ket{0}, \text e \beta\ket{1}$ conforme é explicitado na equação~\eqref{eq:alfabetasoma}. Para isso, foram construídas quatro variáveis, uma para cada estado nomeadas \textit{$estadoa0, estadoa1, estadob0 \text e \quad estadob1$}, cada uma dessas variáveis recupera a posição do valor das matrizes \textit{$H\_estado\_alfa$} e \textit{$H\_estado\_beta$} para satisfazer a condição dada em~\eqref{eq:alfabetasoma}, ou seja, o primeiro elemento da matriz está associado aos estados $\alpha\ket{0}$ e $\beta\ket{0}$ e o segundo elemento da matriz associado aos estados $\alpha\ket{1}$ e $\beta\ket{1}$. Essa associação é feita pela posição do elemento na matriz, lembrando que a contagem de matrizes em Python inicia em 0 de modo que as variáveis foram definidas como:

\begin{center}
\begin{tiny}estadoa0 = float(H\_estado\_alfa[0][0]), estadoa1 = float(H\_estado\_alfa[1][0]),\linebreak
estadob0 = float(H\_estado\_beta[0][0]) e  estadob1 = float(H\_estado\_beta[1][0])\end{tiny}
\end{center}

Em seguida, recriamos as matrizes que representam os estados de posse dos elementos localizados anteriormente, temos portanto:

\begin{center}
\begin{tiny}a0 = np.matrix([estadoa0,0]).transpose(),a1 = np.matrix([0,estadoa1]).transpose(),\linebreak
b0 = np.matrix([estadob0,0]).transpose() e b1 = np.matrix([0,estadob1]).transpose()\end{tiny}
\end{center}

Nesse momento temos em nosso código dois estados separados para $\ket{\psi_2}$, o estado associado ao qubit mensagem e o estado associado ao par emaranhado. Para recuperar os estados possíveis do par emaranhado, criamos as variáveis \textit{$estado\_00, estado\_11, estado\_10$ e $estado\_01$} usando a função de produto tensorial:

\begin{center}
\begin{tiny}estado\_00 = TP(qbit0, qbit0), estado\_11 = TP(qbit1, qbit1),\linebreak
estado\_10 = TP(qbit1, qbit0) e \quad estado\_01 = TP(qbit0, qbit1)\end{tiny}
\end{center}

Para simular a operação de medição, criamos uma lista de variáveis, armazenada em \textit{$group\_estados$}, determinamos com a função \textit{$random$} a escolha de um desses estados de maneira aleatória, armazenamos a escolha na variável \textit{Medição} e utilizando a função \textit{$del$} apagamos todos os estados, simulando o colapso do sistema após a medida.

Para reconstruir a mensagem, a condição estabelecida foi associar o estado medido armazenado na variável \textit{Medição} com o estado da mensagem baseando-se na equação~\eqref{eq:final}, pela relação apresentada na Tabela~\ref{tab:medidas}. O código para essa implementação é dado por:

\begin{center}
\begin{lstlisting}[language=Python, caption=Relação de condição para o estado teletransportado em função do estado medido em A.]
if  np.all(Medição == (TP(qbit0,qbit0))):
    estado_tp = (a0-b1)
elif  np.all(Medição == (TP(qbit1,qbit1))):
    estado_tp = (a1-b0)
elif  np.all(Medição == (TP(qbit0,qbit1))):
    estado_tp = (a1+b0)
elif np.all(Medição == (TP(qbit1,qbit0))):
    estado_tp = (a0+b1)
\end{lstlisting}
\end{center}

Com a variável \textit{$estado\_tp$}, temos portando o estado que o qubit no local B possui. Para recuperar a mensagem, iremos aplicar as matrizes de Pauli, seguindo as relações propostas na Tabela~\ref{tab:acao-das-portas}. A condição estabelecida foi:

\begin{center}
\begin{lstlisting}[language=Python, caption=Relação de condição para aplicação das portas de Pauli.]
if np.all(estado_tp == (a0-b1)):
    estado_final = estado_tp
elif np.all(estado_tp == (a1-b0)):
    estado_final = Z * X * estado_tp 
elif np.all(estado_tp == (a1+b0)):
    estado_final = X * estado_tp 
elif np.all(estado_tp == (a0+b1)):
    estado_final = Z * estado_tp 
    
estado_teletransportado = estado_final*sqrt(2)    

alfa = float(estado_teletransportado[0][0])
beta = float(estado_teletransportado[1][0])

alfa_final = np.matrix([alfa,0]).transpose()
beta_final = np.matrix([0,beta]).transpose()
\end{lstlisting}
\end{center}

Vale ressaltar que durante as operações, herdamos a probabilidade associada dos estados, que nesse caso é $1/4$ e portanto, retiramos ela para recuperar a mensagem idêntica a informada no inicio com a expressão Nesse momento, a variável \textit{$estado\_teletransportado$} armazena a mensagem reconstruída no local B. Para verificarmos se o teletransporte teve sucesso, separamos as probabilidades associadas a $\alpha$ e $\beta$ para comparamos com o estado original.

Para simular os ruídos, implementamos as variáveis \textit{$bitflip$}	e \textit{$phaseflip$} equivalentes com as seguintes matrizes\begin{tiny}\textbf{$bitflip = np.matrix([[0, 1], [1, 0]]), phaseflip = np.matrix([[1, 0], [0, -1]])$}\end{tiny}. O usuário pode escolher qual ruído deseja implementar, ou se não deseja implementar ruídos.
Na aplicação do ruído, as variáveis $alfa\_final$ e $beta\_final$ serão modificadas ou não, segundo as propriedades dos ruídos. Para verificar se o protocolo funcionou, foram realizados uma sequencia de testes comparando os valores dos elementos das matrizes descritas pelas variáveis $alfa\_final$ e $beta\_final$ e $alfa\_inicial$ e $beta\_inicial$. As possibilidades são:
\begin{description}
\item [Valores iniciais e finais iguais]: Se todos os elementos das variáveis $\alpha$ e $\beta$ iniciais forem iguais aos elementos das variáveis finais, significa que o protocolo ocorreu com sucesso e não houve a interferência de ruídos no processo.
\item [Valores iniciais e finais invertidos]: Se a posição dos elementos finais e iniciais estiver invertida, significa que o protocolo ocorreu com sucesso porém houve a interferência de ruídos do tipo \textit{bitflip}.
\item [Sinal dos valores iniciais e finais invertidos]: Se o sinal dos valores dos elementos iniciais e finais estiver invertido, significa que o protocolo ocorreu com sucesso porém houve a interferência de ruídos do tipo \textit{phaseflip}.
\item [Valores distintos]: Se os valores dos elementos não forem comparáveis por nenhuma situação descrita acima, significa que houve falha no teletransporte.
\end{description}

\clearpage

\chapter{Discussão dos resultados e perspectivas de estudo}\label{sec:resultados}

Durante a criação da descrição matricial do protocolo apresentado no Apêndice~\ref{app:matricial}, utilizamos matrizes do tipo coluna para representar os estados quânticos e os qubits, pois ambos são vetores. Além disso, pudemos expressar as portas lógicas quânticas como matrizes unitárias, o que permitiu uma implementação eficiente das operações entre as portas e os qubits. Apesar da descrição matricial simplificar o protocolo, desconsiderando a variável temporal, é possível estabelecer uma cronologia das operações, uma vez que algoritmos quânticos podem ser expressos na forma de circuitos.

Durante a construção da descrição matricial, identificamos os seguintes postulados descritos na seção~\ref{sec:postulados}:

\begin{description}
\item O Postulado~\ref{post:p1} mostra que os qubits podem ser representados pela expressão $\ket{\psi} = \alpha \begin{bmatrix} 1 \\ 0 \end{bmatrix} + \beta \begin{bmatrix} 0 \\ 1 \end{bmatrix}$, que é uma combinação linear de vetores unitários, representados por matrizes coluna.
\item O Postulado~\ref{post:p2} evidencia que os operadores quânticos, que são as portas lógicas quânticas, são a única operação capaz de alterar os estados quânticos. Na descrição proposta neste trabalho, observamos que, ao realizar uma operação entre qubits, o estado permanece inalterado, mas ao realizar uma operação entre um qubit e uma porta lógica quântica, o primeiro sempre é modificado. É interessante notar que, a menos que se realize uma operação de medida, o estado de um qubit não é modificado definitivamente ao interagir com uma porta lógica quântica, sendo reversível com a aplicação de uma operação inversa.
\item O Postulado~\ref{post:p4} foi constatado nas operações que envolvem todos os estados do sistema, como na aplicação da porta \(\CNOT\). O estado geral do sistema é obtido com o produto tensorial entre todos os estados que o compõem, o que nos permite descrever detalhadamente as possibilidades de colapso após uma medida em função dos estados obtidos.
\end{description}

A possibilidade de observar os postulados na descrição matricial justifica e valida sua construção, de modo que, apesar de simplificada, ela não viola nenhum dos postulados propostos. Além disso é possível acompanhar a evolução do sistema com o circuito lógico da figura~\ref{fig:protocoloteletransporte} com a ordem de aplicação das portas e os estados dos qubits sendo modificados.

Sobre o desenvolvimento do protocolo em Python, a utilização da biblioteca "Numpy" permitiu uma implementação mais eficiente e simplificada, enquanto a função "TensorProduct" otimizou as operações de produto tensorial necessárias durante todo o protocolo.

A validação dos resultados ocorreu comparando-os com a descrição teórica do protocolo de teletransporte quântico, e confirmou-se que o simulador reproduziu corretamente o estado final do sistema para todos os valores de probabilidade de $\alpha$ e $\beta$, com precisão de até 15 casas decimais.

O impacto da introdução de ruídos no processo de teletransporte quântico foi analisado comparando as variáveis iniciais e finais do protocolo para as probabilidades $\alpha$ e $\beta$ de modo que os estados obtidos em cada situação estão apresentados na tabela~\ref{tab:resultadoruidos}.

\begin{table}[ht!]
  \centering
  \caption{Resultado dos estados de $\alpha$ e $\beta$ na presença de ruídos}\label{tab:resultadoruidos}
  \begin{tabular}{ccccc}
    \toprule
   		 \textbf{Tipo de ruído} & \textbf{$\alpha_{inicial}$} & \textbf{$\beta_{inicial}$} & \textbf{$\alpha_{final}$} & \textbf{$\beta_{final}$} \\
   		 \midrule
   			 \textbf{bitflip} & $\begin{bmatrix} 1 \\ 0 \end{bmatrix}$ & $\begin{bmatrix} 0 \\ 1 \end{bmatrix}$ & $\begin{bmatrix} 0 \\ 1 \end{bmatrix}$ & $\begin{bmatrix} 1 \\ 0 \end{bmatrix}$ \\
    			\textbf{phaseflip} & $\begin{bmatrix} 1 \\ 0 \end{bmatrix}$ & $\begin{bmatrix} 0 \\ 1 \end{bmatrix}$ & $\begin{bmatrix} 1 \\ 0 \end{bmatrix}$ & $\begin{bmatrix} 0 \\ -1 \end{bmatrix}$   
  \end{tabular}
\end{table}

Os resultados da tabela~\ref{tab:resultadoruidos}, pode ser comparado com os descritos em~\ref{tiposruidos} e a inversão de estados é confirmada para o ruído bitflip e a de fase para o ruído phaseflip, o que demonstra que a presença de ruídos compromete o estado quântico teletransportado, o que é relevante para a implementação prática de sistemas de teletransporte quântico.

Embora o protocolo desenvolvido seja introdutório, sua compreensão permite o desenvolvimento de trabalhos mais complexos, como a implementação de um protocolo que considere a evolução temporal como variável e possibilite a descrição mais precisa dos efeitos de ruídos no sistema, ou ainda, a descrição de qubits utilizando sua função de onda e operando sobre ela.

Vale ressaltar que existem algumas ferramentas já consolidadas para o estudo e aperfeiçoamento de algoritmos quânticos, como o Qiskit. O Qiskit é uma plataforma de código aberto desenvolvida pela IBM para programação e simulação de circuitos quânticos, amplamente utilizada em pesquisas e aplicações práticas em diversas áreas, como criptografia, simulação química e inteligência artificial.

Com este estudo introdutório, algumas perspectivas ficam delineadas para estudos futuros, como a implementação do mesmo protocolo utilizando o Qiskit em um ambiente clássico e posteriormente em um ambiente quântico para verificar possíveis divergências nos resultados. Outra perspectiva futura é o mapeamento de erros para o desenvolvimento e aperfeiçoamento de algoritmos de correção de erros quânticos, na tentativa de tornar a implementação de um protocolo de teletransporte real mais efetiva para a evolução da computação quântica.

\clearpage

\chapter{Conclusões}

A realização deste trabalho possibilitou além de desenvolver uma notação matricial para o protocolo de teletransporte o que facilita a visualização da alteração dos estados como também sua implementação como estrutura de dados para a automatização do processo, aprofundando os estudos em MQ em uma aplicação direta de suas propriedades.

Como um computador puramente quântico ainda está em desenvolvimento, os estudos sobre o tema estão em seu auge e a possibilidade de estudar e desenvolver algoritmos que um dia atuarão nestes dispositivos ressalta a necessidade da constante expansão da ciência. Nos últimos 60 anos, vimos o computador clássico passar de uma versão gigantesca para dispositivos miniaturizados e apenas nos últimos 20 o computador quântico saiu do papel com o desenvolvimento de processadores quânticos. Portanto, é plausível acreditar que existe muito ainda por vir.

Para além, os resultados deste estudo introdutório se mostraram otimistas para a compreensão de algoritmos quânticos, seja por sua descrição físico-matemática ou por sua representação gráfica na forma dos circuitos quânticos e podem servir de base para a compreensão de sistemas mais complexos e sofisticados, possíveis de ser implementados com a tecnologia quântica.

Espera-se que este trabalho possa servir para a orientação e o estudo dessa temática visto que, por ainda estar em desenvolvimento, possuí aspectos ainda não consolidados e que dependem de novos estudos.

 




