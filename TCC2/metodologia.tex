%%%  __  __      _            _
%%% |  \/  | ___| |_ ___   __| | ___
%%% | |\/| |/ _ \ __/ _ \ / _` |/ _ \
%%% | |  | |  __/ || (_) | (_| | (_) |
%%% |_|  |_|\___|\__\___/ \__,_|\___/
%%%
%%% TCC de Bianca Miyabe Santos Freitas
%%% Licenciatura em Física - UFSCar, Sorocaba
%%%

\chapter{Metodologia}

Para iniciar o desenvolvimento de um protocolo de teletransporte quântico, inicialmente realizou-se a dedução do mesmo em forma matricial, conforme apresentado detalhadamente no Apêndice~\ref{app:matricial}.

Para alcançar o objetivo deste trabalho, ou seja, construir uma simulação de um teletransporte quântico, foi necessário inicialmente decidir a linguagem a ser utilizada para o projeto. A escolha da linguagem Python (versão 3.11) se deveu ao fato dela ser amplamente utilizada para simular situações físicas e possuir uma série de pacotes abrangendo as mais diversas áreas do conhecimento. Os principais pacotes utilizados no projeto foram \py{sympy} \cite{sympy}, \py{numpy} \cite{harris2020array}, \py{math}\footnote{\url{https://docs.python.org/3/library/math.html}}, \py{sys}\footnote{\url{https://docs.python.org/3/library/sys.html}} e \py{random}\footnote{\url{https://docs.python.org/3/library/random.html}}, e sua utilização será explicitada durante a explicação das etapas do código fonte. Antes de iniciar a construção do código, foi necessário analisar a estrutura de dados que seria utilizada e a escolhida foi a representação matricial. Essa escolha foi fundamentada no fato de que desse modo, as operações ficam mais explícitas, facilitando a compreensão do protocolo.

De posse da estrutura de dados a ser operacionalizada, da linguagem de programação e dos pacotes necessários, foi desenvolvido o código apresentado no Apêndice~\ref{app:protocolo}. Vamos detalhar as operações e correlacioná-las a seguir.

A primeira etapa da elaboração do protocolo consistiu em importar as funções e pacotes necessários como a \py{TensorProduct} em \py{sympy.physics.quantum}. Essa função irá facilitar a realização dos produtos tensoriais durante o desenvolvimento do protocolo.

Em seguida é feita a criação das variáveis que descrevem os estados quânticos $\ket{0}$ e $\ket{1}$ com o auxílio da clase \py{numpy.matrix}, que cria matrizes a partir de listas, como pode ser conferido na Tabela~\ref{tab:cria_quant}.

\begin{table}[ht!]
  \centering
  \caption{Criação dos estados quânticos \(\ket{0}\) e \(\ket{1}\) em Python.} \label{tab:cria_quant}
  \begin{tabular}{cl}
    \toprule
    Estado    & Python                                    \\
    \midrule
    $\ket{0}$ & \py{qbit0 = np.matrix([1,0]).transpose()} \\
    $\ket{1}$ & \py{qbit1 = np.matrix([0,1]).transpose()} \\
    \bottomrule
  \end{tabular}
\end{table}
Foram estabelecidas também as variáveis que representam os operadores quânticos, para a porta \(\HAD\), \(\CNOT\), \(\III\), \(\XXX\) e \(\ZZZ\) foram criadas as variáveis \py{H}, \py{CNOT}, \py{I}, \py{X} e \py{Z}, respectivamente, de acordo com a Tabela~\ref{tab:op_quant}.

\begin{table}[ht!]
  \centering
  \caption{Operadores quânticos.}\label{tab:op_quant}
  \begin{tabular}{cl}
    \toprule
    Operador  & Python                                                           \\
    \midrule
    \(\HAD\)  & \py{H = 1/sqrt(2)*(np.matrix([[1,1], [1,-1]]))}                  \\
    \(\CNOT\) & \py{CNOT = np.matrix([[1,0,0,0],[0,1,0,0],[0,0,0,1],[0,0,1,0]])} \\
    \(\III\)  & \py{I = np.matrix ([[1,0], [0,1]])}                              \\
    \(\XXX\)  & \py{X = np.matrix([[0, 1], [1, 0]])}                             \\
    \(\ZZZ\)  & \py{Z = np.matrix([[1, 0], [0, -1]])}                            \\
    \bottomrule
  \end{tabular}
\end{table}

Conforme a Figura~\ref{fig:protocoloteletransporte}, a primeira operação realizada é o emaranhamento de dois qubits, que no nosso caso, estão emaranhados pelo estado $\ket{00}$. Iniciamos portanto, com a aplicação da porta \(\HAD\) no qubit $\ket{q_x}$ \textcolor{orange}{que está no estado}, para isso, criou-se a variável \py{Hqbit0} como \py{H*qbit0}. Nesse momento, os resultados obtidos são comparáveis com a Equação~\eqref{eq:Hqbit0}.

Em seguida, a porta \(\CNOT\) deve ser aplicada e a definição dos qubits alvo e controle foi implementada na variável \py{AC} e é definida por \py{AC= TP(Hqbit0, qbit0)}, ou seja, o produto tensorial de todos os estados possíveis dentro do sistema. A aplicação de \(\CNOT\) resulta na base de Bell no estado $\ket{00}$ e é obtida na variável \py{Bell00 = CNOT * AC}. Nesse momento, temos o sistema descrito pela base de Bell da mesma forma da Equação~\eqref{eq:matrizbell00}.

\todo[inline]{Fiquei pensando\ldots Talvez fosse mais esperto pedir o \(\alpha\) e definir o \(\beta = 1-\alpha\)}

No protocolo construído, é solicitado ao usuário que descreva as probabilidades associadas às variáveis $\alpha$ e $\beta$. Para tal, utiliza-se a função \py{input} e declarando a variável como decimal (\py{float}), solicita-se ao usuário que entre com os valores desejados para cada variável. Note que, como a superposição de estados tem natureza probabilística, é necessário implementar uma verificação da condição de normalização, para que os valores inseridos pelo usuário contemplem a probabilidade somada de $100\%$ caso contrário, o protocolo é interrompido.

A determinação do qubit que será enviado é construída pela variável \py{estado_inicial} de modo que a probabilidade de $\alpha$ fique associada ao estado $\ket{0}$ e a probabilidade de $\beta$ fique associada ao estado $\ket{1}$, a variável é descrita portanto por
\begin{pycode}
  estado_inicial = (estado_alfa * qbit0) + (estado_beta * qbit1)
\end{pycode}

A próxima etapa do protocolo consiste na aplicação da porta \(\CNOT\) nos três qubits que compõem o estado geral do sistema. Os três qubits são o estado inicial a ser teletransportado atuando como controle e o par emaranhado atuando como alvo. A atuação ocorre apenas no qubit presente no local \(A\), porém afeta probabilisticamente o qubit no local \(B\). Para definir o estado geral são somados os produtos tensoriais de todos os estados dos qubits com sua representação no código descrita como
\begin{pycode}
  psi_0 = TP(qbit0, qbit0, qbit0) + TP(qbit0, qbit1, qbit1) + \
          TP(qbit1, qbit0, qbit0) + TP(qbit1, qbit1, qbit1)
\end{pycode}
Em seguida, a porta \(\CNOT\) é dimensionada para atuar sob três qubits com a realização do produto tensorial entre \(\CNOT\) e \(\III\) e por último multiplicamos a porta \(\CNOT\) pela soma dos estados dos qubits. Portanto, o estado $\ket{\psi_1}$ é operacionalizado pela multiplicação da porta \(\CNOT\) dimensionada no estado $\ket{\psi_0}$ ficando evidenciada em \py{psi_1 = CNOT * psi_0}.

Para proceder com a aplicação da porta Hadamard no qubit de mensagem, os estados associados a $\alpha$ e $\beta$ foram separados e a porta aplicada em cada um deles individualmente pelas multiplicações
\begin{pycode}
  H_estado_alfa = H * (estado_alfa * qbit0)
  H_estado_beta = H * (estado_beta * qbit1)
\end{pycode}
Com esses resultados, foram reestruturados os estados $\alpha\ket{0}, \alpha\ket{1}, \beta\ket{0},$ e $\beta\ket{1}$ conforme é explicitado na Equação~\eqref{eq:alfabetasoma}. Para isso, foram construídas quatro variáveis, uma para cada estado nomeadas \py{estadoa0}, \py{estadoa1}, \py{estadob0} e \py{estadob1}, cada uma dessas variáveis recupera a posição do valor das matrizes \py{H_estado_alfa} e \py{H_estado_beta} para satisfazer a condição dada em~\eqref{eq:alfabetasoma}, ou seja, o primeiro elemento da matriz está associado aos estados $\alpha\ket{0}$ e $\beta\ket{0}$ e o segundo elemento da matriz associado aos estados $\alpha\ket{1}$ e $\beta\ket{1}$. Essa associação é feita pela posição do elemento na matriz, lembrando que a contagem de matrizes em Python inicia em 0 de modo que as variáveis foram definidas como:
\begin{pycode}
  estadoa0 = float(H_estado_alfa[0][0])
  estadoa1 = float(H_estado_alfa[1][0])
  estadob0 = float(H_estado_beta[0][0])
  estadob1 = float(H_estado_beta[1][0])
\end{pycode}
Em seguida, recriamos as matrizes que representam os estados de posse dos elementos localizados anteriormente, temos portanto:
\begin{pycode}
  a0 = np.matrix([estadoa0,0]).transpose()
  a1 = np.matrix([0,estadoa1]).transpose()
  b0 = np.matrix([estadob0,0]).transpose()
  b1 = np.matrix([0,estadob1]).transpose()
\end{pycode}
Nesse momento temos em nosso código dois estados separados para $\ket{\psi_2}$, o estado associado ao qubit mensagem e o estado associado ao par emaranhado. Para recuperar os estados possíveis do par emaranhado, criamos as variáveis \py{estado_00}, \py{estado_11}, \py{estado_10} e \py{estado_01} usando a função de produto tensorial:
\begin{pycode}
  estado_00 = TP(qbit0, qbit0)
  estado_11 = TP(qbit1, qbit1)
  estado_10 = TP(qbit1, qbit0)
  estado_01 = TP(qbit0, qbit1)
\end{pycode}
Para simular a operação de medição, criamos uma lista de variáveis, armazenada na variável \py{group_estados}, determinamos com a função \py{random} a escolha de um desses estados de maneira aleatória, armazenamos a escolha na variável \py{Medição} e utilizando o comando \py{del} apagamos todos os estados, simulando o colapso do sistema após a medida.

Para reconstruir a mensagem, a condição estabelecida foi associar o estado medido armazenado na variável \py{Medição} com o estado da mensagem baseando-se na Equação~\eqref{eq:final}, pela relação apresentada na Tabela~\ref{tab:medidas}. O código para essa implementação é apresentado na Listagem~\ref{lst:teleA}.

\begin{listing}[ht!]
  \caption{Relação de condição para o estado teletransportado em função do estado medido em~\(A\).}\label{lst:teleA}
  \begin{pycode}
      if np.all(Medição == TP(qbit0, qbit0)):
          estado_tp = a0-b1
      elif np.all(Medição == TP(qbit1, qbit1)):
          estado_tp = a1-b0
      elif np.all(Medição == TP(qbit0, qbit1)):
          estado_tp = a1+b0
      elif np.all(Medição == TP(qbit1, qbit0)):
          estado_tp = a0+b1
  \end{pycode}
\end{listing}

Com a variável \py{estado_tp} temos portando o estado que o qubit no local \(B\) possui. Para recuperar a mensagem, iremos aplicar as matrizes de Pauli, seguindo as relações propostas na Tabela~\ref{tab:acao-das-portas}. A condição estabelecida é dada pelo código da Listagem~\ref{lst:condPauli}.

\begin{listing}[ht!]
  \caption{Relação de condição para aplicação das portas de Pauli.}\label{lst:condPauli}
  \begin{pycode}
      if np.all(estado_tp == (a0-b1)):
          estado_final = estado_tp
      elif np.all(estado_tp == (a1-b0)):
          estado_final = Z * X * estado_tp
      elif np.all(estado_tp == (a1+b0)):
          estado_final = X * estado_tp
      elif np.all(estado_tp == (a0+b1)):
          estado_final = Z * estado_tp
    
      estado_teletransportado = estado_final*sqrt(2)

      alfa = float(estado_teletransportado[0][0])
      beta = float(estado_teletransportado[1][0])

      alfa_final = np.matrix([alfa,0]).transpose()
      beta_final = np.matrix([0,beta]).transpose()
  \end{pycode}
\end{listing}

Vale ressaltar que durante as operações, herdamos a probabilidade associada dos estados, que nesse caso é $1/4$ e portanto, a retiramos para recuperar a mensagem idêntica a informada no inicio com a expressão Nesse momento, a variável \py{estado_teletransportado} armazena a mensagem reconstruída no local \(B\). Para verificarmos se o teletransporte teve sucesso, separamos as probabilidades associadas a $\alpha$ e $\beta$ para comparamos com o estado original.

Para simular os ruídos, implementamos as variáveis \py{bitflip} e \py{phaseflip} equivalentes com as seguintes matrizes
\begin{pycode}
  bitflip = np.matrix([[0, 1], [1, 0]])
  phaseflip = np.matrix([[1, 0], [0, -1]])
\end{pycode}
O usuário pode escolher qual ruído deseja implementar, ou se deseja não implementar ruídos. Na aplicação do ruído, as variáveis \py{alfa_final} e \py{beta_final} serão modificadas ou não, segundo as propriedades dos ruídos. Para verificar se o protocolo funcionou, foi realizada uma sequência de testes comparando os valores dos elementos das matrizes descritas pelas variáveis \py{alfa_final} e \py{beta_final} e \py{alfa_inicial} e \py{beta_inicial}. As possibilidades são:
\begin{description}
  \item [Valores iniciais e finais iguais] Se todos os elementos das variáveis $\alpha$ e $\beta$ iniciais forem iguais aos elementos das variáveis finais, significa que o protocolo ocorreu com sucesso e não houve a interferência de ruídos no processo.
  \item [Valores iniciais e finais invertidos] Se a posição dos elementos finais e iniciais estiver invertida, significa que o protocolo ocorreu com sucesso porém houve a interferência de ruídos do tipo \textit{bitflip}.
  \item [Sinal dos valores iniciais e finais invertidos] Se o sinal dos valores dos elementos iniciais e finais estiver invertido, significa que o protocolo ocorreu com sucesso porém houve a interferência de ruídos do tipo \textit{phaseflip}.
  \item [Valores distintos] Se os valores dos elementos não forem comparáveis por nenhuma situação descrita acima, significa que houve falha no teletransporte.
\end{description}
