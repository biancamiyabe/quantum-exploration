%%%  ____  _____ ____  _   _ _   _____
%%% |  _ \| ____/ ___|| | | | | |_   _|
%%% | |_) |  _| \___ \| | | | |   | |
%%% |  _ <| |___ ___) | |_| | |___| |
%%% |_| \_\_____|____/ \___/|_____|_|
%%%
%%% TCC de Bianca Miyabe Santos Freitas
%%% Licenciatura em Física - UFSCar, Sorocaba
%%%
\chapter{Resultados}

Para aferir a implementação proposta, será apresentado um exemplo com os valores atribuídos de \alpha = \norma\ e \beta = \normb.  Assim, a mensagem a ser enviada é descrita por:
\begin{equation}
  \ket{\psi} =  \begin{bmatrix} {\norma} \\ 0 \end{bmatrix} +
  \begin{bmatrix} 0 \\ {\normb} \end{bmatrix} =
  \begin{bmatrix} {\norma} \\ {\normb} \end{bmatrix}.
\end{equation}
O estado $\ket{\psi_0}$ descrito pela Equação~\eqref{eq:psi0} torna-se:
\begin{equation}
  \ket{\psi_0} = \ket{\psi}\ket{\Phi_+} =
  \begin{bmatrix} {\norma} \\ {\normb} \end{bmatrix}
  \left(\frac{1}{\sqrt{2}} \begin{bmatrix} 1 \\ 0 \\ 0 \\ 1 \end{bmatrix}\right).
\end{equation}
Rearranjando os termos segundo~\eqref{eq:psi0cnot}
\begin{equation}
  \ket{\psi_0} = \frac{1}{\sqrt{2}}\left(
      \begin{bmatrix} {\norma} \\ 0 \end{bmatrix}
      \begin{bmatrix} 1 \\ 0 \\ 0 \\ 1 \end{bmatrix} +
      \begin{bmatrix} 0 \\ {\normb} \end{bmatrix}
      \begin{bmatrix} 1 \\ 0 \\ 0 \\ 1 \end{bmatrix}\right).
  \end{equation}

Explicitando os estados possíveis conforme~\eqref{eq:3qubit}
\begin{equation}
  \begin{split}
    \ket{\alpha 00} &= \begin{bmatrix} {\norma} \\ 0 \end{bmatrix} \otimes
                     \begin{bmatrix} 1 \\ 0 \end{bmatrix} \otimes
                     \begin{bmatrix} 1 \\ 0 \end{bmatrix} =
                     \begin{bmatrix} {\norma} & 0 & 0 & 0 & 0 & 0 & 0 & 0 \end{bmatrix}^T, \\
    \ket{\alpha 11} &= \begin{bmatrix} {\norma} \\ 0 \end{bmatrix} \otimes
                        \begin{bmatrix} 0 \\ 1 \end{bmatrix} \otimes
                        \begin{bmatrix} 0 \\ 1 \end{bmatrix} =
                        \begin{bmatrix} 0 & 0 & 0 & {\norma} & 0 & 0 & 0 & 0 \end{bmatrix}^T, \\
    \ket{\beta 00} &= \begin{bmatrix} 0 \\ {\normb} \end{bmatrix} \otimes
                    \begin{bmatrix} 1 \\ 0 \end{bmatrix} \otimes
                    \begin{bmatrix} 1 \\ 0 \end{bmatrix} =
                    \begin{bmatrix} 0 & 0 & 0 & 0 & {\normb} & 0 & 0 & 0 \end{bmatrix}^T, \\
    \ket{\beta 11} &= \begin{bmatrix} 0 \\ {\normb} \end{bmatrix} \otimes
                       \begin{bmatrix} 0 \\ 1 \end{bmatrix} \otimes
                       \begin{bmatrix} 0 \\ 1 \end{bmatrix} =
                       \begin{bmatrix} 0 & 0 & 0 & 0 & 0 & 0 & 0 & {\normb} \end{bmatrix}^T.
  \end{split}
\end{equation}
Somando todos os estados possíveis temos:
\begin{equation}
  \ket{\alpha 00}+\ket{\alpha 01}+\ket{\beta 00}+\ket{\beta 11} =
  \begin{bmatrix} {\norma} & 0 & 0 & {\norma} & {\normb} & 0 & 0 & {\normb} \end{bmatrix}^T.
\end{equation}
Aplicando a porta \(\CNOT\)
\begin{equation}
  \begin{split}
    \ket{\psi_1} = \CNOT (\ket{\alpha 00}+\ket{\alpha 01}+\ket{\beta 00}+\ket{\beta 11}) = \\
   = \begin{bmatrix}
    1 & 0 & 0 & 0 & 0 & 0 & 0 & 0 \\
    0 & 1 & 0 & 0 & 0 & 0 & 0 & 0 \\
    0 & 0 & 1 & 0 & 0 & 0 & 0 & 0 \\
    0 & 0 & 0 & 1 & 0 & 0 & 0 & 0 \\
    0 & 0 & 0 & 0 & 0 & 0 & 1 & 0 \\
    0 & 0 & 0 & 0 & 0 & 0 & 0 & 1 \\
    0 & 0 & 0 & 0 & 1 & 0 & 0 & 0 \\
    0 & 0 & 0 & 0 & 0 & 1 & 0 & 0
    \end{bmatrix}
    \begin{bmatrix} {\norma} \\ 0 \\ 0 \\ {\norma} \\ {\normb} \\ 0 \\ 0 \\ {\normb} \end{bmatrix} =
    \begin{bmatrix} {\norma} \\ 0 \\ 0 \\ {\norma} \\ 0 \\ {\normb} \\ {\normb} \\ 0 \end{bmatrix}.
  \end{split}
\end{equation}

Reorganizando o estado $\ket{\psi_1}$ temos, segundo~\eqref{eq:psi1},
\begin{equation}
  \ket{\psi_1} = \frac{1}{\sqrt{2}} \left\{ \begin{bmatrix} {\norma} \\ 0 \end{bmatrix} \otimes
    \left(
      \begin{bmatrix} 1 \\ 0 \\ 0 \\ 0 \end{bmatrix} +
      \begin{bmatrix} 0 \\ 0 \\ 0 \\ 1 \end{bmatrix}
    \right) +
    \begin{bmatrix} 0 \\ {\normb} \end{bmatrix} \otimes
    \left(
      \begin{bmatrix} 0 \\ 0 \\ 1 \\ 0 \end{bmatrix} +
      \begin{bmatrix} 0 \\ 1 \\ 0 \\ 0 \end{bmatrix}
    \right) \right\}.
\end{equation}
Aplicando a porta \(\HAD\):
\begin{equation}
  \label{eq:1}
  \begin{aligned}
    \HAD \ket{\psi_{1}} &= \frac{1}{\sqrt{2}}
                          \begin{bmatrix*}[r] 1 & 1 \\ 1 & -1 \end{bmatrix*}
                          \frac{1}{\sqrt{2}}\left(
                          \begin{bmatrix} {\norma} \\ 0 \end{bmatrix} +
                          \begin{bmatrix} 0 \\ {\normb} \end{bmatrix} \right)\\
                        &= \frac{1}{2} \left(
                          \begin{bmatrix} {\norma} \\ {\norma} \end{bmatrix} +
                          \beta \begin{bmatrix*}[r] {\normb} \\ {0,3-0,6i} \end{bmatrix*} \right).
  \end{aligned}
\end{equation}
Reorganizando os resultados para definir o estado $\ket{\psi_2}$ temos
\begin{equation}
 \begin{split}
   \ket{\psi_2} &=\frac{1}{2} \left\{ \left(\begin{bmatrix} 1 \\ 0 \end{bmatrix} \otimes
                  \begin{bmatrix} 1 \\ 0 \end{bmatrix}\right) \left( \begin{bmatrix} {\norma} \\ 0 \end{bmatrix} +
                  \begin{bmatrix} 0 \\ {\normb} \end{bmatrix}\right)\right\} \\
                &+\frac{1}{2} \left\{ \left(\begin{bmatrix} 0 \\ 1 \end{bmatrix} \otimes
                  \begin{bmatrix} 0 \\ 1 \end{bmatrix}\right) \left( \begin{bmatrix} 0 \\ {\norma} \end{bmatrix} -
                  \begin{bmatrix} {\normb} \\ 0 \end{bmatrix}\right)\right\} \\
                &+\frac{1}{2} \left\{\left( \begin{bmatrix} 1 \\ 0 \end{bmatrix} \otimes
                  \begin{bmatrix} 0 \\ 1 \end{bmatrix}\right) \left( \begin{bmatrix} 0 \\ {\norma} \end{bmatrix} +
                  \begin{bmatrix} {\normb} \\ 0 \end{bmatrix}\right)\right\} \\
                &+\frac{1}{2} \left\{ \left[ \begin{bmatrix} 0 \\ 1 \end{bmatrix} \otimes
                  \begin{bmatrix} 1 \\ 0 \end{bmatrix}\right] \left( \begin{bmatrix} {\norma} \\ 0 \end{bmatrix} -
                  \begin{bmatrix} 0 \\ {\normb} \end{bmatrix}\right) \right\}.
  \end{split}
\end{equation}

Para reconstruir o estado, aplicando as condições descritas na Tabela~\ref{tab:acao-das-portas} teremos:

\begin{enumerate}
  \item Se a medida realizada for o estado $\ket{00}$, o estado enviado foi
        \[\ket{\psi} = \begin{bmatrix} {\norma} \\ 0 \end{bmatrix} + \begin{bmatrix} 0 \\ {\normb} \end{bmatrix} = \begin{bmatrix} {\norma} \\ {\normb} \end{bmatrix}. \]

  \item Se a medida realizada for o estado $\ket{11}$:
        \[
        \begin{bmatrix} 0 \\ {\norma} \end{bmatrix} - \begin{bmatrix} {\normb} \\ 0 \end{bmatrix} = \begin{bmatrix*}[r] {0,3-0,6i} \\ {\norma} \end{bmatrix*}.
        \]

        Aplicando \(\XXX\):
        \[
        \begin{bmatrix} 0 & 1 \\ 1 & 0 \end{bmatrix} \begin{bmatrix*}[r] {0,3-0,6i} \\ {\norma} \end{bmatrix*} = \begin{bmatrix*}[r] {\norma} \\ {0,3-0,6i} \end{bmatrix*}.
        \]

        Aplicando \(\ZZZ\):
        \[
        \begin{bmatrix*}[r] 1 & 0 \\ 0 & -1 \end{bmatrix*}\begin{bmatrix*}[r] {\norma} \\ {0,3-0,6i} \end{bmatrix*} = \begin{bmatrix} {\norma} \\ {\normb} \end{bmatrix}.
        \]

        Portanto o estado recuperado é:
        \[
        \begin{bmatrix} {\norma} \\ {\normb} \end{bmatrix} = \begin{bmatrix} {\norma} \\ 0 \end{bmatrix} +  \begin{bmatrix} 0 \\ {\normb} \end{bmatrix}.
        \]

  \item Se a medida realizada for o estado $\ket{10}$:
        \[
        \begin{bmatrix} {\norma} \\ 0 \end{bmatrix} - \begin{bmatrix} 0 \\ {\normb} \end{bmatrix} = \begin{bmatrix*}[r] {\norma} \\ {0,3-0,6i} \end{bmatrix*}.
        \]

        Aplicando \(\ZZZ\):
        \[
        \begin{bmatrix*}[r] 1 & 0 \\ 0 & -1 \end{bmatrix*}\begin{bmatrix*}[r] {\norma} \\ {0,3-0,6i} \end{bmatrix*} = \begin{bmatrix} {\norma} \\ {\normb} \end{bmatrix}.
        \]

  \item Por último, se a medida realizada for o estado $\ket{01}$:
        \[
        \begin{bmatrix} {\normb} \\ 0 \end{bmatrix} +  \begin{bmatrix} 0 \\ {\norma} \end{bmatrix} = \begin{bmatrix} {\normb} \\ {\norma} \end{bmatrix}.
        \]

        Aplicando \(\XXX\):
        \[
        \begin{bmatrix} 0 & 1 \\ 1 & 0 \end{bmatrix} \begin{bmatrix} {\normb} \\ {\norma} \end{bmatrix} = \begin{bmatrix} {\norma} \\ {\normb} \end{bmatrix}.
        \]
\end{enumerate}

Todos os estados recuperados são iguais aos estados enviados e portanto a dedução apresentada no Apêndice~\ref{app:matricial} pode ser utilizada como um guia para um programa que automatize tais operações.
\begin{table}[ht]
  \centering
  \caption{Resultado dos estados de $\alpha$ e $\beta$ na presença de ruídos.}\label{tab:resultadoruidos}
  \begin{tabular}{ccccc}
    \toprule
    \multirow{2}{*}{Tipo de ruído} & \multicolumn{2}{c}{Valores iniciais} & \multicolumn{2}{c}{Valores finais}                                     \\
    \cmidrule{2-5}
                                   & $\alpha$ & $\beta$ & $\alpha$ & $\beta$ \\
    \midrule
    \textit{bitflip}   & $\begin{bmatrix} 1 \\ 0 \end{bmatrix}$ & $\begin{bmatrix} 0 \\ 1 \end{bmatrix}$ & $\begin{bmatrix} 0 \\ 1 \end{bmatrix}$ & $\begin{bmatrix} 1 \\ 0 \end{bmatrix}$ \\
    \textit{phaseflip} & $\begin{bmatrix} 1 \\ 0 \end{bmatrix}$ & $\begin{bmatrix} 0 \\ 1 \end{bmatrix}$ & $\begin{bmatrix} 1 \\ 0 \end{bmatrix}$ & $\begin{bmatrix*}[r] 0 \\ -1 \end{bmatrix*}$ \\
    \bottomrule
  \end{tabular}
\end{table}
O protocolo de teletransporte quântico apresentado possui como entradas os valores de probabilidade associados ao estado que se deseja teletransportar, e como saída a comparação deste com possíveis interações com ruídos. O impacto da introdução de ruídos no processo de teletransporte quântico foi analisado comparando as variáveis iniciais e finais do protocolo para as amplitudes de probabilidades $\alpha$ e $\beta$ de modo que os estados obtidos em cada situação estão apresentados na Tabela~\ref{tab:resultadoruidos}.

Portanto, de acordo com o resultado obtido pelo protocolo, é possível saber se ele foi realizado na presença ou não de ruídos, e ainda qual o tipo de ruído interferiu no estado que foi teletransportado.

% resultados.tex
