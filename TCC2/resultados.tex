%%%  ____  _____ ____  _   _ _   _____
%%% |  _ \| ____/ ___|| | | | | |_   _|
%%% | |_) |  _| \___ \| | | | |   | |
%%% |  _ <| |___ ___) | |_| | |___| |
%%% |_| \_\_____|____/ \___/|_____|_|
%%%
%%% TCC de Bianca Miyabe Santos Freitas
%%% Licenciatura em Física - UFSCar, Sorocaba
%%%
\chapter{Resultados}

Para aferir a implementação proposta, será apresentado um exemplo com os valores atribuídos de $\alpha = \num{0.75}$ e $\beta = \num{0.25}$, teremos que a mensagem a ser enviada é descrita por:
\begin{equation}
  \ket{\psi} = \num{0.75} \begin{bmatrix} 1 \\ 0 \end{bmatrix} +
  \num{0.25} \begin{bmatrix} 0 \\ 1 \end{bmatrix} =
  \begin{bmatrix} \num{0.75} \\ 0 \end{bmatrix} +
  \begin{bmatrix} 0 \\ \num{0.25} \end{bmatrix} =
  \begin{bmatrix} \num{0.75} \\ \num{0.25} \end{bmatrix}.
\end{equation}
O estado $\ket{\psi_0}$ descrito pela Equação~\eqref{eq:psi0} é descrito por:
\begin{equation}
  \ket{\psi_0} = \ket{\psi}\ket{\beta_{00}} =
  \begin{bmatrix} \num{0.75} \\ \num{0.25} \end{bmatrix}
  \left(\frac{1}{\sqrt{2}} \begin{bmatrix} 1 \\ 0 \\ 0 \\ 1 \end{bmatrix}\right).
\end{equation}
Rearranjando os termos segundo~\eqref{eq:psi0cnot}
\begin{equation}
  \ket{\psi_0} = \frac{1}{\sqrt{2}}\left(
      \begin{bmatrix} \num{0.75} \\ 0 \end{bmatrix}
      \begin{bmatrix} 1 \\ 0 \\ 0 \\ 1 \end{bmatrix} +
      \begin{bmatrix} 0 \\ \num{0.25} \end{bmatrix}
      \begin{bmatrix} 1 \\ 0 \\ 0 \\ 1 \end{bmatrix}\right).
  \end{equation}

Explicitando os estados possíveis conforme~\eqref{eq:3qubit}
\begin{equation}
  \begin{split}
    \ket{\alpha_{00}} &= \begin{bmatrix} \num{0.75} \\ 0 \end{bmatrix} \otimes
                     \begin{bmatrix} 1 \\ 0 \end{bmatrix} \otimes
                     \begin{bmatrix} 1 \\ 0 \end{bmatrix} =
                     \begin{bmatrix} \num{0.75} & 0 & 0 & 0 & 0 & 0 & 0 & 0 \end{bmatrix}^T, \\
    \ket{\alpha_{11}} &= \begin{bmatrix} \num{0.75} \\ 0 \end{bmatrix} \otimes
                        \begin{bmatrix} 0 \\ 1 \end{bmatrix} \otimes
                        \begin{bmatrix} 0 \\ 1 \end{bmatrix} =
                        \begin{bmatrix} 0 & 0 & 0 & \num{0.75} & 0 & 0 & 0 & 0 \end{bmatrix}^T, \\
    \ket{\beta_{00}} &= \begin{bmatrix} 0 \\ \num{0.25} \end{bmatrix} \otimes
                    \begin{bmatrix} 1 \\ 0 \end{bmatrix} \otimes
                    \begin{bmatrix} 1 \\ 0 \end{bmatrix} =
                    \begin{bmatrix} 0 & 0 & 0 & 0 & \num{0.25} & 0 & 0 & 0 \end{bmatrix}^T, \\
    \ket{\beta_{11}} &= \begin{bmatrix} 0 \\ \num{0.25} \end{bmatrix} \otimes
                       \begin{bmatrix} 0 \\ 1 \end{bmatrix} \otimes
                       \begin{bmatrix} 0 \\ 1 \end{bmatrix} =
                       \begin{bmatrix} 0 & 0 & 0 & 0 & 0 & 0 & 0 & \num{0.25} \end{bmatrix}^T.
  \end{split}
\end{equation}
Somando todos os estados possíveis temos:
\begin{equation}
  \ket{\alpha_{00}}+\ket{\alpha_{01}}+\ket{\beta_{00}}+\ket{\beta_{11}} =
  \begin{bmatrix} \num{0.75} & 0 & 0 & \num{0.75} & \num{0.25} & 0 & 0 & \num{0.25} \end{bmatrix}^T.
\end{equation}
Aplicando a porta \(\CNOT\)
\begin{equation}
  \begin{split}
    \CNOT (\ket{\alpha_{00}}\ket{\alpha_{01}}\ket{\beta_{00}}\ket{\beta_{11}}) &=
    \begin{bmatrix}
    1 & 0 & 0 & 0 & 0 & 0 & 0 & 0 \\
    0 & 1 & 0 & 0 & 0 & 0 & 0 & 0 \\
    0 & 0 & 1 & 0 & 0 & 0 & 0 & 0 \\
    0 & 0 & 0 & 1 & 0 & 0 & 0 & 0 \\
    0 & 0 & 0 & 0 & 0 & 0 & 1 & 0 \\
    0 & 0 & 0 & 0 & 0 & 0 & 0 & 1 \\
    0 & 0 & 0 & 0 & 1 & 0 & 0 & 0 \\
    0 & 0 & 0 & 0 & 0 & 1 & 0 & 0
    \end{bmatrix}
    \begin{bmatrix} \num{0.75} \\ 0 \\ 0 \\ \num{0.75} \\ \num{0.25} \\ 0 \\ 0 \\ \num{0.25} \end{bmatrix} =
    \begin{bmatrix} \num{0.75} \\ 0 \\ 0 \\ \num{0.75} \\ 0 \\ \num{0.25} \\ \num{0.25} \\ 0 \end{bmatrix}.
  \end{split}
\end{equation}

Reorganizando o estado $\ket{\psi_1}$ temos, segundo~\eqref{eq:psi1}
\begin{equation}
  \ket{\psi_1} = \frac{1}{\sqrt{2}} \left\{ \begin{bmatrix} \num{0.75} \\ 0 \end{bmatrix} \otimes
    \left(
      \begin{bmatrix} 1 \\ 0 \\ 0 \\ 0 \end{bmatrix} +
      \begin{bmatrix} 0 \\ 0 \\ 0 \\ 1 \end{bmatrix}
    \right) +
    \begin{bmatrix} 0 \\ \num{0.25} \end{bmatrix} \otimes
    \left(
      \begin{bmatrix} 0 \\ 0 \\ 1 \\ 0 \end{bmatrix} +
      \begin{bmatrix} 0 \\ 1 \\ 0 \\ 0 \end{bmatrix}
    \right) \right\}.
\end{equation}
Aplicando a porta \(\HAD\):
\begin{equation}
  \HAD \ket{\psi} = \frac{1}{\sqrt{2}}
  \begin{bmatrix*}[r] 1 & 1 \\ 1 & -1 \end{bmatrix*}
  \frac{1}{\sqrt{2}}\left(
    \begin{bmatrix} \num{0.75} \\ 0 \end{bmatrix} +
    \begin{bmatrix} 0 \\ \num{0.25} \end{bmatrix} \right) =
  \frac{1}{2} \left( \begin{bmatrix} \num{0.75} \\ \num{0.75} \end{bmatrix} +
    \beta \begin{bmatrix*}[r] \num{0.25} \\ -\num{0.25} \end{bmatrix*} \right).
\end{equation}

Reorganizando os resultados para definir o estado $\ket{\psi_2}$ temos
\begin{equation}
 \begin{split}
   \ket{\psi_2} &=\frac{1}{2} \left\{ \left(\begin{bmatrix} 1 \\ 0 \end{bmatrix} \otimes
                  \begin{bmatrix} 1 \\ 0 \end{bmatrix}\right) \left( \begin{bmatrix} \num{0.75} \\ 0 \end{bmatrix} +
                  \begin{bmatrix} 0 \\ \num{0.25} \end{bmatrix}\right)\right\} \\
                &+\frac{1}{2} \left\{ \left(\begin{bmatrix} 0 \\ 1 \end{bmatrix} \otimes
                  \begin{bmatrix} 0 \\ 1 \end{bmatrix}\right) \left( \begin{bmatrix} 0 \\ \num{0.75} \end{bmatrix} -
                  \begin{bmatrix} \num{0.25} \\ 0 \end{bmatrix}\right)\right\} \\
                &+\frac{1}{2} \left\{\left( \begin{bmatrix} 1 \\ 0 \end{bmatrix} \otimes
                  \begin{bmatrix} 0 \\ 1 \end{bmatrix}\right) \left( \begin{bmatrix} 0 \\ \num{0.75} \end{bmatrix} +
                  \begin{bmatrix} \num{0.25} \\ 0 \end{bmatrix}\right)\right\} \\
                &+\frac{1}{2} \left\{ \left[ \begin{bmatrix} 0 \\ 1 \end{bmatrix} \otimes
                  \begin{bmatrix} 1 \\ 0 \end{bmatrix}\right] \left( \begin{bmatrix} \num{0.75} \\ 0 \end{bmatrix} -
                  \begin{bmatrix} 0 \\ \num{0.25} \end{bmatrix}\right) \right\}.
  \end{split}
\end{equation}

Para reconstruir o estado, aplicando as condições descritas na Tabela~\ref{tab:acao-das-portas} teremos:

\begin{enumerate}
  \item Se a medida realizada for o estado $\ket{00}$, o estado enviado foi
        \[\ket{\psi} = \begin{bmatrix} \num{0.75} \\ 0 \end{bmatrix} + \begin{bmatrix} 0 \\ \num{0.25} \end{bmatrix} = \begin{bmatrix} \num{0.75} \\ \num{0.25} \end{bmatrix}. \]

  \item Se a medida realizada for o estado $\ket{11}$:
        \[
        \begin{bmatrix} 0 \\ \num{0.75} \end{bmatrix} - \begin{bmatrix} \num{0.25} \\ 0 \end{bmatrix} = \begin{bmatrix*}[r] -\num{0.25} \\ \num{0.75} \end{bmatrix*}.
        \]

        Aplicando \(\XXX\):
        \[
        \begin{bmatrix} 0 & 1 \\ 1 & 0 \end{bmatrix} \begin{bmatrix*}[r] -\num{0.25} \\ \num{0.75} \end{bmatrix*} = \begin{bmatrix*}[r] \num{0.75} \\ -\num{0.25} \end{bmatrix*}.
        \]

        Aplicando \(\ZZZ\):
        \[
        \begin{bmatrix*}[r] 1 & 0 \\ 0 & -1 \end{bmatrix*}\begin{bmatrix*}[r] \num{0.75} \\ -\num{0.25} \end{bmatrix*} = \begin{bmatrix} \num{0.75} \\ \num{0.25} \end{bmatrix}.
        \]

        Portanto o estado recuperado é:
        \[
        \begin{bmatrix} \num{0.75} \\ \num{0.25} \end{bmatrix} = \begin{bmatrix} \num{0.75} \\ 0 \end{bmatrix} +  \begin{bmatrix} 0 \\ \num{0.25} \end{bmatrix}.
        \]

  \item Se a medida realizada for o estado $\ket{10}$:
        \[
        \begin{bmatrix} \num{0.75} \\ 0 \end{bmatrix} - \begin{bmatrix} 0 \\ \num{0.25} \end{bmatrix} = \begin{bmatrix*}[r] \num{0.75} \\ -\num{0.25} \end{bmatrix*}.
        \]

        Aplicando \(\ZZZ\):
        \[
        \begin{bmatrix*}[r] 1 & 0 \\ 0 & -1 \end{bmatrix*}\begin{bmatrix*}[r] \num{0.75} \\ -\num{0.25} \end{bmatrix*} = \begin{bmatrix} \num{0.75} \\ \num{0.25} \end{bmatrix}.
        \]

  \item Por último, se a medida realizada for o estado $\ket{01}$:
        \[
        \begin{bmatrix} \num{0.25} \\ 0 \end{bmatrix} +  \begin{bmatrix} 0 \\ \num{0.75} \end{bmatrix} = \begin{bmatrix} \num{0.25} \\ \num{0.75} \end{bmatrix}.
        \]

        Aplicando \(\XXX\):
        \[
        \begin{bmatrix} 0 & 1 \\ 1 & 0 \end{bmatrix} \begin{bmatrix} \num{0.25} \\ \num{0.75} \end{bmatrix} = \begin{bmatrix} \num{0.75} \\ \num{0.25} \end{bmatrix}.
        \]
\end{enumerate}

Todos os estados recuperados são iguais aos estados enviados e portanto a dedução apresentada no Apêndice~\ref{app:matricial} pode ser utilizada como um guia para um programa que automatize tais operações.
