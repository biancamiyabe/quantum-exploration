%%%   ____ ___  _   _  ____ _    _   _ ___
%%%  / ___/ _ \| \ | |/ ___| |  | | | |_ _|
%%% | |  | | | |  \| | |   | |  | | | || |
%%% | |__| |_| | |\  | |___| |__| |_| || |
%%%  \____\___/|_| \_|\____|_____\___/|___|
%%%
%%% TCC de Bianca Miyabe Santos Freitas
%%% Licenciatura em Física - UFSCar, Sorocaba
%%%

\chapter{Conclusões}

A realização deste trabalho possibilitou além de desenvolver uma notação matricial para o protocolo de teletransporte o que facilita a visualização da alteração dos estados como também sua implementação como estrutura de dados para a automatização do processo, aprofundando os estudos em MQ em uma aplicação direta de suas propriedades.

Como um computador puramente quântico ainda está em desenvolvimento, os estudos sobre o tema estão em seu auge e a possibilidade de estudar e desenvolver algoritmos que um dia atuarão nestes dispositivos ressalta a necessidade da constante expansão da ciência. Nos últimos 60 anos, vimos o computador clássico passar de uma versão gigantesca para dispositivos miniaturizados e apenas nos últimos 20 o computador quântico saiu do papel com o desenvolvimento de processadores quânticos. Portanto, é plausível acreditar que existe muito ainda para se desenvolver.

Para além, os resultados deste estudo introdutório se mostraram otimistas para a compreensão de algoritmos quânticos, seja por sua descrição físico-matemática ou por sua representação gráfica na forma dos circuitos quânticos e podem servir de base para a compreensão de sistemas mais complexos e sofisticados, possíveis de ser implementados com a tecnologia quântica.

Espera-se que este trabalho possa servir para a orientação e o estudo dessa temática visto que, por ainda estar em desenvolvimento, possuí aspectos ainda não consolidados e que dependem de novos estudos.
