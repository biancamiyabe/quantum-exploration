%%%
%%%  ____
%%% |  _ \ ___  ___ _   _ _ __ ___   ___  ___
%%% | |_) / _ \/ __| | | | '_ ` _ \ / _ \/ __|
%%% |  _ <  __/\__ \ |_| | | | | | | (_) \__ \
%%% |_| \_\___||___/\__,_|_| |_| |_|\___/|___/
%%%
%%% Baseado em 'Atualizado_em_maio_2021_trabalho_academico_formato_classico.docx'
%%% fornecido pela BSo, UFSCar, campus Sorocaba.
%%%

%%%
%%% RESUMO EM PORTUGUÊS
%%%----------------------------------------------------------------------------
\begin{resumo} % Resumo em PT-BR
  FREITAS, Bianca Miyabe Santos. Computação Quântica: estudo do protocolo de Teletransporte Quântico na presença de ruídos. 2023. Trabalho de Conclusão de Curso (Licenciatura Plena em Física) -- Universidade Federal de São Carlos, Sorocaba, 2023.

  \vspace*{\onelineskip}
  Com a hipótese da utilização da Mecânica Quântica para o processamento de informação, surge uma nova área de estudo chamada Computação Quântica. As pesquisas nesta área remetem ao comportamento e implementação do qubit como unidade básica de informação. Apesar da implementação física de um Computador Quântico já existir, os estudos sobre o comportamento do qubit, bem como dos mecanismos aos quais este pode se submeter se fazem necessários para o desenvolvimento de processadores com maior número e melhor qualidade destas entidades quânticas. Nesse sentido, a proposta deste trabalho consiste na elaboração de uma simulação para o estudo do fenômeno de Teletransporte Quântico utilizando qubits emaranhados, bem como na verificação dos efeitos de possíveis ruídos na transmissão. Os resultados se mostraram promissores para a utilização deste protocolo como material introdutório ao estudo de algoritmos quânticos por evidenciar as operações que ocorrem no sistema quântico, facilitando sua compreensão.

  \vspace{\onelineskip}

  \noindent
  Palavras-chave: Computação Quântica; teletransporte quântico; qubits.
\end{resumo}

\cleardoublepage

%%%
%%% RESUMO EM INGLÊS
%%%----------------------------------------------------------------------------
\begin{resumo}[Abstract] % Resumo em EN
  \selectlanguage{english}

  With the hypothesis of using Quantum Mechanics for information processing, a new area of ​​study called Quantum Computing emerges. Research in this area addresses the behavior and implementation of the qubit as a basic unit of information. Although the implementation of a real Quantum Computer already exists, studies on the behavior of the qubit, as well as the mechanisms to which it can be subjected, are necessary for the development of processors with a greater number and better quality of these quantum entities. In this sense, the purpose of this work consists in the elaboration of a simulation for the study of the phenomenon of Quantum Teleportation using entangled qubits, as well as in the verification of the effects of possible noises in the transmission. The results proved to be promising for the use of this protocol as an introductory material to the study of quantum algorithms by highlighting the operations that occur in the quantum system, facilitating its understanding.
  \vspace{\onelineskip}

  \noindent
  Keywords: Quantum Computing; Quantum teleportation; qubits.
\end{resumo}

\cleardoublepage

%%% end of resumos.tex
