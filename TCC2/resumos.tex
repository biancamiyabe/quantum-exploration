%%%
%%%  ____
%%% |  _ \ ___  ___ _   _ _ __ ___   ___  ___
%%% | |_) / _ \/ __| | | | '_ ` _ \ / _ \/ __|
%%% |  _ <  __/\__ \ |_| | | | | | | (_) \__ \
%%% |_| \_\___||___/\__,_|_| |_| |_|\___/|___/
%%%
%%% Baseado em 'Atualizado_em_maio_2021_trabalho_academico_formato_classico.docx'
%%% fornecido pela BSo, UFSCar, campus Sorocaba.
%%%

%%%
%%% RESUMO EM PORTUGUÊS
%%%----------------------------------------------------------------------------
\begin{resumo} % Resumo em PT-BR
  SOBRENOME, Nome. Título: subtítulo. 20XX. Trabalho de Conclusão de Curso (Licenciatura Plena em [Matemática/Física]) -- Universidade Federal de São Carlos, Sorocaba, 20XX.
  \vspace*{\onelineskip}

  \begin{destaque}
    [A referência acima é opcional quando o resumo estiver contido no próprio documento e deve ficar logo após o título da seção (Resumo).]
  \end{destaque}

  Item obrigatório. Resumo é a apresentação concisa dos pontos relevantes de um documento. O resumo deve ressaltar sucintamente o conteúdo de um texto. A ordem e a extensão dos elementos dependem do tipo de resumo (informativo ou indicativo) e do tratamento que cada item recebe no documento original. O resumo deve ser composto por uma sequência de frases concisas em parágrafo único, sem enumeração de tópicos. Em documento técnico ou científico, recomenda-se o resumo informativo. Convém usar o verbo na terceira pessoa. Convém que, nos trabalhos acadêmicos, os resumos tenham de 150 a 500 palavras. Segundo a Associação Brasileira de Normas Técnicas (ABNT) 6028:2021, as palavras-chave devem figurar logo abaixo do resumo, antecedidas da expressão Palavras-chave, seguida de dois-pontos, separadas entre si por ponto e vírgula e finalizadas por ponto. Devem ser grafadas com as iniciais em letra minúscula, com exceção dos substantivos próprios e nomes científicos.
  \vspace{\onelineskip}

  \noindent
  Palavras-chave: resumo; Associação Brasileira de Normas Técnicas; trabalho acadêmico.
\end{resumo}

\cleardoublepage

%%%
%%% RESUMO EM INGLÊS
%%%----------------------------------------------------------------------------
\begin{resumo}[Abstract] % Resumo em EN
  \selectlanguage{english}

  Item obrigatório. É a versão do Resumo em língua vernácula para um idioma de divulgação internacional.
  \vspace{\onelineskip}

  \noindent
  Keywords: word 1; word 2; word 3.
\end{resumo}

\cleardoublepage

%%% end of resumos.tex
