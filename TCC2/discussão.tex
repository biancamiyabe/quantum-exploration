%%%  ____ ___ ____   ____ _   _ _____ _____
%%% |  _ \_ _/ ___| / ___| | | |_   _| ____|
%%% | | | | |\___ \| |   | | | | | | |  _|
%%% | |_| | | ___) | |___| |_| | | | | |___
%%% |____/___|____/ \____|\___/  |_| |_____|
%%%
%%% TCC de Bianca Miyabe Santos Freitas
%%% Licenciatura em Física - UFSCar, Sorocaba
%%%
\chapter{Discussão dos resultados e perspectivas de estudo}\label{sec:resultados}

Durante a criação da descrição matricial do protocolo apresentado no Apêndice~\ref{app:matricial}, utilizamos matrizes do tipo coluna para representar os estados quânticos e os qubits, pois ambos são vetores. Além disso, pudemos expressar as portas lógicas quânticas como matrizes unitárias, o que permitiu uma implementação eficiente das operações entre as portas e os qubits. Apesar da descrição matricial simplificar o protocolo, desconsiderando a variável temporal, é possível estabelecer uma cronologia das operações, uma vez que algoritmos quânticos podem ser expressos na forma de circuitos.

Durante a construção da descrição matricial, identificamos os seguintes postulados descritos na Seção~\ref{sec:postulados}:
\begin{itemize}
  \item O Postulado~\ref{post:p1} mostra que os qubits podem ser representados pela expressão
        \[
        \ket{\psi} =
        \alpha \begin{bmatrix} 1 \\ 0 \end{bmatrix} +
        \beta \begin{bmatrix} 0 \\ 1 \end{bmatrix},
        \]
        que é uma combinação linear de vetores unitários, representados por matrizes coluna.

  \item O Postulado~\ref{post:p2} evidencia que os operadores quânticos, que são as portas lógicas quânticas, são a única operação capaz de alterar os estados quânticos. Na descrição proposta neste trabalho, observamos que, ao realizar uma operação entre qubits, o estado permanece inalterado, mas ao realizar uma operação entre um qubit e uma porta lógica quântica, o primeiro sempre é modificado. É interessante notar que, a menos que se realize uma operação de medida, o estado de um qubit não é modificado definitivamente ao interagir com uma porta lógica quântica, sendo reversível com a aplicação de uma operação inversa.

  \item O Postulado~\ref{post:p4} foi constatado nas operações que envolvem todos os estados do sistema, como na aplicação da porta \(\CNOT\). O estado geral do sistema é obtido com o produto tensorial entre todos os estados que o compõem, o que nos permite descrever detalhadamente as possibilidades de colapso após uma medida em função dos estados obtidos.
\end{itemize}

A possibilidade de observar os postulados na descrição matricial justifica e valida sua construção, de modo que, apesar de simplificada, ela não viola nenhum dos postulados propostos. Além disso é possível acompanhar a evolução do sistema com o circuito lógico da Figura~\ref{fig:protocoloteletransporte} com a ordem de aplicação das portas e os estados dos qubits sendo modificados.

Sobre o desenvolvimento do protocolo em Python, a utilização da biblioteca \py{numpy} permitiu uma implementação mais eficiente e simplificada, enquanto a função \py{kron} otimizou as operações de produto tensorial necessárias durante todo o protocolo.

A validação dos resultados ocorreu comparando-os com a descrição teórica do protocolo de teletransporte quântico, e confirmou-se que o simulador reproduziu corretamente o estado final do sistema para todos os valores de amplitude de probabilidade de $\alpha$ e $\beta$, com precisão de até 5 casas decimais.

Os resultados da Tabela~\ref{tab:resultadoruidos} podem ser comparados com os descritos em~\ref{tiposruidos}, e a inversão de estados é confirmada para o ruído \textit{bitflip} e a de fase para o ruído \textit{phaseflip}, o que demonstra que a presença de ruídos compromete o estado quântico teletransportado, o que é relevante para a implementação prática de sistemas de teletransporte quântico.

Embora o protocolo desenvolvido seja introdutório, sua compreensão permite o desenvolvimento de trabalhos mais complexos, como a implementação de um protocolo que considere a evolução temporal como variável e possibilite a descrição mais precisa dos efeitos de ruídos no sistema, ou ainda, a descrição de qubits utilizando sua função de onda e operando sobre ela.

Vale ressaltar que existem algumas ferramentas já consolidadas para o estudo e aperfeiçoamento de algoritmos quânticos, como o Qiskit\footnote{https://qiskit.org/}. O Qiskit é uma plataforma de código aberto desenvolvida pela IBM para programação e simulação de circuitos quânticos, amplamente utilizada em pesquisas e aplicações práticas em diversas áreas, como criptografia, simulação química e inteligência artificial.

Com este estudo introdutório, algumas perspectivas ficam delineadas para estudos futuros, como a implementação do mesmo protocolo utilizando o Qiskit em um ambiente clássico e posteriormente em um ambiente quântico para verificar possíveis divergências nos resultados. Outra perspectiva futura é o mapeamento de erros para o desenvolvimento e aperfeiçoamento de algoritmos de correção de erros quânticos, na tentativa de tornar a implementação de um protocolo de teletransporte real mais efetiva para a evolução da computação quântica.
