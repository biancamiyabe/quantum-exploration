%%%
%%%  ___ _                     __  __ _           _
%%% | _ |_)__ _ _ _  __ __ _  |  \/  (_)_  _ __ _| |__  ___
%%% | _ \ / _` | ' \/ _/ _` | | |\/| | | || / _` | '_ \/ -_)
%%% |___/_\__,_|_||_\__\__,_| |_|  |_|_|\_, \__,_|_.__/\___|
%%%                                     |__/
%%%
%%% PROJETO DE TCC: Documento modelo em LaTeX, padrão ABNT usando ABNTeX2
%%%
%%% ----------------------------------------------------------------------------
%%%
%%% Para compilar, na linha de comando:
%%%
%%%
%%%    latexmk -lualatex skeleton_tcc_proj.tex
%%%
%%% ----------------------------------------------------------------------------
%%%
%%% By Cantão! <rfcantao@gmail.com>
%%% Baseado em template do Prof. James A. Souza
%%%

\documentclass[12pt,oneside,brazil,hidelinks,article,sumario=tradicional,a4paper]{abntex2}

% Pacotes padrão da American Mathematical Society
\usepackage{amsmath}
\usepackage{amssymb}
\usepackage{mathtools}

%%%
%%% São usadas as fontes:
%%% - Crimsom Pro: texto principal
%%% - Noto Sans: sem serifa, usada nas seções, subseções, etc
%%% - DejaVu Sans Mono: monoespaçada para listagems, URLs, etc
%%%
\usepackage{fontspec}
% CrimsonText é similar à Minion Pro, que é comercial
\setmainfont{CrimsonPro}[
  Path           = ./fonts/,
  Extension      = .ttf,
  UprightFont    = *-Regular,
  BoldFont       = *-Bold,
  ItalicFont     = *-Italic,
  BoldItalicFont = *-BoldItalic,
  SmallCapsFont  = AlegreyaSC-Regular
]
% NotoSans é similar à Myriad Pro, que é comercial
\setsansfont{NotoSans}[
  Path           = ./fonts/,
  Extension      = .ttf,
  UprightFont    = *-Regular,
  BoldFont       = *-Bold,
  ItalicFont     = *-Italic,
  BoldItalicFont = *-BoldItalic
]
\setmonofont{DejaVuSansMono}[
  Path           = ./fonts/,
  Extension      = .ttf,
  UprightFont    = *,
  BoldFont       = *-Bold,
  ItalicFont     = *-Oblique,
  BoldItalicFont = *-BoldOblique,
  Scale          = MatchLowercase
]
\usepackage[math-style=TeX]{unicode-math}
%\setmathfont{Asana-Math.otf}[Scale=MatchLowercase]
%\setmathfont{STIXTwoMath-Regular.otf}[Scale=MatchLowercase]
%\setmathfont{texgyredejavu-math.otf}[Scale=MatchLowercase]


%%%
%%% Idiomas: Português e Inglês
%%%
%%% Certifique-se de ter os pacotes corretos (GNU/Linux):
%%%   sudo apt install texlive-lang-en texlive-lang-portuguese texlive-xetex
%%%
%%% O TeXLive (recomendo no caso do Windows) em geral já tem esses pacotes instalados
%%%
\usepackage{polyglossia}
\setmainlanguage{portuges}
\setotherlanguage{english}
\usepackage{csquotes}

%%%
%%% Outros pacotes úteis
%%%
\usepackage{graphicx}  % Figuras em vários formatos (png, pdf)
\usepackage{color}     % Cores
\usepackage{pgfgantt}  % Cronogramas
\usepackage{multicol}  % Documentos com várias colunas
\usepackage{booktabs}  % Tabelas profissionais
\usepackage[final]{microtype}
\usepackage{siunitx}   % Pacote para unidades físicas (recomendo muito!)
\sisetup{output-decimal-marker={,}}
\sisetup{mode=text}

%%%
%%% CORES MANEIRÍSSIMAS
%%%
\definecolor{green}{RGB}{174,226,57}    % colourlovers.com/palette/46688/fresh_cut_day (atomic bikini)
\definecolor{yellow}{RGB}{237,229,116}  % colourlovers.com/palette/937624/Dance_To_Forget (Give Your Heart)
\definecolor{orange}{RGB}{255,164,70}   % colourlovers.com/palette/1107950/Indecent_Proposal (exotic orange)
\definecolor{onlyorange}{RGB}{191,77,40}% colourlovers.com/palette/953498/Headache (Only Orange)
\definecolor{cyan}{RGB}{108,243,213}    % colourlovers.com/palette/940927/Acused (wrong cyan)
\definecolor{red}{RGB}{199,8,8}         % colourlovers.com/palette/79468/LipstickOnHisCollar (Flagged Down)
\definecolor{melon}{RGB}{209,49,92}     % colourlovers.com/palette/2350697/This_is_for_YOU! (melon)
\definecolor{blue}{RGB}{62,122,162}     % colourlovers.com/palette/794774/be_here_for_me (kreuger)
\definecolor{berry}{RGB}{95,13,59}      % colourlovers.com/palette/117122/BurberryTenderTouch (BurberryTender)
\definecolor{violet}{RGB}{87,30,240}    % From the original template (Violet Thanos)
\definecolor{hymnroyale}{RGB}{42,4,72}  % colourlovers.com/palette/81885/Hymn_For_My_Soul (Hymn Royale)
\definecolor{think}{HTML}{607848}       % colourlovers.com/palette/38562/Hands_On
\definecolor{gray}{HTML}{444444}
\definecolor{intelligentsia}{RGB}{7,69,111} % colourlovers.com/palette/4792400/Intelligentsia (Intelligentsia circle)

%%%
%%% Informações do PDF
%%%
\makeatletter
\hypersetup{%
  pdftitle={\@title},
  pdfauthor={\@author},
  pdfsubject={\imprimirpreambulo},
  pdfcreator={XeLaTeX with abnTeX2},
  pdfkeywords={tcc}{licenciatura em física}{projeto de pesquisa},
  colorlinks=true,    % false: boxed links; true: colored links
  linkcolor=gray,     % color of internal links
  citecolor=gray,     % color of links to bibliography
  filecolor=gray,     % color of file links
  urlcolor=gray,
  bookmarksdepth=4
}
\makeatother

%%%
%%% Bibliografia
%%%
%%% Aqui estamos usando por padrão um programa chamado 'biber'. Ele é o responsável
%%% por converter o arquivo 'skeleton.bib' nas referências formatadas no padrão
%%% ABNT.
%%%
%%% Para saber mais:
%%% https://www.overleaf.com/learn/latex/Bibliography_management_in_LaTeX
%%%
\usepackage[backend=biber,style=abnt,noslsn,repeatfields]{biblatex}
\addbibresource{./tcc.bib}

%%% Função seno em PT-BR
\DeclareMathOperator{\sen}{sen}

%%%
%%% Configurações do documento para ABNTeX2
%%%
\titulo{Simulação de Circuitos Quânticos para o estudo de ruídos no fenômeno de teletransporte quântico}
\autor{Bianca Miyabe Santos Freitas}
\orientador{Prof. Dr. Renato Fernandes Cantão}
\instituicao{%
  UNIVERSIDADE FEDERAL DE SÃO CARLOS --- \textsl{CAMPUS} SOROCABA
  \par
  CENTRO DE CIÊNCIAS E TECNOLOGIAS PARA A SUSTENTABILIDADE
  \par
  DEPARTAMENTO DE FÍSICA, QUÍMICA E MATEMÁTICA}
\tipotrabalho{Projeto de Trabalho de Conclusão de Curso}
\preambulo{Projeto de Trabalho de Conclusão de curso apresentado ao curso de Licenciatura Plena em Física da Universidade Federal de São Carlos, \textsl{Campus} Sorocaba, como requisito para a conclusão da disciplina TCC 1.}
\local{Sorocaba}
\data{Abril, 2022}

% Modelo sugerido pela UFSCar
\renewcommand{\imprimircapa}{%
  \begin{capa}%
    \centering
    {\imprimirinstituicao\vfill}

    {\ABNTEXchapterfont\large\imprimirautor}

    \vfill
    {\ABNTEXchapterfont\bfseries\LARGE\imprimirtitulo}
    \vfill

    \large\imprimirlocal

    \large\imprimirdata

    \vspace*{15mm}
  \end{capa}
}

\begin{document}

%%% Elementos pré-textuais (capa, sumário, etc)
\pretextual
\imprimircapa
% \imprimirfolhaderosto

\begin{resumo} % Resumo em PT-BR
  O resumo é um mini projeto de pesquisa onde deve conter uma breve descrição de toda a proposta, desde a área de pesquisa, referencial teórico no caso de ensino, teoria, experimento, objetivos a serem alcançados, metodologia e resultados esperados. Este deve indicar quais questões vocês (aluno e orientador), como pesquisadores, pretendem responder, permitindo ao leitor acessar rapidamente a ideia básica e os objetivos de sua proposta. O resumo deve ser escrito em um único parágrafo sem espaçamento no início, como nesse exemplo.
  \vspace{\onelineskip}

  \noindent
  \textbf{Palavras-chave}: são aquelas que mais aparecem no texto especificando o fenômeno em estudo, a área de concentração, alguma técnica específica, etc. Pelo menos três palavras e no máximo cinco, separadas por ponto (Palavra 1. Palavra 2. Palavra 3).
\end{resumo}

\begin{resumo}[Abstract] % Resumo em EN
  \begin{otherlanguage*}{english}
    Same as above, but in English.
    \vspace{\onelineskip}

    \noindent
    \textbf{Keywords}: Word 1. Word 2. Word 3.
  \end{otherlanguage*}
\end{resumo}

%%% Caso já seja um trabalho final
% \listoffigures*
% \clearpage

% \listoftables*
% \clearpage

% \tableofcontents*
% \clearpage

%%% Pular página
% \clearpage

%%% Elementos textuais (o documento em si)
\textual%

%%%
%%% INTRODUÇÃO
%%%
\section{Introdução}\label{sec:intro}

Na atualidade, a presença de computadores em nossa rotina é cada vez mais corriqueira, visto que, desde efetivamente nossos computadores, até nossos eletrodomésticos possuem cada vez mais a capacidade de processamento de informação.

Desde a proposição da Máquina de Turing, em 1936, onde o mesmo apresenta um aparato lógico para o processamento de informação e ainda define as unidades conhecidas como bits (binary digit), até os computadores mais modernos, a maneira de se processar a informação é basicamente a mesma. A ideia central de um computador, e aqui vamos classificá-lo como clássico, consiste em realizar operações lógicas de combinações de bits cujos valores podem ser 0 (zero) ou 1 (um), é assim que o computador que escrevo esse trabalho consegue traduzir e apresentar o que digito no teclado na tela do mesmo. 

É fato que, ao longo dos anos, a velocidade com que as operações são realizadas em um computador aumentou devido ao aperfeiçoamento das unidades de processamento (processadores/microprocessadores), o que trouxe por consequência o aumento das possibilidades de resolução de problemas utilizando a programação.(1)

Em linhas gerais, portanto, um computador clássico utiliza um sistema binário, cuja unidade representativa é o bit, para decodificar e processar a informação. Esse processo porém, possui um limite físico, associado a quantidade de bits processados e a velocidade para execução de determinadas tarefas. Esse limite que excede a capacidade de processadores clássicos se encontra principalmente nos estudos de fronteira das ciências da natureza.(2)

Uma proposta para a resolução desse limite foi apresentada por Richard P. Feymann na década de 1980, de maneira a apenas apresentar uma hipótese: Seria possível a construção de um dispositivo de processamento que se baseasse no modo que a natureza se expressa em sua essência? A resposta para essa pergunta, surgiu anos depois com a criação do computador de arquitetura quântica.(3)

Para compreender melhor como este funciona, vale ressaltar alguns tópicos que descrevem a teoria em que esses dispositivos se baseiam, a Mecânica Quântica.

%PARAGRAFO SOBRE MECÂNICA QUÂNTICA E ARQUITETURA QUÂNTICA (QUBITS) (PROVAVELMENTE BASEADO NO LIVRO DO MARKOV)(griffths)

Utilizando desses postulados, temos uma sequência de estudos que culmina na efetiva consolidação do computador quântico a se destacar:
\begin{itemize}
    \item  David Deustch (1985), propõe matematicamente o primeiro computador quântico universal, conhecido também com Maquina de Turing Quântica.%(VER SE VALE A PENA EXPLICAR UM POUCO MAIS SOBRE ELA, TALVEZ EM COMPARAÇÃO COM A CLÁSSICA, TALVEZ SEJA ALGO PARA O TCC EM SÍ); se for bem resumido ok!
    
    \item  Peter Shor (1994), criou o primeiro programa essencialmente quântico, ou seja, ele não poderia ser executado em um computador clássico. Este programa, conhecido como Algoritmo de Shor reduziria o tempo de fatoração de números grandes de possíveis meses para apenas segundos caso fosse utilizado em um computador de arquitetura quântica.
    
    \item  Em 1999 o MIT apresenta o primeiro protótipo de um computador quântico real % (BUSCAR MAIS INFORMAÇÕES SOBRE)
    
    \item  E empresa D-Wave apresenta, em 2007, o primeiro computador essencialmente quântico.
\end{itemize}
 
Com a efetiva construção física de um computador de arquitetura quântica, as possibilidades para resolução de problemas ainda não estavam totalmente definidas. Isso porque, os primeiros processadores quânticos conseguiam utilizar apenas uma quantidade muito pequena de qubits emaranhados, o que, apesar de útil, apenas reduz o tempo de resposta de um programa, deixando alguns problemas mais complexos, ainda sem solução.

A evolução da Computação Quântica consiste, portanto, em não apenas aumentar o número desses qubits processados simultaneamente, mas no aumento da qualidade dos mesmos no processamento da informação.

%PARAGRAFO SOBRE AS PROMESSAS DA IBM, GOOGLE E MICROSOFT SOBRE QUANTIDADE DE QUBITS SIMULTÂNEOS
De modo a aprimorar as relações de qubits e se preparar para as possibilidades apresentadas pelas empresas acima, a utilização de simulações para o estudo de temas de Computação Quântica é cada vez mais necessário. Uma simulação é essencialmente diferente da realidade, mas tenta, o melhor possível, se aproximar desta o que se torna útil na previsão de possíveis problemas na evolução de sistemas quânticos.

%PARÁGRAFO SOBRE O FUNCIONAMENTO DE UMA SIMULAÇÃO QUÂNTICA
Algumas áreas exploram com afinco as possibilidades da simulação quântica, mesmo que em computadores de arquitetura clássica, para prever demandas de quando esta tecnologia avançar o bastante tornando possível, por exemplo, a transmissão de informação utilizando propriedades quânticas, o que é chamado de Teoria da Informação Quântica.

%PARAGRAFO SOBRE A TEORIA DA INFORMAÇÃO QUÂNTICA
Uma situação onde a teoria da informação e a mecânica quântica se unem é no fenômeno de teletransporte quântico.
%PARÁGRAFO SOBRE TELETRANSPORTE QUÂNTICO (EXPERIMENTO DA ALICE E DO BOB), PROBLEMAS POSSÍVEIS NO PROCESSO (BREVE DESCRIÇÃO DO QUE É UM RUÍDO E DA CORRELAÇÃO DO APERFEIÇOAMENTO DAS MAQUINAS QUÂNTICAS EM NÃO APENAS NÚMERO DE QUBITS MAS NA QUALIDADE DESTES)

Diante do apresentado, a proposta de estudo deste trabalho consiste na elaboração de uma simulação, em um computador de arquitetura clássica, do fenômeno de teletransporte quântico para o estudo da transmissão de informação no mesmo.

%\begin{citacao}
 % \color{red}
%  Citação direta com mais de 3 linhas, deve ser digitada com letra em tamanho menor da que foi utilizada no texto, sem aspas, e com recuo de 4 cm da margem esquerda e espaçamento simples. Deve-se mencionar, além do sobrenome do autor e data de publicação, o(s) número(s) da página de onde retirou a citação \cite[p. 32]{Collobert2011}.
%\end{citacao}

%{\color{red}Nas citações indiretas não são colocadas ``\emph{aspas}'' e nem o número da página \cite{Collobert2011}.}

Evitem o uso de citações diretas para o caso de definições de conceitos físicos, técnicas experimentais ou mesmo para opiniões de outros autores. Na maioria das vezes não é necessário fazer esse tipo de citações. Dê preferência para as citações indiretas. Leiam o texto referido até vocês entenderem e poderem expressar as ideias e conceitos por trás dos mesmos com as próprias palavras. Com isso vocês evitam que o texto de vocês seja composto por vários recortes de dizeres de outros autores. A citação direta é usualmente utilizada quando queremos expressar exatamente o que o autor citado escreveu.

%No caso de ser necessário fazer uma citação indireta, como uma citação de uma citação, utilizem a palavra latina ``\emph{apud}'', cujo significado é ``\emph{junto a, perto de, em}'', mas que no contexto científico e acadêmico é utilizado como sinônimo de ``\emph{citado por}''. Especificamente, quando você utilizar o apud significa que você está citando a obra de um autor que você não leu, mas que foi citada na obra de outro autor cuja obra você leu e estudou. Isso ocorre comumente quando livros muito antigos e que não estão disponíveis em formato eletrônico ou mesmo impresso são citados. Isso pode ocorrer também quando vocês não têm acesso à revista (acesso pago) em que o artigo original foi publicado. Veja o exemplo:

%De acordo com \textapud[p. 148]{Collobert2011}[p. 8]{Souza2010} a estabilidade de um foguete de garrafas PET pode ser obtida se o centro de massa do foguete estiver em torno de \SI{1.5}{\centi\meter} acima do seu centro de pressão.

%As normas exigidas neste trabalho para a citação de livros e revistas são impostas automaticamente pelo estilo de ABN{\TeX}2. Vale ressaltar que não existe o padrão mais correto para a escrita de projetos e artigos científicos, o que existe são padrões e normativas exigidos. Um dos objetivos deste trabalho é aprender a seguir um padrão.

%Apesar de toda a riqueza da língua portuguesa e inglesa para a escrita expositiva, o estilo da escrita científica é puramente técnico e conciso. Esta deve ser o mais claro possível, evitem utilizar uma linguagem barroca, rebuscada, transforme o texto de vocês em algo poderoso e convincente com simples e poucas palavras, evitem redundâncias. Isso é válido para projetos e artigos científicos, dissertações de mestrado, teses de doutorado e relatórios de prática. {\color{red}Com relação ao tempo verbal para a escrita do projeto, este deve estar no futuro, pois o projeto consiste de uma proposta que será executada no decorrer da graduação de vocês e finalizada, provavelmente, quando vocês cursarem a disciplina de TCC 2.} 

%{\color{red}Mesmo que vocês escrevessem o projeto sozinhos, um único autor, utilizem o que chamamos de ``\emph{eu formal}'', que consiste em se referir às atividades de pesquisa na primeira pessoa do plural.} É como se vocês estivessem convidando o leitor para participar do projeto, por exemplo, ``\emph{Neste projeto desenvolveremos\ldots executaremos\ldots pesquisaremos sobre\ldots etc.}'' Em relatórios de prática é usual utilizarmos o tempo verbal na terceira pessoa, como ``\emph{foi feito\ldots foi analisado\ldots foi executado\ldots foi estabelecido\ldots etc.}'' Neste caso não existe autoria sobre o que foi desenvolvido, pois trata-se de uma reprodução, geralmente orientada através de um roteiro. Em um projeto ou artigo científico é importante exaltar a autoria do trabalho de vocês. Terceira pessoa será utilizada para os trabalhos de outros autores, pois vocês precisarão citar o que já foi feito por outros pesquisadores no tema escolhido. Quando vocês escrevem nós, fica claro o que foi realmente feito por vocês. Esta é uma tendência que já está sendo sugerida em várias revistas e jornais do mundo todo.

%\noindent
%\textbf{\color{red}IMPORTANTÍSSIMO:} Quando vocês forem utilizar informações de outras fontes, como definições ou técnicas experimentais específicas evitem se remeter ao texto original diretamente. Ou seja, não copiem na íntegra as frases de outros autores. Isso é plágio! No Brasil é considerado como crime pelo código penal (Art. 184) e em outros países trata-se de uma prática antiética e extremamente desrespeitosa. Para vocês se tornarem autores profissionais, competentes e respeitados leiam o artigo, revista ou livro até entenderem a ideia que vocês pretendem abordar na proposta, descrevendo a mesma com as próprias palavras. A ideia pode ser a mesma, mas cada um tem sua maneira de se expressar. É esta a razão de termos diversos autores de livros didáticos falando sobre os mesmos tópicos de um determinado assunto. Se vocês não forem capazes disso é porque o assunto ainda não foi entendido, o que significa que mais estudos precisam ser conduzidos. Em ciência procedemos dessa forma, o entendimento sobre qualquer assunto é efetivado quando vocês são capazes de processar a informação de modo a poder expressar os conceitos com as próprias palavras.

%%%
%%% OBJETIVOS
%%%
\section{Objetivo(s) da Proposta}\label{sec:objs}

O objetivo principal deste trabalho consiste em realizar um estudo acerca dos efeitos de possíveis ruídos no fenômeno de teletransporte quântico, utilizando uma simulação de circuitos quânticos em um computador de arquitetura clássica. Para atingir tal proposta, pretende-se:
\begin{enumerate}
\item Definir o conceito de Teletransporte Quântico;
\item Delinear as situações onde o fenômeno é utilizado;
\item Elencar os tipos de ruídos que podem interferir na transmissão de informação durante o fenômeno;
\item Estruturar as portas lógicas quânticas a serem simuladas para o presente estudo;
\item Estabelecer uma sequência de testes com os algoritmos de ruído;
\end{enumerate} 

\subsection{Nome do tópico 1}\label{obj1}

%Descreva a importância do tópico 1 para a execução do seu projeto de pesquisa \cites{vene2008}{Collobert2011}{braess2007}.


%\subsection{Nome do tópico 2}\label{obj2}

%Descreva a importância do tópico 2 para a execução do seu projeto de pesquisa e assim por diante.

%Caso vocês e os respectivos orientadores prefiram iniciar uma nova seção para isso, como por exemplo: \emph{Fundamentação Teórica} ou algum título específico sobre o que será abordado no projeto não tem problema. O importante é que o projeto fique bem organizado e o assessor capte com facilidade a proposta e os principais conceitos a serem abordados no trabalho. O uso de subseções na seção de objetivos é apenas sugestivo.

%\subsection{Nome do tópico 3}\label{obj3}

%E por ai vai\ldots

%%%
%%% METODOLOGIA
%%%
\section{Metodologia}

Descreva sua proposta metodológica de maneira detalhada seja ela experimental, através de técnicas específicas, ou teórica, desenvolvida matematicamente através de modelos analíticos ou simulações computacionais. Quais processos ou materiais vocês pretendem utilizar? Que tipos de equipamentos serão necessários?

%\noindent
%\textbf{\color{red}IMPORTANTE:} Se for necessário na apresentação de quaisquer tópicos listados acima o uso de figuras para melhorar o entendimento da mesma, siga o exemplo da Figura~\ref{fig1} abaixo. Lembrando que não existe um padrão geral para isso. Seguiremos as normas da ABNT para o projeto. Siga a formatação dos exemplos abaixo.

%{\color{red}Figuras ilustrativas podem ser coloridas ou em preto e branco. As ilustrações devem ser esclarecedoras, nítidas e com boa resolução (300 dpi ou mais). Se houver palavras ou números na figura faça com que o tamanho da fonte seja o mais próximo possível do tamanho da fonte do texto. Todas as Figuras devem ser numeradas e citadas no texto. ESTAS DEVEM SER ESCLARECEDORAS, NÍTIDAS E COM BOA RESOLUÇÃO.}

%\begin{figure}[ht!]
%  \centering
%  \caption{Esquema ilustrativo mostrando o aparato utilizado para colocar a vela em movimento retilíneo acelerado através da queda de uma massa de \(m = \SI{50}{\gram}\). O sistema é composto por (1) mesa, (2) plataforma de madeira, (3) carrinho de plástico, (4) base de madeira, (5) vela, (6) tubo de vidro aberto, (7) barbante, (8) roldana de varal de roupas, (9) cilindro de aço e (10) régua para liberar o sistema.}\label{fig1}
%  \includegraphics[width=0.65\textwidth]{fig1.png}
%  \fonte{\textcite[p. 37]{Souza2010}.}
%\end{figure}

%Outro exemplo é a ilustração de um pêndulo simples, mostrado na Figura~\ref{fig2} {\color{red}(citar as figuras no texto desta forma)}, em que uma massa \(m\) presa a um fio de comprimento \(l\) oscila em torno de seu ponto de equilíbrio com ângulo \(\theta\) sob a ação da força da gravidade. Note que a legenda da Figura~\ref{fig2} contém basicamente as mesmas informações descritas no parágrafo acima, reforçando a ideia de que uma Figura deve ser independente do texto.

%\begin{figure}[ht!]
 % \centering
%  \caption{Pêndulo simples, mostrando uma massa \(m\) presa a um fio de comprimento \(l\) oscilando com ângulo \(\theta\) em torno do ponto de equilíbrio.}\label{fig2}
%  \includegraphics[width=0.4\textwidth]{pendulo.png}
%  \fonte{Elaborada pelo autor.}
%\end{figure}

%\textcolor{red}{\textbf{Observação:} As figuras, tabelas, ilustrações, etc., devem estar juntas com suas legendas e suas fontes na mesma página. A figura não pode ficar em uma página e sua legenda ou fonte em outra.}

Se for necessário incluir alguma foto tirada por vocês mesmos ou uma foto retirada da internet siga o exemplo da Figura~\ref{fig3}.

Quando for necessário discutir ou apresentar equações em qualquer seção do projeto, estas devem ser numeradas e citadas no texto de acordo com o número. Veja o exemplo abaixo:

O período de oscilação \(T\) de um pêndulo simples pode ser calculado através da seguinte expressão:

\begin{equation}\label{periodo}
    T = 2 \pi \sqrt{\dfrac{l}{g}},
\end{equation}
sendo $l$ o comprimento do fio que sustenta a massa $m$ do pêndulo e $g$ a aceleração da gravidade.

\noindent
{\textbf{\color{red}IMPORTANTE:} Todos os parâmetros das equações devem ser explicados no texto, como neste exemplo.}

Se o comprimento $l$ do fio for dado podemos utilizar a Equação~\eqref{periodo} ({\color{red}citar as equações no texto desta forma}) para determinar a aceleração da gravidade através da medida do período de oscilação do pêndulo.

%\begin{figure}[ht!]
%  \centering
%  \caption{Tubo de chama de onda estacionária.}\label{fig3}
%  \vspace{1mm}
%  \includegraphics[width=0.6\textwidth]{Chamas.jpg}
%  \fonte{JUNIOR, G. \underline{Usando Física para criar um incrível experimento com Fogo e Música.} Disponível em: \url{http://marteeparaosfracos.blogspot.com/2014/04/usando-fisica-para-criar-um-incrivel.html}. Acesso em: 16 nov. 2018.}
%\end{figure}

%%%
%%% CRONOGRAMA
%%%
\section{Cronograma de Atividades}

Este tópico é importante para vocês mostrarem aos assessores que avaliarão o projeto, ou no caso para o professor-orientador da disciplina de TCC 1, como vocês pretendem executar a proposta diante do prazo disponível ou determinado. Listem as tarefas que serão executadas em determinado período como pesquisa bibliográfica, busca de materiais para realização do experimento, montagem ou construção do experimento, análise dos resultados obtidos, etc. Esta pode ser feita através de uma tabela. Se caso for necessário incluir outras tabelas na proposta de pesquisa siga o modelo da Tabela~\ref{tabela1} abaixo.

\begin{table}[ht!]
  \centering
  \caption{Cronograma tentativo de atividades.}\label{tabela1}
  \begin{tabular}{cc}
    \toprule
    \textbf{\emph{Período}} & \textbf{\emph{Atividades}}\\
    \midrule
    3 primeiros meses      & Aquisição de materiais para a realização do experimento;        \\
    2º semestre de 2021    & Montagem e testes do experimento;                               \\
    E assim por diante ... & O período pode ser especificado em semanas, meses ou semestres. \\
    \bottomrule
  \end{tabular}
  \fonte{Elaborada pelo autor.}
\end{table}

É importante que o cronograma tentativo apresente de maneira clara o início das atividades, sua execução e a finalização das mesmas, com uma previsão de discussão dos resultados para a escrita da monografia final e a defesa do TCC.

\begin{table}[ht!]
  \begin{center}
    \caption{Cronograma de desenvolvimento do projeto. Cada linha corresponde
      a um objetivo específico da Seção~\ref{sec:objs}. Cada coluna
      corresponde a um mês.}\label{tab:cronog}
    \begin{ganttchart}[hgrid,
                      vgrid,
                      bar/.append style={fill=blue!10,draw=blue},
                      bar top shift=0.1,
                      bar height=0.8,
                      x unit=20mm,
                      y unit chart=6mm]{1}{6}
      \gantttitle{Meses}{6} \\
      \gantttitlelist{1,...,6}{1} \\
      \ganttbar{Seção~\ref{obj1}}{1}{2} \\
      \ganttbar{Seção~\ref{obj2}}{2}{3} \ganttbar{}{5}{6} \\
      \ganttbar{Seção~\ref{obj3}}{4}{6} \\
    \end{ganttchart}
    \fonte{Elaborada pelo autor.}
  \end{center}
\end{table}

% Imprime a bibliografia
\clearpage
\printbibliography[heading=subbibliography]

\end{document}

%%% end of skeleton_tcc_proj.tex
